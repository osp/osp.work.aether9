% \setupoutput[pdftex]



% DEBUG {{{
    % uncomment those 2 lines to show the layout and the grid
    %\showframe 
    %\showgrid
% }}}}

% UTF-8 {{{
    \enableregime[utf-8]              % Enters unicode 
    %\setupencoding[default=uc]       % Uses unicode fonts only
% }}}

% LANGUAGE {{{
    \setuplanguage[en][
        leftquote=\upperleftdoublesixquote,
        rightquote=\upperrightdoublesixquote,
    ]
    \mainlanguage[en]
    \setcharacterspacing [frenchpunctuation]
% }}}

% LAYOUT {{{
    % General layout
    \definepapersize[sheet][width=165mm,height=255mm]         
    \setuplayout[
        backspace=21mm,
        topspace=12mm,
        width=138mm,
        height=225mm,
        leftmargin=15mm,
        rightmargin=12mm,
        header=0mm, % removes header
        footer=6mm, 
        grid=yes, % activates baseline grid
        ]
    \definepagebreak[emptyodd][yes,header,footer, odd]
    \definepagebreak[emptyeven][yes,header,footer, even]
    \setuphead[chapter][page=emptyodd]
    % Left margin
    \setupinmargin[
        %align={outer, nothyphenated, verytolerant, stretch},
        location=left,
        align=left,
        stack=yes,
        style={\setupbodyfont[crimsontext, 9pt] \setupalign[hyphenated]},
        before={\blank\startalignment[inner]},
        after={\stopalignment},
        line=0.5,
        ]
    % Indented block
    \setupnarrower[left=3mm, right=0mm, middle=8mm]
    % Folio
     \setuppagenumber[number=3]
    \setuppagenumbering[location=footer, style={\setupbodyfont[karlaregular, 6pt]}] 
% }}}

% LOAD FONTS {{{
    \starttypescript [serif] [crimsontext] 
        \definefontsynonym [Sans]            [name:crimsontext] [handling=pure]
    \stoptypescript
    \starttypescript [sans] [karlaregular] 
        \definefontsynonym [Sans]            [name:karlaregular] [handling=pure]
    \stoptypescript
    \starttypescript [sans] [consolamono] 
        \definefontsynonym [Mono]            [name:consolamono]
    \stoptypescript

    \definetypeface [crimsontext] [ss] [serif] [crimsontext]
    \definetypeface [karlaregular]  [ss] [sans]  [karlaregular]
    \definetypeface [consolamono]  [ss] [mono]  [consolamono]
% }}}

% TYPOGRAPHY {{{
    \setupbodyfont[crimsontext, ss, 9pt]
    \definebodyfontenvironment[default][a=1.5, b=1.75, c=2,d=3.4] % defines higher sizes of fonts 
    \setupinterlinespace[11pt] 
    \setupinteraction[state=start, color=black] % activate links
    \setupnote[footnote][location=text, way=bypage, numberconversion=set 2, align={hyphenated}]
    \setupnotedefinition[footnote][location=left, before={\startalignment[flushleft]}, after=\stopalignment] 
% }}}

% COLORS {{{
    \setupcolors[rgb=no,cmyk=yes,state=start,overprint=yes]
    \definecolor[blackos][c=0,m=0,y=0,k=1] 
% }}}


% JUSTIFICATION {{{
    \setuptolerance[verytolerant] % justification settings
    \interlinepenalty=8000 % tries to avoid widows
    \widowpenalty=8000 % tries to avoid widows
    \clubpenalty=8000 % tries to avoid widows
    \doublehyphendemerits=1000000 % tries to avoid consecutive hyphenations
    %\setupfonthandling [hz] [min=50,max=50,step=10]
    %\setupalign[hz, hanging]
% }}}


\starttext
\subject[mail:0]{\ss0 - remote fairytales - performance invitation}{
\it{bk bk\high{\ref[p][mail:1].1, \ref[p][mail:2].2, \ref[p][mail:3].3, \ref[p][mail:9].9, \ref[p][mail:10].10, \ref[p][mail:29].29, \ref[p][mail:34].34, \ref[p][mail:65].65} --- 06.06.2007}

\tf{
Dear aethernauts

the next performance of the aether9 group will be on Thursday 14th of 
june, online and in brussels
at 7 PM in europe = 10 AM california = noon in colombia
in an audience packed gallery in the center of  Brussels (n3krozoft Ltd 
will occupy this space for 3 days (see the attached flyer). the 
performance and the exhibit is called
"remote fairytales, contes lontains, fabel op afstand"
(last expression is flemmish - brussels is bilingual french/flemish. 
flemish is very very close to dutch).

the idea for this performance:

1) on location in brussels, next to the screen, there will be 2 persons 
on stage:
- Deirdre Foster (she is a storyteller and actress from Geneva) - she 
will tell her version on the classical tale (from the Grimm brothers) 
"La jeune fille sans mains" (The girl without hands --> 
http://en.wikipedia.org/wiki/The\_Girl\_Without\_Hands).
- Me, Boris, i will be conferencing the audience about the reunion of an 
important group of patchers (i'm almost over writing this 
conference-like text extrapolating from my aether9 experience!)
we will intermingle our texts and form a strange dialogue.
our idea is that Deirdre \& I will be quite confused about what is going 
on: i will begin my conference by adressing myself to the audience in a 
similar way and setting (chair, table, lamp) than academic conferences 
are usually given, with the intention of informing he audience. Deirdre, 
dressed in a ... fancy way,  will then "interrupt me" and start telling 
her tale, with the intention of enchanting her audience. I will react by 
answering to her, in my conference mode, but with the intention of 
understanding what Deirdre is doing here. Deirdre will continue her 
tale, with the intention of unveiling the question: what am i doing in a 
digital-media performance? i thought i was in a storyteller convention!
etc. and our dialogue will go on according to what happens between us 
two, the audience and aether9 screens, but we will always stick more or 
less to the stories we wrote.


2) remote aether9 part:
in parralel to the stage action, aether9 remote action will flow on the 
screen (maybe 2 screens).
we didn't develop any script yet. but we want to.
the script will integrate formal and narrative elements related to the 
stage action.
sometimes the stage action will be on standby for the screen action to 
take all the attention. in these moments, formal elements of aether9 
will prevail.
when the stage action will be running, the narrative elements of aether9 
wil prevail, providing a visual tale.
i guess that there will be an irc channel for aether9 with somebody in 
brussels giving cues for actions.
our texts are in french and should be translated to some extent in 
english for everybody to grasp the stories...  well for Deirdre's tale, 
the basis is on the net and for mine i'll adress this later today.
we wish that all remote performers have a common "neutral" set , for 
instance something simple like : a table with a chair, a neutral wall, a 
picture on the wall, a book and paper + pen  or typemachine on the table.
this in order to avoid total randomness of the visual aspect and 
confusion for the viewers (online especially), in order to have some unity.
not all performers would be perorming at the same time, and it would be 
good to have a common neutral state in a way.

3) sound
- there will be some sound environnment in the brussel performance.
somebody will be in charge of the sound diffusion.
we are not thinking of stremaing the sound for the moment.
- we would like to invite you to upload some sound files on 
aether at artslashmedia.ftp server in the >web >"sounds\_remote\_fairytales" 
folder
- please view D. Lynch "rabbit" project for interesting sound treatment 
(in our view...)
--> here are 2 videos:
http://www.bzzp.biz/temp/
(1st episode: full first episode, 10min
fragment: 2 min extract)
- you could also record yourselves saying short sentences (cf. lynch again).


4) rehearsals
- on the 13th of june, one day before the performance, we should do a 
videocheck to see how everything is running in the gallery (at the same 
time scheduled for the next day's performance.)
- this friday (8th of june), we would like to do a jam session.
this session will adress technical problems (the aether9 system was  
working well on the "electronic café international" session of the 10th 
of may.
since manuel has been working more on the patch and we need to test the 
latest versions.
we will also write some script with formal elements to test how it goes.
for instance:
- the table setting described before
- make a cross between the 9 screens
- 2 persons making phone calls at the same time
etc.


----------------------------------

so,
who will participate in this landmark performance?
- to my knowledge, the following aethernauts said that in principle they 
are ready to take part:
Nathalie from paris, Audrey from holland, Paula from medellin, cym from 
walkendorf, Chris from yorba linda - california.
i guess that a group from geneva will do a part too.
- Dr. Gomez \& the league?
- Rhys from australia? (i heard that he would use puppets - that would 
be good - performance time in au. would be early early in the morning.)
- Johan from berlin?
- Milena from the brazil group?
- Laure form the Kingdom of belgium?

-- 
so we need to know asap:
who will be there for the jam session this friday
who wishes to take part in the "remote fairytales" perfomance on the 
14th of june (and, if possible, the rehearsal on the 13th of june).

all ideas, suggestion, disagreements etc. welcome and awaited!!!
i love this team!
looking forward,
Boris




}
}
\subject[mail:1]{1 - tommorow's jam session - details}{
\it{bk bk\high{\ref[p][mail:0].0, \ref[p][mail:2].2, \ref[p][mail:3].3, \ref[p][mail:9].9, \ref[p][mail:10].10, \ref[p][mail:29].29, \ref[p][mail:34].34, \ref[p][mail:65].65} --- 07.06.2007}

\tf{
hello!

tommorow's (8.06.2007) jam session is at
9 PM in europe
Noon in califonia
10 AM in colombia
11 PM in moscow

new working patches!!!
new 9 frame viewer html-page!!!

---

a couple exercices for us:

-let's try to upload monochromatic images only at times (whether with 
colored light if you have or filters to put in front of the camera or 
any other way we can find)
please prepare to upload images using red, blue and yellow.
+ darkness (not total)

- anything that has to do with the "Girl without hands" tale is welcome.

- if your camera films a setting with a wall (and some picture on it), a 
table, a chair and sombody (you?) sitting behind the table, perfect! 
(would be nice to try this all at the same time).

---
server list. 08.06.2007

1 \textbar{} 2 \textbar{} 3
4 \textbar{} 5 \textbar{} 6
7 \textbar{} 8 \textbar{} 9

Frame 1 - http://1904.cc/\textasciitilde{}aether/1
Frame 2 - http://1904.cc/\textasciitilde{}aether/2/
Frame 3 - http://10111.org/\textasciitilde{}aether/0/

Frame 4 - http://10111.org/\textasciitilde{}aether/1/
Frame 5 - http://artslashmedia.net/\textasciitilde{}aether/5/
Frame 6 - http://aether.front.ru/6/

Frame 7 - http://aether.front.ru/7/
Frame 8 - http://aether.smtp.ru/cym/
Frame 9 - http://aether.smtp.ru/9/

for ftp uploaders: username and password on each server are the same as 
usual.
In each case the jpegs are now named 0x.jpg to 9x.jpg
(except for cym, who uses arbitrarily frame 8. your jpg is named pw.jpg)

the html interface will be available tomorrow, linked from 1904.cc/aether

participants should announce on the list which frame they want to use.

for maxers: use the latest patch: http://1904.cc/kode/upload\_107.zip (22 k)

or, if the new javaobjects aren't working, use this simplified version 
(without visual feedback; you'll visualise through the html interface) - 
http://1904.cc/kode/upload\_109.zip (18 k)

---

boris

ps: i will gladly use frame no. 7!


}
}
\subject[mail:2]{2 - last news and performance preparation}{
\it{bk bk\high{\ref[p][mail:0].0, \ref[p][mail:1].1, \ref[p][mail:3].3, \ref[p][mail:9].9, \ref[p][mail:10].10, \ref[p][mail:29].29, \ref[p][mail:34].34, \ref[p][mail:65].65} --- 09.06.2007}

\tf{
hello aethernauts,

we jammed yesterday with
laure\_bruxell1, paula\_medellin, manu\_geneva, boris\_brussel2 and 
amirali\_teheran (chat\&images)
+
christiaan\_california, cym\_walkendorf, ideakritik\_rotterdam and the 
league\_california (chat only)
no script was established so it began slowly in a quite random jam, then 
ideakritik brilliantly gave cues according to elements of the "girl 
without hands" tale. it lasted 3 hours or so.

--

technical conclusions:
- the viewing 9frame html page works great (i attached some screenshots, 
some slots are black because we weren't 9)
- for maxers: the patch upload\_109 works perfectly
- for ftp uploaders: it works well also
this means that the transmission system for next thursday's performance 
is fully fonctional!

--

practical conclusions:

- we all agreed that it is extremely helpfull to be 2 persons for the 
performance, for example:
1 person in charge of following the irc chat for the cues, for the 
computer manipulations and the camera manipulations
1 person in charge of the image (as actor, puppetist, drawer etc)

- we need a fine tuned script - this script should be finished on 
wednesday at the latest. (see my propositions below)
- each performer needs a clear method. everybody develops one naturally, 
but it's better to find it before the performance than during the 
performance... will you use a fix or moving camera?  etc etc.
we need law\&order: more the formal aspects are defined, the better it 
"feels" (knowing that anyway it will be quite hazardeous - i like this 
struggle).
it's possible at any time now to upload images and visualise it on 
 >http://1904.cc/aether/ >new test interface!!!

- perform simple \& small things
we may think that what we are uploading is not "rich" or interesting 
enough, but then we forget that there are 8 other images!

- react quicky \& act slowly
it's difficult to be all alert \& super fast to react to cues and in the 
same way remember not to rush the action

- obviously prepare a maximum of stuff before the beggining, in this way 
you can concentrate on less things during the performance.

maxers:
- use the 109 for upload and the htlm 9frame page to view (not the more 
complex upload\_107 patch, wich may crash)
- use the start/stop button often - do not constantly upload images. 
press the stop button to rearrange the setting and prepare for next actions.
pressing once "start" then immediatly press "stop" will upload just one 
image. this is great to create controlled animations (although it's hard 
to control the order of the images, but it's verry fun nevertheless).

--

proposals for the basis of the script:
- the script form in grid that we used for the first aether9 performance 
was great i think.

- it is now certain that we will use the "Girl Without Hands" tale. 
Deirdre will tell it during the performance in brussels (we don't know 
exaclty "how")

- think it would be efficient and elegant to make a clear distinction 
between when we upload "formal" elements (like "film an empty desk 
before an empy chair") and "narrative" elements like
("The devil sends a messenger to kill the queen and daugther!")

i guess some of us prefer to do a performance with emphasis on formal 
visuals, while others are more interested in emphasizing the 
video-storytelling - it's different types of freedom and constraint.  
both aspects are very important but cannot realistically be reunited by 
every single performer yet (because it would need so much preparation).
one way to do this would be to have a constant distinction on the screen 
between who is working "fomal" and who is working "narrative".
yesterday when everybody displayed an empty chair - the impression was 
striking! common action works, and they never fail to create an 
extremely deep impression.
but the formal elements should always be thought of as going with the 
tale in a way but not interpretating it litterally.
all this shall be scripted of course.

- the n3krozoft team can edit the script according to all the input on 
the mailing list and related discussions, but if somebody or a group 
wants to take the responsability to put this sript together, it would be 
a great relief for us (because we are also in the rush of preparing the 
exhibit: building fine zoetropes, editing dvd's, printing very large 
video stills and more...). Please anonce it asap if you want to do this.

--

lineup for thursday the 14th:

sure: cym, laure, paula, nathalie, christiaan
  maybe: geneva, amirali:teheran, ideakritik
no news: johan (berlin), alejo
--> so we aren't ready yet participation-wise (there won't be any live 
performance in the brussel gallery space itself.)

--

lookking forward,
boris


}
}
\subject[mail:3]{3 - brussels performance report}{
\it{bk bk\high{\ref[p][mail:0].0, \ref[p][mail:1].1, \ref[p][mail:2].2, \ref[p][mail:9].9, \ref[p][mail:10].10, \ref[p][mail:29].29, \ref[p][mail:34].34, \ref[p][mail:65].65} --- 16.06.2007}

\tf{
hello all,

we just opened the gallery this morning in brussels for the last day of 
the exhibit here.

i write now a little performance report

/////
1. Mother Nature
we where about ready (one hour before the performance)
 when mother nature invited herself
it started to rain and rain and rain
and all at once, water filtered through the roof
of the large performance hall
and poured in buckets on our command table
we ran to unplug the curent and move tables in the dark
to protect the computers and cameras and stuff
the sound was ear-crushing: torrential rain
in 5 minutes, there was 5 cm of water everywhere
and we where soaked... and sort of annoyed...
so we rushed to transfer everything from the big hall to the
smaller exhibit hall
while the lady-gallerist was calling the fire department
/////
2. Action
quite stessfull we managed to be not too late for the beginning
and the performance started in front of about 30 people (that where 
brave enough to come in spite of the exceptional storm)
some more walked in during the show
they where welcomed with champagne and other drinks
Deirdre told the tale in a baroque white dress colored by the projection 
rays
Chloe mixed discreet sounds \& music
Boris told his story sitting in front of the audience on a desk while 
trying to film his part of the tale.
Manuel managed the projected images and chated on the skype conference
/////
3. Audience reaction
some people where quite confused, but most of them liked it to put it 
simply and shortly.
///////
4. Debriefing
it was hard to be fully concentrate under these climatic conditions - 
but it was exciting...
- as there was no actual stage anymore, Deirdre found it extremely 
difficult to properly perform her craft, storytelling. she also feels 
frustration that we didn't have enough time to prepare a more refined 
"mise en scene" between her and Boris.
- Manuel was trying to give cues according to Deirdre's words, and 
obviously it was going much too quick for the remote performers (\& for 
manuel also who had to solve some techicalities...) - it was a BIG 
mistake to do so. we apologize for that. a better thing would be for the 
performers to switch between acts every 3 minutes for instance - like 
that everything would be co-ordinated whitout questions...
it is a shame really that the performers weren't able to display their 
work in a calm manner with sufficient time to do so. this will not 
happen again.
- Boris is basically satisfied of the opportunity to read loud his 
speech (adapted for the circumstance...).

--

pictures \& movies will come soon on the aether website.

and check cristiaan site:  http://zurcnaaitsirhc.blogspot.com/

--

à bientôt,
Boris




}
}
\subject[mail:4]{4 - Rotterdam script}{
\it{1.1 [*] 1904.cc 1.1 [*] 1904.cc \high{\ref[p][mail:11].11} --- 02.07.2007}

\tf{
to the rotterdam performers,
here the basic version of the script.

9h00 opening (empty desks, chairs, etc)
9h01 character 1 appears
9h02 character 2 appears
9h03 character 3 appears
9h04 character 1: action (fills a glass of water)

9h05 character 2: action (...)
9h06 character 3: action (...)
9h07 character 2: talks (in direction to 1)
      character 1: answers (in direction to 2)
9h08 character 3: leaves the space
9h09 character 2+1 seem nervous
9h10 character 3: comes back

9h11 character 3: talks
      character 2: looks to character 3
      character 1: close up action
9h12 character 3+2: close up action
      character 1: looks at the others
9h13 character 3: close up action
9h14 character 1,2,3: stand up, saluts au public.
9h15 empty desk


for additions or changes, use the wiki:
http://1904.cc/timeline/tiki-index.php?page=020707+rotterdam+script

}
}
\subject[mail:5]{5 - Re : today -- hotel new york}{
\it{::audrey:: ideacritik\high{\ref[p][mail:17].17, \ref[p][mail:18].18, \ref[p][mail:22].22, \ref[p][mail:27].27} --- 02.07.2007}

\tf{
i apologise - i will be late for what i thought was a noon online-ness from the space.
i am waiting for the dvcam which was supposed to be handed to me at 11.
audrey


}
}
\subject[mail:6]{6 - Rotterdam performance}{
\it{Boba project.location\high{} --- 03.07.2007}

\tf{
Dear all,

It was a true vibe to watch you all perform!!! I'm sending you million
virtual kisses from somewhere on the way to Belgrade, and I'll be watching
you on the 7th :)*

I hope to keep in touch,
Boba



}
}
\subject[mail:7]{7 - final TWO for 7/7/7}{
\it{cym net cymnet\high{\ref[p][mail:8].8, \ref[p][mail:14].14, \ref[p][mail:23].23} --- 06.07.2007}

\tf{
i will be making tree-color pasta (fusilli tricolori) with tomato
sauce, with some vegetables added to the tomato sauce

in fact i am eating the pasta right now already and didn't start
making the sauce yet, but i will try to do that still later tonight to
have all the material ready for tomorrow

i need to first download the movies of the first part and charge the
batteries before i can continue the second part of cooking. it takes
quite some planning to cook a meal online

are there any other recipes? does anyone else already know what s/he will cook?

by the way, the three colored pasta i used is green, orange and white.
i might stick to those colors also with the vegetables and add the red
tomatoes as a contrast...

any suggestions how to add numbers and formulas to the recipe?

happy cooking,

cym




}
}
\subject[mail:8]{8 - question - logistics}{
\it{cym net cymnet\high{\ref[p][mail:7].7, \ref[p][mail:14].14, \ref[p][mail:23].23} --- 06.07.2007}

\tf{
it seems we are 8 now? is that okay? can we make it a 7+1 performance?
7+1 squares? for me it would be okay





}
}
\subject[mail:9]{9 - 070707 news}{
\it{bk bk\high{\ref[p][mail:0].0, \ref[p][mail:1].1, \ref[p][mail:2].2, \ref[p][mail:3].3, \ref[p][mail:10].10, \ref[p][mail:29].29, \ref[p][mail:34].34, \ref[p][mail:65].65} --- 07.07.2007}

\tf{
http://1904.cc/aether/live/index.html\#

Hello,
so we will be 7!

following the set-up shaped as number 7,

XXX
-XX
XX-

i propose:

NORWAY      LA1              LA2
--------              BRUSSEL    GENEVA
SLOVENIA    AUSTRIA    --------

or in other words:

league front
n3krozoft front
slovenia front

in this disposition for instance:

Dr. Gomez   Dr. Shleidan   Dr. Madmann

-------------     BXL n3kro     GVA n3kro

Luka             Cym                ------------

///////////////

if the 3 league members can coordinate something, great!

here in brussels we are of course goig to respect the first and last 7 
minutes segments (presentation \& eating) but haven't decided yet if we 
are going to explicitly respect the 5 middle 7 minute segments. surprise 
it will be.

i will carefully record the performance with a camera filming a computer 
screen.

see you all later,
Boris
}
}
\subject[mail:10]{10 - next performance SETUP}{
\it{bk bk\high{\ref[p][mail:0].0, \ref[p][mail:1].1, \ref[p][mail:2].2, \ref[p][mail:3].3, \ref[p][mail:9].9, \ref[p][mail:29].29, \ref[p][mail:34].34, \ref[p][mail:65].65} --- 16.07.2007}

\tf{
hello

here is the setup situation for the central balkan mountains performance 
in Bulgaria on the 19th of this month
confirmed:
 > Cym + Luka (wd8\_austria)
 > Lucy/The League (norway)
 > Boris (belgium)
 > Manu + Alejo (Dortmund Germany)
 > Nathalie - did you solve your camera problem? (Paris France)

unconfirmed:
 >>> Chris?
 >>> Laure?
 >>> Judy?
 >>> Amirali?
 >>> ???

--

have a look at the pic gallery of the place, it's quite something: 
http://netuser.bg/2007/en/?gal=1\&page\_id=500

-- 

i had a look to one of the chats alejo pointed to: http://home.gabbly.com/
super easy to embed... for bulgaria it would be no problem. (the 
simplest way is to type gabbly.com/ in front of the url of the page you 
visit, and all others visitors doing the same will be reunited in a 
chatroom embedded on the visited url...)
issues are:
it eats quite some place on a web page.
the design is... what it is - but more horrifying is the ad embed in the 
top corner of the chat (i did not understand if we can replace it by 
what we wish).
you can see this chat at  http://gabbly.com/colboard.com/index.php - 
when i tried it, 8 people where chatting at supersonic speed and it 
worked perfectly.

--

Coco Islands Time Zone: UTC+6½

--

Boris



}
}
\subject[mail:11]{11 - bulgaria - interface}{
\it{1.1 [*] 1904.cc 1.1 [*] 1904.cc \high{\ref[p][mail:4].4} --- 19.07.2007}

\tf{
hi all,
here is a possible setup for the Hotel Pleven performance (changes 
are possible of course):

laure(gva).JPG \textbar{} cym+luka.PNG \textbar{} audrey(rdam).JPG

chris.JPG      \textbar{} boris.JPG    \textbar{} league.JPG

objects        \textbar{} objects      \textbar{} objects

the audio player is now a bit more hidden, see the new interface 
from: http://1904.cc/aether/live/

since we had some trouble (uncontrollable delays) with the .ru 
servers, i propose to drop them for now.

also, after some longer testing, it seems that png has a tendency to 
be quite heavier than jpg, specially with very colourful/contrasted 
material. so, since the max patch has now a contrast function, i 
suggest we switch back to jpg (not for the pd users though).

this are now the corresponding FTP directories:

frame 1 - ftp://1904.cc/1/  format: jpg
frame 2 - ftp://1904.cc/2/  format: png
frame 3 - ftp://imaginaryscience.org/3/  format: jpg

frame 4 - ftp://imaginaryscience.org/4/  format: jpg
frame 5 - ftp://10111.org/web/5/  format: jpg
frame 6 - ftp://10111.org/web/6/  format: jpg

frame 7 - ftp://artslashmedia.net/web/7/ - OBJECTS
frame 8 - ftp://artslashmedia.net/web/8/ - OBJECTS
frame 9 - ftp://artslashmedia.net/web/9/ - OBJECTS

note: user is "aether" for all servers, except imaginaryscience.org 
where the user is aether

note2: for frame 6 (league), there is only 1 self-refreshing png 
file, called 1.png

note3: i didn't include myself among the performers - will have a 
busy day tomorrow and preparing the technical/html stuff + preparing 
the performance/finding the objects will be hardly possible.

if any of you needs special settings (filenames etc), tell me as soon 
as you can.

best,
manuel

}
}
\subject[mail:12]{12 - (no subject)}{
\it{laure deselys lauredinateur\high{} --- 19.07.2007}

\tf{
hello,


leaving brussels pc-naked made it quite hard for me to
organise tomorrows performance and was also very busy.
so couldnt take part so much to the wiki and script
elaboration.
 i am and will be for tomorrow  located in france
(high-savoy), i still have a  few technical problems
but it should be solved by tomorrow midday. cross
finger.

the place where i here grab the net, is in a kind of
basement
and for me it actualy seems easier to use a window
that is ground high, then a  door. i dont know, if
considering this i should be placed somewhere else in
the nine screen strucure.


maybe i can find a solution with a door but cant
confirm it right now

for the four objects i was thinking of:

1. red kimono
2. hunting gun
3. hat(s)
4. mountain boots

boris, i dont expect you to have these important
things at home, but you can also use the color or hand
gesture to make contact with them

or if its not considered usefulll enough for bulgaria
( desk type)

1. pencil sharpner
2. hair brush
3. embrella
4. goggles

and so on . theres such a mess of objects in this
place 
i must make a tactico-narrativ choice

i will send the pictures for the slide show tomorrow
for midday

more in a few hours
gotta catch up some sleep



laure.



Luka Princic / Nova Viator nova at viator.si 
Thu Jul 19 17:44:05 CEST 2007 
Previous message: [aether] 115 or 114 
Next message: [aether] wd8 -> live 
Messages sorted by: [ date ] [ thread ] [ subject ] [ author ] 

On Thu, 19 Jul 2007 17:03:28 +0200
1.1 [*] 1904.cc wrote:
> i set everything back to JPG except the frame used by cym/viator 
> (which is using the 0x.png - 9x.png naming scheme, hopefully it will 
> work with Pd)
> 
> cym, i remind your FTP access:
> frame 2 - ftp://1904.cc/2/

yes.

i just drove to wd8 at cym's, managed to download the linux driver for this nifty cym's hercules webcam, and it works!

we're uploading.

what's the chat channel this time then?

ll.

attaching the in-process screenshot *hope this list doesn't eat attaches for dinner*, wink wink...


-- 
<aav> coffee on an empty stomach is pretty nasy
<knghtbrd> aav: time to run to the vending machine for cheetos
<aav> cheetos? :)
}
}
\subject[mail:13]{13 - Thoughts, and a riddle}{
\it{Lucy H G art\high{} --- 19.07.2007}

\tf{
Quick thoughts!

I think this project in Bulgaria is a bit of a riddle - the script and
performance - figure out as you go - search/consume/share.  The first aether
project with its highly developed storyline, pseudo-science and mythological
themes still holds the most for me...  It was quite fun to do, and I think
it is because of the intensive scripting - even though this made it more
challenging.

I like the subject/object- consumerism theme for this unnamed last piece,
but I think timing is essential.  Coordination of objects moved from frame
to frame, etc.  Perhaps we are all in our own space time bubbles - we are
still quite disconnected during these performances, and I think a live audio
conference between us would help us to synchronize, as well as get to know
one another.

Instead of a new theme for each new project - perhaps we could return to
this study of an object, of coordination, of passing items through the frame
and communicating more "physically" with one other?  Boris had called for us
to simplify, I think...  With very little setup or concept - just play
catch, as it were.

Lucy/League


}
}
\subject[mail:14]{14 - Re : Thoughts, and a riddle}{
\it{cym net cymnet\high{\ref[p][mail:7].7, \ref[p][mail:8].8, \ref[p][mail:23].23} --- 20.07.2007}

\tf{


personally, i like to see aether 9 as a band that is playing live
concerts. (or a theatre group that is doing theatre shows, but for me
it is easier to compare it to a band)
a band has a set of songs that they perform live. they practise the
songs until they know them and can play them on the stage. they have a
set of material and can change the set/order a little for each
performance, and also create moments in the performance that leave
space for improvisation. but still the basic structure is very clear
for each band member. and at the same time it doesn't get boring to
perform the same songs/compositions (or same theatre play) again and
again.


}
}
\subject[mail:15]{15 - Re : Thoughts, and a riddle}{
\it{Judy Nylon knickerbockerjuju\high{\ref[p][mail:28].28, \ref[p][mail:30].30, \ref[p][mail:31].31, \ref[p][mail:32].32, \ref[p][mail:33].33, \ref[p][mail:35].35} --- 20.07.2007}

\tf{}
}
\subject[mail:16]{16 - Re : Thoughts, and a riddle}{
\it{Luka Princic / Nova Viator nova\high{} --- 21.07.2007}

\tf{understanding the feeling of people involved in 'meat space'...
anyway, i feel there is certain beauty (if i may use such a term here)
in the efforts and connections that are happening here, and during the
performances. 

i'm also learning a lot about group performance and how things are
bringing people closer together. the people who are performing, and the
notions of distance and global and speed and time and place and space.
i wish i had more theoretical tools to explain more precisely what i
mean. but poetics can be also quite telling.

i will/can work on a pd patch and linux-enabled approaches to aether9.
}
}
\subject[mail:17]{17 - Re : Re : Thoughts, and a riddle}{
\it{::audrey:: ideacritik\high{\ref[p][mail:5].5, \ref[p][mail:18].18, \ref[p][mail:22].22, \ref[p][mail:27].27} --- 24.07.2007}

\tf{
in response to 'getting to know each other' - as i also see this as a crucial element in a collaborative performance. especially since we cannot for the most part meet each other physically, small tidbits of knowledge become the extent of what we know about each other. these tiny tidbits are therefore also the only thing we have to go on in so far as expectation or imagination of how the other will act during a performance. in our excitement over 'the patches', have we perhaps overlooked that we must not only play, but play together? the person who 'assisted' me last performance was amazed at how little attention we payed to each others performances, 6 little autisms neatly assembled on a page. perhaps getting to know each other, even if only through these unsatisfactory mediums, is a start towards an aether9 embodied performance as opposed to several individuals operating in the same screen though otherwise seemingly detached.

i was lucky enough to meet cym last week in austria as well as to attend the workshop in geneva where i of course met manu/aether9, alejo and others who are more or less inactive for the moment. i am based in rotterdam. i've just finished my MA in media design. my thesis is about the interaction/affect between media and our mourning process, and how we perceive death. as in many of my endeavors, i interviewed many people about this very idea and incorporated their accounts into my final project/thesis. i am explaining this only because my somewhat anthropological methodology outlines one of my main interests in this group;  how do remote performers work together, communicate, make decisions, disagree, in short, form a group. the limitations of our communication media form a particular set of 'constraints' which inherently shape our performance together. perhaps my use of the word constraints is not just because it is usually derogatory. i see these
 constraints as borders from which to work with, in the same light as our 'low-tech one frame per second' refresh speed. i do not wish the frames to load quicker or to attain the perfect resolution. i prefer to take on this dimension and exploit it, in all its 'dramaturgical' glory. in the same way, i think we should exploit our 'thwarted' methods of communication. we often throw together a script at the last moment and more or less blindly perform all together without knowing what each other is doing. could we take this on as some sort of methodology? could this be made explicit as a performance tool? in this vein, i can imagine that we use simple 'unification' props such as text (which we have yet to integrate successfully but in which i have confidence). these props could be constant during each performance, and this aspect would be practiced, but the rest could be 'off the cuff' or improvised as usual. as cym noted, using text, written on a piece of
 paper as we did in the first performance, was really nice. this is an idea which i feel is a beautiful metaphor for our communication together if it is used to communicate between frames during a performance. i imagine inmates passing each other notes between jailcells, all being watched on simultaneously by the jailguard on nine tiny screens hooked up to those surveillance cameras. we use the technology but we still attempt to communicate with paper and pen... this could be a constant which we can rehearse... for example.

i hope this hasn't been too much of a digression to describe my feeling/desire to incorporate our communication process implicitly in our performances as well as to keep some things constant in order to refine our practice, focus our experiments.

audrey/ideacritik






}
}
\subject[mail:18]{18 - Re : more archives online}{
\it{::audrey:: ideacritik\high{\ref[p][mail:5].5, \ref[p][mail:17].17, \ref[p][mail:22].22, \ref[p][mail:27].27} --- 06.10.2007}

\tf{
we are trying to find someone to do the music-audio ... the idea i mentionned about streaming audio from there, to participants, and back into the room... no one confirmed thus far. do we have a backup audio streamer available - i know boris and laure are in transit so not available...

i`m off to a chalet for a couple of days, more upon return.

so who?s in???

audrey

}
}
\subject[mail:19]{19 - 26 october reconfirmation}{
\it{Paula Vélez Bravo ciruela\high{\ref[p][mail:24].24, \ref[p][mail:25].25, \ref[p][mail:26].26, \ref[p][mail:40].40, \ref[p][mail:46].46, \ref[p][mail:52].52, \ref[p][mail:59].59} --- 11.10.2007}

\tf{
so, the script and everything will be at the same link i supposed.

will connect today to skype to see if we can try things up.

other thing, the sound streeming... i would like to try. it will be  
using giss tv?

ok.
see you soon on the net.

paula

}
}
\subject[mail:20]{20 - rehearsal times}{
\it{nicola unger nicola_unger\high{} --- 19.10.2007}

\tf{wednesday 24 morning 11.00-14.00

thursday to be confirmed and friday gigtime as well
audrey will figure out techsetup today, not sure how
far we get.

cu soon!
nicola

}
}
\subject[mail:21]{21 - RE : last thougths this evening}{
\it{fougeras nathalie fougeras_nathalie\high{} --- 22.10.2007}

\tf{sorry for today.... ok i bought my webcam so tomorrow morning it's possible for me to participate to the rehearsal
see you soon
cheers
Natali

}
}
\subject[mail:22]{22 - Re : Re : RE : last thougths this evening}{
\it{::audrey:: ideacritik\high{\ref[p][mail:5].5, \ref[p][mail:17].17, \ref[p][mail:18].18, \ref[p][mail:27].27} --- 23.10.2007}

\tf{
26.10.2007
14:30: general rehearsal (all *must* be present)
15:15: second/last general rehearsal (all *must* be present)

*things to schedule:
sound check : paula tu es disponible quand? et le vendredi comment tot p-e etre online avec le streaming du son.
je vais uploader de l'audio de trains holandais que j'aimerais incorporer au soundscape, quand sera tu online pour qu'on en discute?

a\textbar{}idea

}
}
\subject[mail:23]{23 - aether9 level2}{
\it{cym net cymnet\high{\ref[p][mail:7].7, \ref[p][mail:8].8, \ref[p][mail:14].14} --- 26.10.2007}

\tf{i just spent the whole week in ljubljana teaching 18 students
HTML/webdesign (from 10am till 21pm every day..), so maybe i am
focused a bit too much on such details, but anyway.. the interface
looks really nice in firefox and would be good if it looks the same in
explorer...

anyway, i am right now watching, waiting for the performance to start.
some of my students also said they will watch tonight, so it would be
good to put some message or something online, that performance is
starting 30 mins later. I am not sure if it is enough just that the
number of seconds how long to wait keeps on changing. would be better
to put a line on the bottom that says
* performance will start 30 mins later - 22:30 CET *

okay, i will keep on watching and waiting and take some photos

cym}
}
\subject[mail:24]{24 - tremor participation confirmed}{
\it{Paula Vélez Bravo ciruela\high{\ref[p][mail:19].19, \ref[p][mail:25].25, \ref[p][mail:26].26, \ref[p][mail:40].40, \ref[p][mail:46].46, \ref[p][mail:52].52, \ref[p][mail:59].59} --- 01.11.2007}

\tf{street,
sur un mur de la rue. outdoors at night, like christian said last  
night : an aether midnight drive-in.
if it is complicated i could do it maybe, i'm just asking ... in a  
cool place call: "pato feo films"

a friend of mine is working in this space... he is the one who  
invites us to participate in this festival

christiaan said yesterday:
its a fun space, looks like some cool sound and video stuff
i like how everything is on the floor. so you have to sit japanese style

there are some chairs and a sofa, but mostly you have to sit on the  
floor. so i'm trying to do it there... i will see. i need people to  
help me. maybe this friend of mine that manage that place.


PAula


}
}
\subject[mail:25]{25 - bogotá tempete}{
\it{Paula Vélez Bravo ciruela\high{\ref[p][mail:19].19, \ref[p][mail:24].24, \ref[p][mail:26].26, \ref[p][mail:40].40, \ref[p][mail:46].46, \ref[p][mail:52].52, \ref[p][mail:59].59} --- 04.11.2007}

\tf{
AETHER9:

à bogotá..... puis, pas facil faire un reperage, a part de prendre  
quelques images... mais il s'est passé un truc incroyable.
Une tempete de verglas, neige, pluie...

je suis venu pour le festival rock al parque. ajourd'hui je regraette  
de ne pas avoir allée, car les photos et films que j'aurai pu prendre  
j'en reve...

voici c'est qui c'est passe aujourd'hui, regardez quelques photos du  
journal.

encore vous dirai que ce n'est pas qu'une tempete de neige. mais en  
colombie ne neige pas!

seulement à 5.000 metres d'altitude. et voici, on est à bogotá à  
2600mts... rarement il y a des phienomenes comme ça.

http://www.eltiempo.com/multimedia/galerias/granizada/GALERIAFOTOS- 
WEB-PLANTILLA\_GALERIAFOTOS-3801694.html

http://www.eltiempo.com/multimedia/galerias/granizadalectores/ 
GALERIAFOTOS-WEB-PLANTILLA\_GALERIAFOTOS-3801410.html


paula



}
}
\subject[mail:26]{26 - urgent tremor RDV2}{
\it{Paula Vélez Bravo ciruela\high{\ref[p][mail:19].19, \ref[p][mail:24].24, \ref[p][mail:25].25, \ref[p][mail:40].40, \ref[p][mail:46].46, \ref[p][mail:52].52, \ref[p][mail:59].59} --- 08.11.2007}

\tf{
nicola and audrey:
sorry about your internet connection.
i don't know how are we going to di now.
i have been waiting aethernautes since more than an hour... to fixe  
things, do a rehersal... mmm i don't know, i start to be worried  
about it.
don't know exacly how are we going to do.
so... i have to go to get some videos fprm the street to use in the  
performance.

i need participants ready to do some test tomorrow night (night in  
european time) it could be 18h eurpoean time.

we have to think about a plan B. maybe a quick new script and see if  
other AETHERnautes want to participate.
I know the hour it is not easy for you all. 1am in saturday morning  
10 november... for me it will be 9 november 19h.

so, tomorrow i wish all participants will be connected the sooner you  
can...
i will try to be connected since 10am  so 16h european time.

ok.
paula


El 8/11/2007, a las 14:31, nicola unger escribió:

> hi all, audrey and me are still in a situation without
> internet at home...
> so better not count on us, it is really stressfull to
> get online and timing is shit...!
> sorry...
>

}
}
\subject[mail:27]{27 - Re : videocapture did works}{
\it{::audrey:: ideacritik\high{\ref[p][mail:5].5, \ref[p][mail:17].17, \ref[p][mail:18].18, \ref[p][mail:22].22} --- 11.11.2007}

\tf{
----- Message initial ----
De : Paula Vélez Bravo ciruela at [nospam] une.net.co

>did somebody put the video already in the net? i would like to see it with sound incorporated.

hi all,
i am just now going through all the aether mails. as you know i am offline. in addition to that, the dv tape of the WORM performance has rather embarassingly ended up in someone else's camera which is now in turkey and etc so i won't even get my paws on the thing for another whole week. for this disorganisation i apologise as i find the immediacy of watching the performance from the 'outside' very important for critique. once i get the tape i'll be on the case asap.

a\textbar{}idea 

ps some feedback i received :
-it would have been interesting if the actions performed/mimiked by the actors have a connection to the medium, its limitations, politics, etc. rather than just stretching for example... 
-the storyline was not very clear
-it was not clear whether the brussels/geneva performances were live or recorded (which made the tension very interesting - or the intrigue).
-the cues which i gave to nicola (by whispering into a microphone) were for some very nice, bringing my role in the performance more clear and also adding to the intrigue of not knowing if the cues go to nicola or also to the other actors (who is guiding who?). for others much for the same reason a bit confusing, why those cues? (i.e. front/back/left/enter/leave/apple/...) they seemed too random.

in general the feedback was positive and the critical points were brought up because they saw potential in the project.

that's all for now. still reading the last 20 aether mails... :S




}
}
\subject[mail:28]{28 - sunday ghost [TITLE suggestion]}{
\it{Judy Nylon knickerbockerjuju\high{\ref[p][mail:15].15, \ref[p][mail:30].30, \ref[p][mail:31].31, \ref[p][mail:32].32, \ref[p][mail:33].33, \ref[p][mail:35].35} --- 09.12.2007}

\tf{Please consider for the Aether adaptation of "Ghost Trio" could be called "Ghost Given". It sounds like 'forgiven' and suggests the intangibility of what is received and the uncertainly of who is actually doing the giving. 

Judy
}
}
\subject[mail:29]{29 - a marvelous fantasio rescripture von the SPOOKY TRIO de beket}{
\it{bk bk\high{\ref[p][mail:0].0, \ref[p][mail:1].1, \ref[p][mail:2].2, \ref[p][mail:3].3, \ref[p][mail:9].9, \ref[p][mail:10].10, \ref[p][mail:34].34, \ref[p][mail:65].65} --- 11.12.2007}

\tf{
 > music:
- i didn't produced anything, contradictory to what i announced 
yesterday on skype. questions:
- do we want to use Beethoven’s Piano Trio?
If yes, do we want to respect more or less the indications from Beckett 
OR do we want to freely play with this trio, with mixed loops or samples 
for example? (i understand that the uploaded music is only a fraction of 
the second movement of the trio. all the music indicated by Beckett 
commes from this second movement. i do not have the full piece - i 
uploaded, for the sake of the example, the part that goes with "Act1, 
l.31" - i also uploaded a couple nice loops).
- There's very little (faint) music in the original indications 
(repertoried in the wikipedia page).
- If we decide to use musical elements, is there sound appart from this 
music, for example do we want to stream some sounds of aint winds? Or 
"pastel noise"?
- Judy, how do you generally feel this music part? What are your ideas?

}
}
\subject[mail:30]{30 - RIGHTS}{
\it{Judy Nylon knickerbockerjuju\high{\ref[p][mail:15].15, \ref[p][mail:28].28, \ref[p][mail:31].31, \ref[p][mail:32].32, \ref[p][mail:33].33, \ref[p][mail:35].35} --- 12.12.2007}

\tf{I called the agency here in NYC about the rights. They have never done anything for live and on-line for anything they represent and have not had any request for this TV play while she has worked there. The agent I spoke with said she would like a write up of what we are doing emailed to her     kate \{a+\}  gbagency.com      so that she might pass it along to Edward Beckett (owns the estate) who lives in England....to request permission and establish a fee for performance rights.

I too suggest we wait. There is not time before Saturday. JUDY
}
}
\subject[mail:31]{31 - Beckett estate & performance rights alarm}{
\it{Judy Nylon knickerbockerjuju\high{\ref[p][mail:15].15, \ref[p][mail:28].28, \ref[p][mail:30].30, \ref[p][mail:32].32, \ref[p][mail:33].33, \ref[p][mail:35].35} --- 13.12.2007}

\tf{
At some point we must consider the ways around this because we will not receive this man's blessing at any price. However, if everything remains web-based in its production and presentation, he will have nowhere to drop his lawsuits \& 'cease and desist' orders. We will have to change the name of the play/never call it an adaptation and perhaps morph it into something else......in my opinion......possibly retaining Beckett himself as the character (f)

Judy--
}
}
\subject[mail:32]{32 - today's reheasal & re-write of announcement}{
\it{Judy Nylon knickerbockerjuju\high{\ref[p][mail:15].15, \ref[p][mail:28].28, \ref[p][mail:30].30, \ref[p][mail:31].31, \ref[p][mail:33].33, \ref[p][mail:35].35} --- 14.12.2007}

\tf{We’re presenting a live multi-based web link up on Saturday December 15th. Loosely adapted from a theater piece written in 1975, taped in ‘76 and televised on BBC2 in 1977. We are mixing it with a bit of dub on classic piano and samples.  It is a parallel universe played out in the same slice of time as the first wave of British punk. The language will be English. The development of live storytelling without a single fixed location is on going. 

My fellow aethernauts and I would be delighted if you would join us on-line at our first chance to bring this piece from TV to a wider audience. It doesn’t matter what you’re wearing and you won’t have to worry about how you’re getting home.

Members of Aether9 will be performing from different locations:  Medellin, Yorba Linda, NYC , Brussels, Geneva, and Paris. The audience, those in the same room, will be joined at ‘Videomedja 2007’ at the Museum of Vojvodina, Novi Sad, Serbia, by two Aether9 artists from Austria and Slovinia who will host the Q \& A which follows the performance.

There are no complicated programs or hardware involved. It is easier than getting out of the house. I hope you will look through our site at  www.1904.cc where the collaborative process involved is quite transparent, our project laid out, and participants named. Here we will make the links “to set up your computer” so easy that you will practically fall through them to the new venue.

You will need to go through one link to find out what the time will be where you are, when it is Saturday at 11:15 PM in Serbia. The second link opens the audience viewing window and a third link leads to instructions to stream sound.

}
}
\subject[mail:33]{33 - new script for ghost trio.....looks more clear. JN}{
\it{Judy Nylon knickerbockerjuju\high{\ref[p][mail:15].15, \ref[p][mail:28].28, \ref[p][mail:30].30, \ref[p][mail:31].31, \ref[p][mail:32].32, \ref[p][mail:35].35} --- 17.12.2007}

\tf{
Re: feedback: Beyond the compliments and stirring of the imagination the useful thoughts from friends were as follows: 1) an actor said that it still lacks emotive room in the small screens and we might insert extreme close-up shots to allow us to receive the interior part of the acting. OR...soon maybe we will be able to feature that one square larger when something like that is going on in the script. 2) a filmmaker in Berlin loved the piece and was familiar with  Beckett and seemed to know that it was impossible to get the rights to do any adaptations of Beckett's plays.  Her only critical comment (among lots of praise) was that the credits were  impossible to follow and she wasn't wild about the handwritten  "live from  wherever" cards.  I thought seeing everyone on camera for a second was reminiscent  of the informal credits on a Doris Dorrie film. I liked it. 3) I had one person who is close to 70 years say that she couldn't get to it and that the interface still
 wasn't easy enough. I mention her because she is someone who has a foundation that gives grants and we do want to get everybody to be able to see what we do. 4) tomorrow at their Xmas party, I will hear if anyone at HWKs (www.harvestworks.org) saw this and will start them thinking about sponsoring us to have a gig at the New Museum which just opened a state of the art building on the Bowery a couple of weeks ago. I have been to Location1 (the art space I mentioned before) and feel that they have very little to offer us.

Onward....Thanks one and all...good show, it was a pleasure.  Judy}
}
\subject[mail:34]{34 - new script proposal}{
\it{bk bk\high{\ref[p][mail:0].0, \ref[p][mail:1].1, \ref[p][mail:2].2, \ref[p][mail:3].3, \ref[p][mail:9].9, \ref[p][mail:10].10, \ref[p][mail:29].29, \ref[p][mail:65].65} --- 15.03.2008}

\tf{
Hi all,
I throw a script idea here. Still needs lots of input fortunately. I 
believe now that it would be better to do twice the same script for the 
2 april performances.

---

It is a first draft sort of inspired by the LRRH tale, an essay from Lev 
Manovich, sites like livejasmin.com and elements of our previous 
discussions. I also had a meeting with Chloé in a brussels bar yesterday 
night where we discussed this.
This requires 5 live performers. I guess that it would last about 40 
minutes.


LAYERS
 > remember the first script, i think we all liked the 3 layers: Gods / 
Stratosphere / underworld. I think we mostly liked it and in this 
proposal, there are also layers. But not so rigid: at times the fixed 
clear structure is broken when the performers decide to upload their 
images to another frame. There would be set moments for this principle, 
it would be chaotic (like an unpredictable frame battle - easy to do for 
the max-patch users, i don't know for performers who use another upload 
system)
(check Lev Manovich essay "Interaction as an Aesthetic Event" (2007), 
found the other day on manovich.net, a short must read that i attach it 
to this mail. Key quote: "Command-line interfaces "deliver the goods", 
that is, they focus on pure functionality and utility. GUI (Graphic User 
Interface) adds "service" to interfaces (like mac OS9). And at next 
stage, interfaces become "experiences" (like mac OSX) " (it helped me to 
imagine the 3 layers as: "goods / services / experiences" but i don't 
think we should remember this for a performance script really).
 

TOP LAYER (frames 1,2,3) - experiences
the Little Red Riding Hood
 > must attract all 5 senses.  
 > Shows 1 girl doing the webcam thing (check www.livejasmin.com to get 
the idea...) - her body is spread out on 3 frames, in the illogical 
horizontally way (assuming the girl is sitting and not laying on a 
matress for example, but she could switch positions of course):
frame 1: head and shoulders
frame 2: chest to tights
frame 3: legs
(This requires 3 cameras and 3 computers!)
 > Differently colored frames (Godard's "Contempt" / "Le mépris" style lol)
 > The girl is "typographed": lighten by a video-beamer (projector) if 
available: light, images and texts (screen surface would be the skin, 
the clothes and the wall in the background). Magical transparency etc.
 > The girl would wear red cape at times.
 > Live stream would be the sound environnement (spoken interactions 
between girls, with the technical assistant(s) and sounds from their 
surroundings).
 > A different girl could appear at times (i mean that they could 
switch, on location of the shooting: people on livesjasmin.com often 
operate from studios where different people work)
 > nb: I talked with Chloé who is enthusiastic for doing this part! And 
in bruxelles it would be totally feasible without big organisation fuzz 
to have 3 cameras and 3 computers for the 3 part upload. So we can 
assume that brussels could do this sort of complicate part :-)


MIDDLE LAYER (frames 4,5,6) - services
 > The Hunter, The Wolf, and other persons/things (grandmother? Dolls? 
Objects? central frame 5 could be a webcam highway all the time?) / 
could be the people through wich the tale is carried and transmitted / 
could be the people looking at the kind of peepshow / the forest / the house
 > The Wolf: a person could be disguised in wolf, wild and chaotic. 
Undecipherable. Basic instincts.
 > The Hunter: could be disguised with mustache, hat with feather, 
riffle. Could spend a lot of time cleaning his gun.


LOWER LAYER (frames 7,8,9) - goods
The transmission operator (frame 8) + 2 text frames (frames 7 + 9)
 > The Operator frame is played at the Mapping festival. He is in front 
of an old-fashionned greyish computer screen.He is the person managing 
the live Skype chat (that should be projected at Mapping). He has a desk 
with a lot of stuff around, like a police inspector (i am thinking of 
the police office at the end of "The Usual Suspects: crowded with 
pictures, notes, etc - here all these documents would relate to the LRRH 
case of course).
 >  black and white image only.
 > The Operator becomes mad when the frames change of place, when it is 
chaotic. Hence he would appear as a puppet-master whose puppets aren't 
obedient. Also as the manager of the webcam girl. I imagine he would 
have a eastern-mafia feel.
 > He gives orders at time to the girl of top layer (with text).

NARRATIVE
So far i've described mostly stuctural elements. But i feel we need 
actions. Like sketches. Series of little interactions (like in the last 
performance when there's the "murder with the typingmachine" betwwen 2 
frames). What will be doing :-) ?
This needs a lot of development. Here are a couple ideas:

 > When the Wolf catches the LRRH (by going up a ladder and exiting his 
frame by the top? bof...) , a curtain could come down on the LRRH (like 
when they do a private show in web-peepshows - well of course the morale 
of the the LRRH story can be read as: the LRRH shall have a lover (the 
wolf) before marriage (with the hunter)...

 > END: for the grand finale, i imagine with pleasure a big showdown 
(after a tense preparation time, like in westerns), with everybody 
shooting against everybody, and constantly changing of frames, so it 
would be a shooting chaotic ballet (before everybody going back to it's 
orîginal frame for the salutes - where the assistants also come in the 
image to salute).

 > There should be a moment of eating alltogether.

 > Text: in 1 text frame, the text could be a chat session between the 
LRRH, the hunter and the Wolf. They could be rambling over their great 
memories of they recall from their story, and the operator could be 
there as well (live on the skype-chat and somebody remote would take 
care to upload the logs into the text frame or we could write this chat 
session in advance and somebody would upload it as well.)
 (n3krozoft did a performance with images of a chatter and the chat: 
http://www.n3krozoft.com/projects/art/lol/index.html\#
In the other text frame, something totally different could be going on.

 > I guess that to emphasize the actual LRRH tale, we could upload 
images from books / illustrations / drawings / dolls / films / comics 
etc. on the theme - or not on the theme actually :-)

 > SOUND:  i imagine BliscappenVonMaria could do "musique descriptive", 
i mean develop expressionistic themes around the themes of the LRRH.

////////

I believe this structure is an invitation to a lot of playfullness and 
inventions.
For instance the roles between the LRRH and the Wolf(es) can be inverted 
at some point. The LRRH could turn into a big eater, in a very 
disgusting way (dirty or jung food only, and alcohol) - well just a 
random thought.

Please don't hesitate to tell clearly your critics. If negative, please 
propose other concrete directions. Or if you wish to bring it to an all 
other level or direction, let's share ideas!


///////

PARTICIPANTS \& DATES:


Confirmed for Festival de la Imagen (but we still need a date):
 > Boris (i'm available at almost any time this week)
 > Paula


Confirmed for Mapping (19th April, we don't know at what time yet):
 > Boris + Manu team
 > Chris
 > Chloé
 > Bliscappen music band

//////////

See you soon,
Boris

}
}
\subject[mail:35]{35 - new script proposal}{
\it{Judy Nylon knickerbockerjuju\high{\ref[p][mail:15].15, \ref[p][mail:28].28, \ref[p][mail:30].30, \ref[p][mail:31].31, \ref[p][mail:32].32, \ref[p][mail:33].33} --- 17.03.2008}

\tf{
This is exciting.....I must still call around for a camera; I still have nothing but my Mac built-in...but I have built a doll house (thinking about Duchamp's valise) in which, with surreal sized objets relating to the Red Riding Hood story, I can build the emotive current of the story with signifiers. I am attaching jpgs. The dollhouse has six rooms and I can access the objets (barely) sticking my hands (with long stick on red fingernails) through the windows on the front. I hope you all like this; I think the smaller (rooms) boxes within on of the larger Aether boxes will give an odd scale/depth sensation that will relate to the body spread across three frames. The tits in the picture are tension toys for executives (weird to touch/squeeze).

The test of the new Aether page looks good.  I think doing the same performance twice is a good move. i will look at the references below. and also check about the web cams ....And if we think this dollhouse idea will work to show the elements of the myth in an emotional/ mental landscape , I will continue to collect ephemeria related to the story. I have watched an artist Laurie Simmons use cutouts from magazines propped up like paperdolls \& moved in these scenerios....it is very dream-like.

And yes, I am glad to have anything to help my half hour lecture on Aether. 
The doll house is out in my own dollhouse sized apartment where I can see it and think...so feed back would be good. Also, are we sticking to having a single source for sound or not?   Great script start..... Judy

}
}
\subject[mail:36]{36 - updated may 23rd performance order/shifts Sunsets Sunrises}{
\it{christiaan cruz fe2cruz\high{\ref[p][mail:38].38, \ref[p][mail:42].42, \ref[p][mail:43].43, \ref[p][mail:60].60, \ref[p][mail:74].74} --- 19.05.2009}

\tf{
12 confirmed aethernaut streams
2 event locations/presenters
3 groups of 2 hour shifts between 2-7pm Colombia time
2 possible sunsets
1 possible sunrise
5 continents
8 timezones
11 cities
according time zones, possible sunsomething \& list request
feel free to trade times if needed

1st Group 2 hours 2PM Colombia Time
1.UTC/GMT+2: Cym, Presenting in Croatia
2.UTC/GMT+3: Mari Keski-Korsu. Tunnila Village, Finland possible @ SunDown 10PM?
3.UTC/GMT+2: Manuel Schmalstieg. Neuchâtel, Switzerland


2nd Group start @ 3:30PM Colombia Time
4.UTC/GMT+2: Frauke Frech, Kiel, Germany
5.UTC/GMT+2: Chloé Cramer. Brussels,Belgium
6.UTC/GMT+2: Boris Kish, Brussels,Belgium
7.UTC+9:30: Vinny Bhagat, Adelaide, South Australia Sunrising 6:13am

Last Group start @ 5:00PM Colombia Time
8.UTC/GMT-6: Alexat0r,Wisconsin or Colorado, USA
9.UTC/GMT-7: Christiaan Cruz. Yorba Linda, California
10.UTC/GMT-5: Paula Vélez. Streaming and Projecting in Medellín, Colombia

Continuous Group full 5 hours
11.UTC/GMT+5:30: Dhanya Pilo. Mumbai, India (15 min intervals)
12.UTC/GMT-4: Judy Nylon. New York. USA (30 min Intervals) Sunseting 7:14pm

http://www.worldtimeserver.com/convert\_time\_in\_CO.aspx?y=2009\&mo=5\&d=23\&h=14\&mn=0
http://www.srrb.noaa.gov/highlights/sunrise/sunrise.html

---
}
}
\subject[mail:37]{37 - correct spelling, names etc?}{
\it{Mari Keski-Korsu mkk\high{\ref[p][mail:62].62, \ref[p][mail:66].66} --- 20.05.2009}

\tf{
Hello,

Sorry for not being able to skype lately. Could someone write a short, 
simple summary about the actions done simultaneously? I won't be able to 
see or hear you due to the slow internets and this would help. There was 
something about luck in group changes and then there was something about 
sunset? I'll be able to stream a little bit about the sunset, it is 
setting behind the trees around the time of the my stream. I can of 
course try to climb on the roof of the house (a bit scared about that). 
What is the idea with the sunset? Is it about just showing a sunset what 
ever way we feel like or is there some sort of concept like everyone is 
trying to show it from same angles or something?

mkk
---
}
}
\subject[mail:38]{38 - skype summary main poimts}{
\it{christiaan cruz fe2cruz\high{\ref[p][mail:36].36, \ref[p][mail:42].42, \ref[p][mail:43].43, \ref[p][mail:60].60, \ref[p][mail:74].74} --- 20.05.2009}

\tf{
Mari \& all the Aethers,

To make it easier I think rather than luck they mean: Chance

Props of luck gambling or chance ect:
Dice
drawing lots - sticks
throwing bones
games of chance
play
gaming imagery
Aethers Aleatoric

synchronization will also happen by chance
hopefully we can work with you via skype
otherwise we can also text your phone messages SMS
we can instruct you to add colors or adjust your stream
simple instructions like its too busy or a cooler or warmer color
likewise if skype is too much for your connection you may SMS me too
213-806-6962
free via http://www.gizmosms.com/

we'll have 1 person in each group function like a moderator
and try to keep all of the cells even and balanced

sunset/sunrise is just another way to accent our locations
if the trees are in your way don't climb your roof
its not important as not many of us won't have a sunrise or sunset
just have a local paper, a clock or snow
anything live from your location that the rest of us won't have

It seems like a proper rehearsal may be difficult to organize.
too many timezones. However since this is a festival about
imporvisation and our main theme could be
 Moving Images of Chance
then rehearsals may actually be the wrong thing to do.

I suggest all streamers just gather as many objects that they can.
Objects that might support or even oppose the idea of Chance.
Lots of Colors, pictures, books. Be prepared to write or Draw.
Have prisms, Kaliedescopes or lenses to put over your camera.
Try puppets in lights or just as shadows.
Use mirrors to open up your space.
Be aware of the other features available in the patch: cam-links \& effects
have your camera near a nice window to grab imagery outside
or even stream outside in a nice location as long as the light isn't an issue

Everyone has at least 2 hours. So we should all bring lots of things
to play with in front of the camera and be ready to change those things
if your moderator feels your cell needs to be adjusted. Anywone with
luck enough to have connections for proper viewing and listening of
all the streams should try to communicate to everyone how all 9 cells
are working and if any adjustments need to be made.

---
}
}
\subject[mail:39]{39 - personnal summary}{
\it{chloé cramer chloecramer\high{\ref[p][mail:44].44, \ref[p][mail:53].53} --- 20.05.2009}

\tf{
*Those are my notes about the content of saturday's jam *:
playing cards
about luck : being lucky (can lead to any personal interpretation)
clocks (time zone, time on location)

following the music : improvising on the music.

abstract lines ...


*Judy wrote :*
**
Nature Weightless Wet Anxious Travel
Hot Languid Food Soft Money Couples/Pairs
Dance Industrial Ancient Magic Cool Depth
Focal Shift
NECESSITY TO PREPARE DIFFERENT KIND OF PROPS, ACCORDING TO THE SUBJECT.
I would like to add that the mutual watching of other frames should not be
forgotten, so that we really play together (if there is no master of
ceremony). Maybe some frames could also be passive or still, or showing a
slow minimalistic action, in order to llow other frames to catch the viewers
attention. Let's consider each other as an orchestra and trully imagine what
it sounds like when all musicians are playing solo in the same time...
by the way, is there a leader of the performance or not ?!
See you soon
chloé
---
}
}
\subject[mail:40]{40 - resume, chat and ideas for JAM}{
\it{Paula Vélez Bravo ciruela\high{\ref[p][mail:19].19, \ref[p][mail:24].24, \ref[p][mail:25].25, \ref[p][mail:26].26, \ref[p][mail:46].46, \ref[p][mail:52].52, \ref[p][mail:59].59} --- 22.05.2009}

\tf{
LIST OF PARTICIPANTS. JAM 23-05-09 (A list of confirmed participants  
and a schedule, division in time/slots)

Final Roll Call, Info, Credits -> Ready To Go!
12 confirmed aethernaut streams
2 event locations/presenters
3 groups of 2 hour shifts between 2-7pm Colombia time
5 continents
8 timezones
11 cities

1st Group 2 hours 2PM Colombia Time
1.UTC/GMT+2: Cym, Presenting in Croatia
  http://networkcultures.org/wpmu/videovortex/video-vortex-split/
2.UTC/GMT+3: Mari Keski-Korsu. Tunnila Village, Finland
3.UTC/GMT+2: Manuel Schmalstieg. Neuchâtel, Switzerland


2nd Group start @ 3:30PM Colombia Time
4.UTC/GMT+2: Frauke Frech, Kiel, Germany
5.UTC/GMT+2: Chloé Cramer. Brussels,Belgium
6.UTC/GMT+2: Boris Kish, Brussels,Belgium
7.UTC+9:30: Vinny Bhagat, Adelaide, South Australia Sunrising 6:13am

Last Group start @ 5:00PM Colombia Time
8.UTC/GMT-6: Axelat0r, Wisconsin, USA
9.UTC/GMT-7: Christiaan Cruz. Yorba Linda, California
10.UTC/GMT-5: Paula Vélez. Streaming and Projecting in Medellín,  
Colombia
   http://translate.google.com/translate?u=http\%3A\%2F\%2Fwww.elmamm.org\%2Fsitio\%2Fprogramacion\%2Ftranssesiones.html\&sl=es\&tl=en\&hl=en\&ie=UTF-8

Continuous Group full 5 hours
11.UTC/GMT+5:30: Dhanya Pilo. Mumbai, India (15 min intervals)
12.UTC/GMT-4: Judy Nylon. New York. USA (30 min Intervals) Sunseting  
7:14pm

http://www.worldtimeserver.com/convert\_time\_in\_CO.aspx?y=2009\&mo=5\&d=23\&h=14\&mn=0
http://www.srrb.noaa.gov/highlights/sunrise/sunrise.html


AETHER9 5 horas de JAM (improvisación y música)
Trans-Sesiones https://transesiones.wordpress.com
Sábado 23 de mayo 2009. Desde las 2:00 pm hasta las 7:00 pm
MAMM. Museo de Arte Moderno de Medellín, Colombia

12 transmisiones de AETHERnautas confirmados
2 eventos dos locaciones/presentadores
3 grupos de 2 horas cada uno entre las 2-7 pm Horario Colombia
5 continentes
8 zonas horarias
11 ciudades

aqui la hora del jam en su zona horaria:
http://www.worldtimeserver.com/convert\_time\_in\_CO.aspx?y=2009\&mo=5\&d=23\&h=14\&mn=0
http://www.srrb.noaa.gov/highlights/sunrise/sunrise.html

SET UP AND EQUIPEMENTS IN MEDELLIN for streaming and projection.

1. Two screens (one for skype chat and patch the other for MAIN AETHER  
WINDOW)
2. Two projectors.
3. Sound transmission   (i would like to know MANU, if that sound  
patch i'm using could have a doble transmission to another channel  
where quality could be maximum as Chris proposed?)

http://giss.tv:8000/aether.mp3

4. Video channels transmission:
http://1904.cc/
http://www.livestream.com/transsesiones
http://www.ustream.tv/channel/transsesiones

RECORDING.
Who will record the hole performance?
We should maybe asign different people for different moments.. it will  
be long!

Who will record sound?
who will record video?

I will try to do it here, but ... cool if someonelse do it.
one at cd quality and the other really low


IDEAS for JAM we talked in last chats

At 2:00 pm (COL time) we will try to start at time with this jam

CYM gives the signal, in a moment of *highter intensity* will start.  
Just need to be sure everyone is clear and knows they must write  
"greetings to split" when Cym gives that command around 20minutes into  
our performance

and then 'greetings to split', in different languages.
(Like, projection in Croatia. Videovortex Festival. Cym, hello from  
Vorva Linda, California, Here  is the sun in Australia, C'est le  
printemps à Brusels,...


1.Playing cards, dice, coins, casino...
about luck : being lucky --> astrology, karma
clocks.  could be a nice "leitmotiv", a clock , like an old alarm. to  
use it in BREAKS.

We say the comand: CLOCKS and everyone put clocks in their frames.

2. Drawings, abstract lines, B\&W

3. Ambience, landscapes, and using remote webcams.

4. Simulating people making, "Foley sound" . Crunching, steps, claps,  
hammers, using in the image objects that makes think about sound!

5. Simulating playing instruments, following music.

6.Adapting to less frames.

7.Words to play with:

Luck Nature    Weightless   Wet   Anxious   Travel
Hot   Languid   Food   Soft   Money   Couples/Pairs
Dance   Industrial   Ancient   Magic   Cool   Depth
Focal Shift

8. spaceships, tesla, transmision, theremyn... representation of  
magnetic fields..

9. Toys, objects, playhouse

10. try to play simon says.

11. use red, green, bleu patch colors, and contrast B\&W images... some  
times of course colored and what ever you want of course....


THINK about a moment of a little perfomance for 30 minutes, to make  
muscicians do the soundtrack live.... (aether9 making images and  
muscicians following)



the interface with 9 frames
dissolve interface
  sometimes some of the frames black
abstract lines

the teams themselves can decide how to lay things out
hell they can move around in just one cell if they want too



IDEAS IN THE AIR of  posible STRUCTUREs FOR THE JAM
For the JAM I have being talking to musicians to see if they have an  
structure to follow. Because i think it is very nice to have knoledge  
of that to prepare something for aether9.

1. One of the musicians will base his part of the JAM direction  
working with SILENCE in music.
2. Another one will work with a synth and use, from the computer,  
algoritms like dinamic systems . That will not be repetitive sound  
secuences. Some could take a long time, others not.
random typical aether atmospheres

1. Writing in skype session , needs of noises, voices, music, strings,  
punctual "bruitage".
2. Using colors as we have being doing, in frames we want to have  
particulary "musicalised".Like in SIMON SAYS toy.
http://www.youtube.com/watch?v=2jmVU2eykuI


ambience

sound effects

foley sound
---

}
}
\subject[mail:41]{41 - ok for 22nd}{
\it{Frauke Frech fraukefrech\high{\ref[p][mail:54].54} --- 17.07.2009}

\tf{
alright then, without any rehearsal
-was just wondering...
anyways i like authentic situations better!

when is starting time on 22nd?

frauke
---
}
}
\subject[mail:42]{42 - 22nd call 9pm Manila time}{
\it{christiaan cruz fe2cruz\high{\ref[p][mail:36].36, \ref[p][mail:38].38, \ref[p][mail:43].43, \ref[p][mail:60].60, \ref[p][mail:74].74} --- 17.07.2009}

\tf{
For the officical Aether Manila Sextet
Performer's Call time is 9pm Manila
with official show start-time 10pm Manila
get your time here:
http://www.worldtimeserver.com/convert\_time\_in\_PH.aspx?y=2009\&mo=7\&d=22\&h=21\&mn=0


Confirmed performers are:
Manuel presentation + performance
Paula presentation + performance
Mari performance
chloe performance
christiaan presentation + performance moderator/screencap
frauke performance
Vinny performance
Tengal - organizer ASEUM Philippines
?Kim skype presentation
boris maybe performance

>>>>>>>>>>>>>>>>>>>>>>>>>>>>>>>>>>>>>>>>>>>>>>>>>>>>>>>>>>>>>>>>
Presentation meeting is locked in for Sunday Evening Europe
19th of July, Aether skype meeting
11-1pm  Califirnia
1-3pm   Colombia
7-9PM   Highlander Time UK
8-10pm  Brussells
2-4am   Manila/Taipei/HK
other times:
http://www.worldtimeserver.com/convert\_time\_in\_US-CA.aspx?y=2009\&mo=7\&d=19\&h=11\&mn=0
---
}
}
\subject[mail:43]{43 - ASEUM epilogue}{
\it{christiaan cruz fe2cruz\high{\ref[p][mail:36].36, \ref[p][mail:38].38, \ref[p][mail:42].42, \ref[p][mail:60].60, \ref[p][mail:74].74} --- 27.07.2009}

\tf{
Don't fret Chloé,

Manuel experienced the brunt of the term "Filipino Time"
Its an island mentality where things always seem to start
go or arrive later than proposed or expected. I had figured
this would happen so I did not want to burden everyone with
rehearsals and other meetings.

I've been on chat with Tengal since the festival and
everything seems to have worked out for the best. It
really was the biggest international symposium in that
lower region of Asia every assembled and Aether was
a big part of it. All the weird technical communications
issues and lack of proper funding aside it really was
a Success for the region. I have been involved in
earlier symposiums where they were not much more
than a website and mailing list. This one had real
events and exchanges.

They had lots of fun listening to Manuel do his presentation.
Even though we had communications issues and we started
slow the large audience still did enjoy listening to Manuel
count backwards. All those Filipino artists and students
hardly ever hear European accents. Its always Hollywood/TV
american english.

So maybe we could all go to the Philippines in the flesh
next year with any new Grant money? I go there regularly
so I wouldn't need a ticket. I'm sure that free housing
would be available then too.

Also sincere apologies to Vinny who was streaming to
dead frames 7-9. We will be sure to Specify that those
cells are not working when we use the SEXTET. maybe we
could create a warning or flag in the Patch.

Thanks to everyone that participated via streaming and or
just piping in on the list.

now on to -> Berlin \& then cimatics?

-christiaan
---


}
}
\subject[mail:44]{44 - berlin PREPARATION}{
\it{chloé cramer chloecramer\high{\ref[p][mail:39].39, \ref[p][mail:53].53} --- 03.09.2009}

\tf{- HOW?
Determine the way we are going to discuss:
> are we going to have a moderator? (A person who takes care that anyone has
the opportunity to express and also refrains us to spend more time than
necessary on each subject)
> are we going to work in small groups or alltogether?
> How much time do we give for general matters and separate topics / how
much time for the performance developpment itself.
> regarding the performance, knowing how long it takes to develop a
performance, it might be good to define it before the actual meeting and
start working on it before, in order to spend more "practical rehearsing
time" than brain-storming/developpment time.
> should we make a schedule with morning, afternoon and night sessions ?

- PREPARATION OF BERLIN
In order to have clear and efficient discussion, we propose to reflect
before the Meeting. Here are a couple of propositions FOR EVERY PERSON WHO
WILL ATTEND THE BERLIN MEETING (physically or remotly):
> reflect on the above subjects.
> if you have a performance idea, summarize it (setting, story, aesthetic,
or any other feature)
> prepare a small selection of screenshots (maximum 5) of your favorite
aether9 moments (that you find inspiring, beautiful, interesting or
whatever) and explain why in a few words (to be presented and discussed at
the begginging of the Meeting)

!!! THE IDEA IS NOT TO HAVE BIG MAILING-LIST CONVERSATION NOW !!! We have
the chance to have a physical meeting to communicate more directly. The idea
is to prepare this meeting according to the subjects we would like to
discuss, and anyone who has strong wish to discuss any aspect of the aether9
should express it before, so that we are ready to interact.

OK ?
See you soon,
Chloé (\& Boris)
---
}
}
\subject[mail:45]{45 - berlin flat/munich}{
\it{ideacritik idea\high{\ref[p][mail:47].47, \ref[p][mail:49].49, \ref[p][mail:50].50, \ref[p][mail:51].51, \ref[p][mail:58].58, \ref[p][mail:61].61, \ref[p][mail:64].64, \ref[p][mail:68].68, \ref[p][mail:71].71, \ref[p][mail:73].73, \ref[p][mail:75].75} --- 16.09.2009}

\tf{
regarding the flat - i also trust your judgement manu.

in addition, i spoke to cym yesterday (IRL!!) and she might be coming as
well so i think a bigger flat is best -

i also think it would be nice to have us all in the same space - though
that will be extremely intense and ironically diametrically opposite to
our usual relationship : )

audrey
---
}
}
\subject[mail:46]{46 - way to berlin}{
\it{Paula Vélez Bravo ciruela\high{\ref[p][mail:19].19, \ref[p][mail:24].24, \ref[p][mail:25].25, \ref[p][mail:26].26, \ref[p][mail:40].40, \ref[p][mail:52].52, \ref[p][mail:59].59} --- 24.10.2009}

\tf{
Tomorrow I will take my plane,

long trip: MEdellín-Bogotá-Madrid\_Paris\_Berlin.

see you then monday.

bye

Paula
---
}
}
\subject[mail:47]{47 - reports of the 1st 1/2 day in residency :}{
\it{ideacritik idea\high{\ref[p][mail:45].45, \ref[p][mail:49].49, \ref[p][mail:50].50, \ref[p][mail:51].51, \ref[p][mail:58].58, \ref[p][mail:61].61, \ref[p][mail:64].64, \ref[p][mail:68].68, \ref[p][mail:71].71, \ref[p][mail:73].73, \ref[p][mail:75].75} --- 26.10.2009}

\tf{
dear aethernautes,

daily reports will be sent to the list with contents of the day's
discussion. after initial group discussion we will split off into smaller
groups to focus on specific issues detailed below (tentative list). for
those remote participants, if you feel there is a topic that interests you
particularly and want to give feedback on - please tell us - we try to
include remote participation : ) we can do this over IRC for example.

currently present are:
boris, judy, manu, frauke, audrey

laure and paula will be joining us this evening.

we have arrived at \_\_\_\_\_-micro research today. we have enjoyed a superb
vedgetable vindaloo cooked by kati! re-energising us for the hours to come
: )

the main subjects of discussion we foresee are:
-website design (manu will give a presentation on the wordpress website at
14:00 tomorrow)
-technical development (specifics to be determined)
-PR (publication - contents, distribution, flipbook, prints)
-performance (narrative, formal stuff, content, nomadic idea,
storyboarding/script, sets/scenography) > directions? i.e. workshops,
festivals, vjing [proposals: "urban screen", pact, rote fabrik, okno,
mousonturm]
-financial - grants, sponsorships, agent/producer
-saturday's performance (who/what) - dr. mabuse still restaged

question: who would be interested in following the discussion over IRC
(irc.freenode.net \#aether9)? what timeframe and what discussion do you
think is of interest to you?

proposed schedule of the week:

tonight: preparation of slideshow for tomorow's presentation, initial
discussion about saturday's performance

tuesday:
offline till 14:00 - painting and prepararing gallery space.
14:00 : manu's presentation about website
19:00 : presentation

wednesday + thursday (schedule not yet determined)
[chloe and cym arrive in berlin]

friday :
11:00\textasciitilde{}14:00 : interviews with diana mccarthy
rest of the day prepare for saturday's performance

saturday >>> PERFORMANCE

sunday :
offline debriefing with diana mccarthy

that's all folks!

updates coming daily : )

frauke/audrey

---}
}
\subject[mail:48]{48 - Fwd: Video-Performance, Rupert Goldsworthy Gallery}{
\it{1.1 1.1\high{\ref[p][mail:56].56, \ref[p][mail:57].57, \ref[p][mail:69].69, \ref[p][mail:70].70} --- 26.10.2009}

\tf{
FYI, here is another short compiled version of our press-text, can be useful to send around:

*****


rupert goldsworthy gallery is premiering an aethereal performance, a wildly dispersed yet intimate collision of live sounds and images.

Developed by an online networked group of international visual artists/collectives working through Oceania, Asia, the Middle East, Europe, and into the Americas, aether9 is a matrix for video/audio performance, the foundations being  distributed authorship, cross-platform, and 'remote' performance of streaming audio and visual material to a nomadic web interface. 

>From October 26th to 31st, an intensive real-life residency of members of the aether9 group is held at at \_\_\_\_\_-micro research [berlin], where the artists meet in physical space for the first time.


further information:
http://1904.cc/aether/
---
}
}
\subject[mail:49]{49 - [Fwd: daily reports - aetheric activity 29.10.2009]}{
\it{ideacritik idea\high{\ref[p][mail:45].45, \ref[p][mail:47].47, \ref[p][mail:50].50, \ref[p][mail:51].51, \ref[p][mail:58].58, \ref[p][mail:61].61, \ref[p][mail:64].64, \ref[p][mail:68].68, \ref[p][mail:71].71, \ref[p][mail:73].73, \ref[p][mail:75].75} --- 29.10.2009}

\tf{
28.10.2009 - daily report aetheric activity

Morning discussions:

-IRC vs skype discussion (what are the pros and cons)
-division into work groups:
        *the radio show group
        *the saturday performance group
        *PR/publication group

Radio show group:
aether9 was broadcasting on Herpstradio.de in the Berliner Runde radio
show at 19:00 CET @ HKW.
http://1904.cc/aether/2009/news/live-on-backyardradio-berlin/
Link in Herpstradio:
http://herbstradio.org/plan/sendung/8614.html\#Magazin-Berliner\_Runde-Aether9

PR group:
the PR group discussed crucial ‘producibles’ that will be invaluable to
potential up and coming young and wealthy collectors. Such as, the deluxe
collector’s kit (both digital and tangible) amongst numerous other
novelties.

Publication:
an aether9 publication is soon to come, including documentation of
performances, interviews conducted this week in berlin and an essay by
Diana McCarthy. discussions are still going around edition and printing
issues.

Performance group:
Saturday, aether agents will be directed by doctor Nylon and her assistant
from the Rupert Goldworthy Gallery in an attempt to materialise the
Californian. This script is an interpretation of Doctor Mabuse by Fritz
Lang. Further script development to follow.


audrey/aether9 group


---
}
}
\subject[mail:50]{50 - performance group get your ass on IRC}{
\it{ideacritik idea\high{\ref[p][mail:45].45, \ref[p][mail:47].47, \ref[p][mail:49].49, \ref[p][mail:51].51, \ref[p][mail:58].58, \ref[p][mail:61].61, \ref[p][mail:64].64, \ref[p][mail:68].68, \ref[p][mail:71].71, \ref[p][mail:73].73, \ref[p][mail:75].75} --- 29.10.2009}

\tf{
log in to IRC \_\_\_\_ !!!!!!!!!

---
}
}
\subject[mail:51]{51 - mid-way almost the end report}{
\it{ideacritik idea\high{\ref[p][mail:45].45, \ref[p][mail:47].47, \ref[p][mail:49].49, \ref[p][mail:50].50, \ref[p][mail:58].58, \ref[p][mail:61].61, \ref[p][mail:64].64, \ref[p][mail:68].68, \ref[p][mail:71].71, \ref[p][mail:73].73, \ref[p][mail:75].75} --- 30.10.2009}

\tf{
yesterday (thursday 29th):

main groups were performance, future directions and PD patch.

** we have bought aether9.com \& .org domains **

Performance:
the performance will reflect of what has 'impressed' the aether9 meeting
in berlin. for the first time in aether9 history the performance will be
partially streamed from a taxi speeding through berlin.

notes: we will insert more documentary in the fiction.

Future directions >>
festivals - residencies - workshops - tours - agent

Festivals:
*FLOSS Orientated festivals
Interested parties: audrey, Manuel.
*Generic media art festivals:
Presenting aether9 as an installation.
Interested parties: Manuel.
*Art fairs/biennale
Interested parties: Boris

Residencies:
Could be dedicated to development of specific aspects:
-       Software development (with some programmer giving technical support)
-       PureData patching
-       Mobile platform support (Symbian, mobile linux...)

workshops: [2 people minimum]
[*move to PD necessary + formal language and notation should be formulated]
*educational context (in art academies and universities)
-> the WORKSHOP PDF IS NOW READY TO BE CIRCULATED
*performative context (in festivals)
Interested parties: Manu, audrey, Judy, cym

speaking engagements:
Interested parties: Judy

tours:
A group of 3-4 aethernauts could be touring with a performance. Activity
in a Festival or a Workshop could be followed by a couple of other dates.

AGENT/Representation:
Should contact possible representants.
Interested parties: Judy.

Clothing of the performers / Visual identity / in-kind sponsorship
We will propose to fashion designers to sponsorize the aether9 group with
clothes, luggage.
Interested parties: cym, Judy

->PD patch:
working with dynabolic boot CD as way around 'shell' object problem. one
still uploaded to server! (micro success)

from the madly working aether9 team in berlin... also we are on IRC : )

---
}
}
\subject[mail:52]{52 - I add names to the mailing message}{
\it{Paula Vélez Bravo ciruela\high{\ref[p][mail:19].19, \ref[p][mail:24].24, \ref[p][mail:25].25, \ref[p][mail:26].26, \ref[p][mail:40].40, \ref[p][mail:46].46, \ref[p][mail:59].59} --- 30.10.2009}

\tf{
OFFICIAL STATEMENT
FOR IMMEDIATE RELEASE
AETHER AGENTS in Berlin
comming from Geneva, Rotterdam, New York, Brussels, Medellin,
or streaming remotely from Sydney, Yorba Linda

Scheduled for October 31st - 21:00 CET

Aether9, an experiment in collaborative realtime storytelling
is gathering souls in a dramatically online fashion.
BERLIN hosts a physical reunion,
a seance,
workshop,
conspiracy, residency, and eventually a performance.
wtf

ALL SOULS BERLIN >>>

Tommorow night, Saturday, Halloween,
as a culmination of a week long residencey @ \_\_\_\_\_-micro research
aether9 will present a live online performance before an audience in  
the rupert goldsworthy gallery in Berlin.

The wireless building of a mass mind, controlling aether9  
agents ...amongst Berliners unaware.
Aether9 presents an interpretation of Fritz Lang's Doktor Mabuse
streaming from a taxi speeding through Berlin.

Join us online at 21:00 CET --> http://1904.cc

///

about aether9
Aether9 is a collaborative art project exploring the field of realtime  
video transmission. It was initiated in May 2007 during a workshop at  
the Mapping Festival in Geneva, Switzerland. Developed by an  
international group of visual artists and collectives working in  
different locations (Europe, North and South America) and  
communicating solely through the Internet, ther9 is a framework for  
networked video/audio performance, and the collaborative development  
of dramarturgical rules particular to Internet modes of communication.  
The system functions as an open platform for participants of any  
technical level to transmit imagery in real-time and interact through  
a structured narrative performance questioning the issues of presence/ 
absence, remote/local, identity and intimacy in the context of the  
electronic space.

http://1904.cc/aether/

///

residence @ \_\_\_\_\_-micro research [berlin] http://www.1010.co.uk/

/// PERFORMERS

Judy Nylon/Audrey Samson/Paula Vélez/Chloé Cramer/Frauke Frech/Boris  
Kish/Laure Deselys/Cym/Manuel Schmalstieg/Christian Cruz/Vinny Bhagat
---
}
}
\subject[mail:53]{53 - flat joke and other thoughts from c3cil}{
\it{chloé cramer chloecramer\high{\ref[p][mail:39].39, \ref[p][mail:44].44} --- 02.11.2009}

\tf{myself. if you agree with this pattern (or would like to think about it with
me), it could be used for future developpment or for workshops. By the way,
I would be interested to lead workshops to guide performance developpment
part, without taking part myself. Would be interesting!
(In addition, I am now starting a project were 100 people are creating a
part of a parade, me and another guy being the artisitic coordinator who
will eventually "stage direct" and write what comes out from all these
imputs...) :

\_ NEEDS FOR A PERFORMANCE:
* one person doing only continuity chat and looking at the interface
* one person doing music only
* one voice person doing the narratives
* each image-uploader should deal 2 frames maximum, not more, in order to be
really aware and creative, and especially in order to be FASTER.

\_PREPARATION: how to write an efficient script step by step:
* brainstorm around a thema connected with smth actual, happening now.
think of images, not ideas. think of interaction of frames.
>> there are very efficient techniques of brainstorm. I learned some
recently and think we should try them on. you can have a first collection of
homogenoeous idea in not more than 3 hours.
* more accurate brainstorm with maximum 4 people, ending with 4 propositions
for a performance-
*1-2 persons, after discussion, write final proposition
* 1 person write the text (the story to be told, it can be a poem, whatever,
but must be entertaining for the lenght of the performance)
* others organise a general disposition of interface, coherent with the
story and esthetic/ image choices.
* rehearsal: absolutely necessary to do at least one with *everybody*.

thank you all and see you soon aetherians,
Chloé, analisa, c3cil and the others
---
}
}
\subject[mail:54]{54 - meeting}{
\it{Frauke Frech fraukefrech\high{\ref[p][mail:41].41} --- 18.11.2009}

\tf{
>  i would be in on the 21st. (next saturday)
>
> ///
>   

this saturday i'm available around noon/1pm.

in case loads of us can't be there we could also meet in smaller groups, no?
i really need to clear some things within our continuation and also just 
getting in touch with you again.

after this intense week it suddenly feels so withdrawn again...

hope an rdv works out soon

excited  frauke
---







}
}
\subject[mail:55]{55 - meeting contents}{
\it{Vinny Bhagat contact\high{\ref[p][mail:72].72} --- 04.12.2009}

\tf{
hi Aethers ..

Sunday is ideal time for a meeting / checks / rehearsal .. [ please ,  
even if some of you are available ]

Monday performance in Adelaide , we are on at 8.30pm Adelaide time  
and play till 9.30 pm  , ie

Adelaide (Australia - South Australia) Monday, December 7, 2009 at  
8:30:00 PM
Brussels : Monday, December 7, 2009 at 11:00:00 AM

please calculate your local times .. checks will start earlier ..

So far the tests i have done , with the wi fi available in the venue  
[ 1-3 Mb - bandwidth ] and mobile broadband/ USB modem  ( varies but  
often less than equal to 1Mb bandwidth ) ... " Slow is the word .. "

I will be using 3 computers in Adelaide with individual internet  
connections ..

1) Projecting Aethers interface :  i would suggest , please remove  
the audio stream .. Atleast for the interface that we use to stream  
in venue . At the moment , whats streaming on 1904.cc , takes long  
time to playback images properly .. there is a long buffering time ..
For me in Sydney , its totally fine , because i am on 15Mb+ bandwidth ..
For online audience we can make an interface that consists of all the  
content < which needs faster internet ..

2) To playback my AUDIO[ Syd]  in Adelaide  : What are your  
suggestions , i can use Ustream , Live stream or icecast/ Giss Tv  
setup using pd  ..
What you suggest ?

3) Stream AV from the performance space in Adelaide

I will write you more soon , please reply, your thoughts n  
suggestions . Sometime today i will send invites  .. people in  
adelaide are excited and it has been advertised properly over there ..

i have been spending way too much time at work , so apologies for not  
writing much , i read all your emails , and maintain the circuit of  
thoughts , inside me ..

Talk to you all soon , beautiful friends ..
'
*bliss *
\textasciitilde{}vinny
---
}
}
\subject[mail:56]{56 - Adelaide performance}{
\it{1.1 1.1\high{\ref[p][mail:48].48, \ref[p][mail:57].57, \ref[p][mail:69].69, \ref[p][mail:70].70} --- 07.12.2009}

\tf{
Hi all,
for the Adelaide performance in a couple of hours, some important points:

* the frameset has been modified -- i removed the two "weak servers", nr
4 and 5, that we struggled with during the berlin performance. at the
moment i write this, those frames should show black, with an occasional
red or blue monochrome appearing.

If you recently accessed 1904.cc, you might use this direct link to make
sure your browser doesn't show you an old version:
http://1904.cc/live/frameset\_09\_adelaide/

* this means of course that i updated the server-list, included with the
Max patch -- the new list is included with the newest patch, nr 400.

You find it here:
http://1904.cc/\textasciitilde{}aether/kode/max\_image\_upload/

The password of the z7/rar archive is the "old" aether password, the one
that's easier to remember.

Meet you in 10 hours:)


Best,
Manuel
---
}
}
\subject[mail:57]{57 - report: Adelaide performance + future}{
\it{1.1 1.1\high{\ref[p][mail:48].48, \ref[p][mail:56].56, \ref[p][mail:69].69, \ref[p][mail:70].70} --- 08.12.2009}

\tf{
Hi all,

Some short report about the Adelaide performance, to which the Aether9
team contributed:

The project was coordinated by Vinny Bhagat (aka Shivnakaun), who was
streaming live electro-acoustics from his studio in Sydney (transmitting
through giss.tv from a PureData patch).

The stream was (supposedly, I didn't hear it) accompanied live
on-location in Adelaide (at The Wheatsheaf Hotel, 39 George St
Thebarton) by Kym Gluyas on saxophone. The aether9 interface was
projected and visible to the live audience.

The Aethernauts who generated the visuals:
Christiaan Cruz, Dhanya Pilo, ideacritik, Manuel Schmalstieg, Maria Fava

This was the first participation in the aether9 project for Maria Fava
(aka eskoitaus), located in Nice (N3krozoft members will remember her
from the CeC festival last February, where she was VJing).

The visuals were fully improvised. The performers decided to restrict
themselves to a black/white color palette for better graphic coherence.
Also, due to the little number of Aethernauts, only 4-5 frames of the
interface were used.

We didn't encounter any technical / upload speed issues this time.
Except ideacritik who was still unable to upload to frame 5, although
the server feeding this frame has been changed. Very mysterious.

Links to the COMA website (Creative Original Music Adelaide)

http://www.coma.net.au/news.php
http://www.coma.net.au/coma-gigs-past.php
http://www.coma.net.au/band\_page.php?Band\_ID=49

Link to the performance page I created:
http://1904.cc/aether/2009/news/adelaide-performance/

Best regards,
Manuel

---

}
}
\subject[mail:58]{58 - ghost trio - write up....Final (I think)}{
\it{ideacritik idea\high{\ref[p][mail:45].45, \ref[p][mail:47].47, \ref[p][mail:49].49, \ref[p][mail:50].50, \ref[p][mail:51].51, \ref[p][mail:61].61, \ref[p][mail:64].64, \ref[p][mail:68].68, \ref[p][mail:71].71, \ref[p][mail:73].73, \ref[p][mail:75].75} --- 12.02.2010}

\tf{
im going with this one from judy (below).
it emcompasses what you said in french laure and it sounds good.
thanx judy!
alright - ill keep us all updated on the when/how of this upcoming Ghost
Study in amsterdam.
laure - will you participate?? boris? cym?
audrey

>
> ³Ghost Story²  aether9
>
> Adapted by aether9 from Samuel Beckett's 1976 Teleplay for the BBC, "Ghost
> Trio".
>
> Both Beckett and aether9 separate sound from image with a voice-over that
> may or may not be located in any of the action frames. Sustained shots of
> figures flowing through rooms and across frames, alternate with the
> tension
> of extreme detail alliterated intimately. In the aether production color
> overlays chart the emotional temperature as well. Beckett recognized
> televisuality by addressing the audience directly in the voice over.
> Aether9, updates having framed the action in a nexus of simultaneous
> streaming, further abandoning the notion of temporality and location ( for
> both audience and performers) in that which is of the mind's domain. This
> time the Beethoven piece gets dubbed with ambient rain and lightning to
> further establish a possibility of web delivered emotional tonality.
>
> Let me know it anyone has any corrections....judy
>
>
}
}
\subject[mail:59]{59 - ghost trio - found this.}{
\it{Paula Vélez Bravo ciruela\high{\ref[p][mail:19].19, \ref[p][mail:24].24, \ref[p][mail:25].25, \ref[p][mail:26].26, \ref[p][mail:40].40, \ref[p][mail:46].46, \ref[p][mail:52].52} --- 15.02.2010}

\tf{
We're revealing our adaptation of  "Ghost Trio," by Beckett on  
Saturday December 15th. Written in 1975, taped in '76 and televised on  
BBC2 in 1977. It is a parallel universe played out in the same slice  
of time as the first wave of British punk. Beckett, I wish we'd met.

My fellow aethernauts and I would be delighted if you would join us  
live on-line at our first chance to bring this play from TV to a wider  
audience. It doesn't matter what you're wearing and you won't have to  
worry about how you're getting home.

Members of Aether9 will be performing from Medellin, Yorba Linda,  
NYC , Brussels, Geneva, and Paris. The audience, those in the same  
room, will be joined at 'Videomedja 2007' at the Museum of Vojvodina,  
Novi Sad, Serbia, by two Aether9 artists from Austria and Slovinia who  
will host the Q \& A which follows the performance.

There are no complicated programs or hardware involved. It is easier  
than getting out of the house. I hope you will look through our site  
at  www.1904.cc where the collaborative process involved is quite  
transparent, our project laid out, and participants credited. Here we  
will make the links "to set up your computer" so easy that you will  
practically fall through them.

You will need to go through one link to find out what the time will be  
where you are, when it is Saturday at 11:15 PM in Serbia. The second  
link opens the audience viewing window and a third link leads to  
instructions to stream sound.

We all wait inside our distractions to see what is outside time,  
beyond words, without location, and never in complete control …….....  
Judy Nylon
}
}
\subject[mail:60]{60 - Pixelache sound loop files needed}{
\it{christiaan cruz fe2cruz\high{\ref[p][mail:36].36, \ref[p][mail:38].38, \ref[p][mail:42].42, \ref[p][mail:43].43, \ref[p][mail:74].74} --- 18.02.2010}

\tf{
create/record sound files yourself or find them from other websites
(with credits)

FTP your files here
http://cave12.org/\textasciitilde{}aether/archive/2010.03.26.Helsinki

just follow these guidelines:

Installation Sound Loop Guidelines
This is an 8+ hour Public installation so please select your audio files with
these things in mind:

1. Voice reading a text in narratively engaging tone. Can be an essay
about aether theories, or narrative. should be possible to
understand/follow.
2. Very short audio-loop of only a few seconds, that is interesting to
listen to when looping for 1-2 minutes (maybe with slight shift in
volume and speed). should not be too disruptive / rhythmic - it should
be usable as a background for a reading voice.
3. Avoid annoying sounds or standard crowd noise that would just blend
into a large room of talking people.
4. The content relates to the visual material we're streaming
(landscapes of us see below)
5. Ambient synthesizer or radio wave sounds in upper frequencies.


Examples:
space sounds - http://www-pw.physics.uiowa.edu/space-audio/sounds/
aurora sounds - http://www.auroralchorus.com/
space radio - http://www.svengrahn.pp.se/sounds/sounds.htm


”LANDSCAPES OF US” (working title)

This idea is inspired somehow by Vimeo group called ”The Pictures
Don't Move” http://www.vimeo.com/groups/thepicturesdontmove and of
course Judy's idea about the leaves. Idea would be to create
landscapes or sceneries by using different streams: even though each
stream is different and from different location, together they create
a whole scenery (this is something Aether has done with drawing, I
guess, but now it would be live footage). It's pretty simple, here's a
sketch that explains a bit: http://www.artsufartsu.net/IMG\_2438.JPG

But what would be the sceneries, then? Should there be one or more?
Could there be themes like, e.g, earth, wind, water and fire? Urban,
forest, countryside? What is nature anyways? It should just be your
localized nature. Like Judy's idea of shooting out your window. Just
pick some image in your area that you can safely stream the whole
8 hours. Or the aethernaut could have different setups say 8 different
1 hour landscapes. I like this change idea.

As of now the 9 aether frames will be like an exqusite corpes landscape image
the top three sky streams
the middle 3 frames horizon
and the lower 3 earth ground streams
A localized scene from your area is preferred


So far we have confirmed
Mari Keski-Korsu (possibly streaming in on location)

Christiaan Cruz
Manuel Schmalstieg
Judy Nylon
Paula Vélez Bravo
Cym

It'd be nice to 4 or more Aethernauts streaming live landscape
so ask Friends and family about availability on Friday March 26th Helsinki time
Just introduce them via the list and one of us can setup a skype chat date to
help them get used to using the patch if needed.

}
}
\subject[mail:61]{61 - residency agence TOPO current situation}{
\it{ideacritik idea\high{\ref[p][mail:45].45, \ref[p][mail:47].47, \ref[p][mail:49].49, \ref[p][mail:50].50, \ref[p][mail:51].51, \ref[p][mail:58].58, \ref[p][mail:64].64, \ref[p][mail:68].68, \ref[p][mail:71].71, \ref[p][mail:73].73, \ref[p][mail:75].75} --- 19.02.2010}

\tf{
hi all,

i've been in contact with Eva Quintas president of agence TOPO where i
sent a proposal for an aether9 residency about a month ago. the current
proposal is the following: (it is still to be confirmed whether they
accept or not)

5 week residency ending with a performance at the 'geo-web installation'
festival at the Cinematheque.

theme: "lost in montreal"
this theme adresses the mobility of the actor (to be developped in
residency) and the aspect of discovery ('getting lost') linked to
cinematographic explorations of the avant-garde (esthetic). (wink wink
manu : )

schedule:

week 1-3: (audrey @ TOPO - other agents remotely present)
-PD patch development
-developemnt of mobile interface (method to control upload, contrast,
etc., actions with a WII contoller)
week 4-5: (audrey and chris @ TOPO - other agents remotely present)
-explorations and experimentation with mobile interface
-script
-performance

dates (approximate) : mid-may to mid-june 2010
place: montreal

this is perhaps a bad translation and certainly an abrieviated version
which does not convey the extremely fun aspect of this new prospect of
running around town with a magic wand uploading images to the interface.
in any case as soon as/if we have confirmation we'll get down to details.

audrey


}
}
\subject[mail:62]{62 - Set up and concept test of PixelAche 25.3. at 11 am (CET)}{
\it{Mari Keski-Korsu mkk\high{\ref[p][mail:37].37, \ref[p][mail:66].66} --- 23.03.2010}

\tf{
Hi,

Finally got info about set up in Kerava Art Museum. They start already 
in the morning at 8 am (CET), but there won't be electricity for two 
hours, so I think 10 - 11 am (CET) would be a good time for us to start 
testing our work. This is Thursday 25.3.

John Hopkins is interested in joining our installation. He would like to 
stream Arizona sunrise at 14:30-15:00, (30min) CET
Manu, from where should he download the needed files and also the server 
listing?

mkk


}
}
\subject[mail:63]{63 - patch 401}{
\it{Dhanya Pilo studio\high{} --- 24.03.2010}

\tf{
Hi
I missed some of the chats, but wanted to know how we have access to  
these other web cameras?
i see it on the patch, but it is some sort of piracy or web channels?
wow!

cheers,
Dhanya



}
}
\subject[mail:64]{64 - rehearsal}{
\it{ideacritik idea\high{\ref[p][mail:45].45, \ref[p][mail:47].47, \ref[p][mail:49].49, \ref[p][mail:50].50, \ref[p][mail:51].51, \ref[p][mail:58].58, \ref[p][mail:61].61, \ref[p][mail:68].68, \ref[p][mail:71].71, \ref[p][mail:73].73, \ref[p][mail:75].75} --- 24.03.2010}

\tf{
i will be online by 12\textasciitilde{}12:30 CET tomorrow for rehearsal, i'll be there n
available for the afternoon.
see some of you tomorrow!

audrey

ps ill be streaming with cym from utrecht(NL), where the botanical gardens
of utrecht university used to be. the scenery will include dutch greenery,
a stream, probably grey-ish sky and a mansion in the background.

pps this is the webcam of the national dutch weather station (KNMI) which
is almost in the backyard from where we will be streaming:
http://www.knmi.nl/webcam/images/ispy.jpg?20427 (havent tested adress in
patch yet but should work :)



}
}
\subject[mail:65]{65 - (no subject)}{
\it{bk bk\high{\ref[p][mail:0].0, \ref[p][mail:1].1, \ref[p][mail:2].2, \ref[p][mail:3].3, \ref[p][mail:9].9, \ref[p][mail:10].10, \ref[p][mail:29].29, \ref[p][mail:34].34} --- 26.03.2010}

\tf{
Hi list,

i have been totally absent the recent times, but trying to follow the 
list. i will remain mostly silent until end of april.
right now 1904.cc looks and sound heavenly tropical. maybe you are 
rehearsing. it's truly beautifull.

MERDE for today's performance (as we say in french) !

Boris



}
}
\subject[mail:66]{66 - Helsinki says thank you!}{
\it{Mari Keski-Korsu mkk\high{\ref[p][mail:37].37, \ref[p][mail:62].62} --- 27.03.2010}

\tf{
Hi,

Manu, Paula, Audrey, Vinny, Cym, Judy, Chris, Dhanya and John.
Once more, THANK YOU very much for your hard work and amazing creation 
in Pixelache Camp yesterday. It was such a nice experience. I also want 
to apologise once more that I had difficulties on concentrating in 
between the virtual discussion of us and talking about the work with 
people in physical space. Both sides lack in this situation, that's for 
sure. But I hope you won't hate me, still :)

I uploaded some pics to my Flickr account, I will ad them to the Aether9 
group after Paula accepts me as a member of it.
http://www.flickr.com/photos/artsufartsu/sets/72157623587582799/

Bows,
mkk


}
}
\subject[mail:67]{67 - Helsinki says thank you!}{
\it{Cym Net cymnet\high{} --- 27.03.2010}

\tf{
Hello Mari

Thank you for uploading the photos. The festival looks very nice!
After seeing the photos I really think that it is a pity that I couldn't
come...
But it was a very nice experience to stream with Audrey from the garden. It
was so strange to walk around there in this very quiet garden next to a busy
street, making pictures from grass, water and clouds, while trying not to
loose the connection. Somehow it seemed very bizarre to me, especially to do
this live. But it is really nice to get feedback and to see some photos of
the installation with the audience. The setting, the projection inside the
white cube, is really nice!

}
}
\subject[mail:68]{68 - Helsinki says thank you!}{
\it{ideacritik idea\high{\ref[p][mail:45].45, \ref[p][mail:47].47, \ref[p][mail:49].49, \ref[p][mail:50].50, \ref[p][mail:51].51, \ref[p][mail:58].58, \ref[p][mail:61].61, \ref[p][mail:64].64, \ref[p][mail:71].71, \ref[p][mail:73].73, \ref[p][mail:75].75} --- 27.03.2010}

\tf{
On Sat, March 27, 2010 10:28 am, Mari Keski-Korsu wrote:
> Hi,
>

hi

> Manu, Paula, Audrey, Vinny, Cym, Judy, Chris, Dhanya and John.
> Once more, THANK YOU very much for your hard work and amazing creation
> in Pixelache Camp yesterday. It was such a nice experience. I also want
> to apologise once more that I had difficulties on concentrating in
> between the virtual discussion of us and talking about the work with
> people in physical space. Both sides lack in this situation, that's for
> sure. But I hope you won't hate me, still :)
>

all very understandable - its impossible to be totally present in both
physical/virtual : )

> I uploaded some pics to my Flickr account, I will ad them to the Aether9
> group after Paula accepts me as a member of it.
> http://www.flickr.com/photos/artsufartsu/sets/72157623587582799/
>

i especially like:
http://www.flickr.com/photos/artsufartsu/4466814176/
(our new aether9 'do' ? :)

apart from the usual chaos of our performances i appreciated yesterday for
a few reasons. first of all it was nice to be perfoming for so long (some
hours) while people where casually walking in and out of the space (i
imagine) because this meant mkk was giving us 'live' feedback on what is
getting through to people. comments like - 'yes text helps' - or - 'add a
shot of your faces in the loop' - etc.

cym and i were indeed reflecting on what could possibly indicate to
viewers that this is live - it is otherwise just pleasant scenery. ideas
like adding the occasional mugshot, or the computer in 'the wild' are the
start of a list we should elaborate upon.

manu i loved your train and 'wandering' streaming! i also thought that
much of the sound that was streaming yesterday was such a huge improvement
on our tired sound bank. at one point the music coupled with the image was
so fitting - i had a moment - one of those randomly programmed ones...

next perfo coming april 21st in amsterdam - more on that soon.


}
}
\subject[mail:69]{69 - Helsinki says thank you!}{
\it{1.1 1.1\high{\ref[p][mail:48].48, \ref[p][mail:56].56, \ref[p][mail:57].57, \ref[p][mail:70].70} --- 28.03.2010}

\tf{
Hi all,

big thanks to Mari for running the show at Camp Pixelache!

It was a very interesting perfo for me as i was testing a new streaming 
setup:
- little AsusEEE netbook (running the patch in WinXP)
- cheapo USB webcam (Aiptek VGA+)
- mobile USB web access

The first session was a near total failure, as it was raining heavily, 
and I realized that it's impossible to operate laptop+cam while holding 
an umbrella! In addition to that, the botanical garden where I was 
trying to stream from turned out to be in a no-network zone between two 
hills.

During the afternoon sessions, rain stopped, and I got used to the 
process of quickly connecting the USB modem, entering PIN, launching Max 
patch, adjusting cam settings, all in a couple of minutes. And as Cym 
says, it's a very weird and fascinating feeling, sending the live images 
from a quiet natural outside location.

The connection was surpraisingly stable, even in a rolling train there 
were no disconnections. It looks like switzerland has a dense coverage, 
appart from mountainous zones. This opens lots of opportunities for 
outside streaming, and I think the direction of "mobile documentary 
event coverage" is something to explore, where the "realtime" element 
can make real sense.

Also, specially in the later hours of the performance, I think we worked 
out a good sense of composition withing our 9 frames (in terms of 
geometry, colors)... hopefully the sign that there *is* some ongoing 
progress in our familiarity with the tool / framework :)


}
}
\subject[mail:70]{70 - Helsinki}{
\it{1.1 1.1\high{\ref[p][mail:48].48, \ref[p][mail:56].56, \ref[p][mail:57].57, \ref[p][mail:69].69} --- 28.03.2010}

\tf{
more videos:
http://1904.cc/aether/2010/community-news/trains-forest/

i just pasted together a few clips that i had the good idea to record 
while streaming ...

do not forget: the patch *has the ability to record* with the little 
"rec" button under the preview window ... it creates small quicktime 
files that don't take much space on the harddrive, so you can record a 
lot ...

best,
m.

}
}
\subject[mail:71]{71 - rain rain rain}{
\it{ideacritik idea\high{\ref[p][mail:45].45, \ref[p][mail:47].47, \ref[p][mail:49].49, \ref[p][mail:50].50, \ref[p][mail:51].51, \ref[p][mail:58].58, \ref[p][mail:61].61, \ref[p][mail:64].64, \ref[p][mail:68].68, \ref[p][mail:73].73, \ref[p][mail:75].75} --- 12.04.2010}

\tf{
famous/classic/cult film - rainy scene - rain hitting the window - black n
white movie - death - memory ----

anyone have any ideas?

we need rain.

audrey

--

}
}
\subject[mail:72]{72 - 21st april}{
\it{Vinny Bhagat contact\high{\ref[p][mail:55].55} --- 15.04.2010}

\tf{
Hi Aethers
I can be there if needed ..
But I need to understand the script , my role and go through /  
rehearse before the performance..
It will be 5 am/22.4 in Sydney [ 21hour CET/21.4] , and i can be  
there till 7.45 am = 23.45 CET ..
i am reading (now) the notes on the pad..

v
}
}
\subject[mail:73]{73 - perfo - amsterdam - feedback an dso on}{
\it{ideacritik idea\high{\ref[p][mail:45].45, \ref[p][mail:47].47, \ref[p][mail:49].49, \ref[p][mail:50].50, \ref[p][mail:51].51, \ref[p][mail:58].58, \ref[p][mail:61].61, \ref[p][mail:64].64, \ref[p][mail:68].68, \ref[p][mail:71].71, \ref[p][mail:75].75} --- 22.04.2010}

\tf{
good afternoon,

yesterday immediately after the performance maaike (actress), cym and i
had to take everything down in less than 10min. this meant we had to
immediately shut down computers, beamers, etc.

some feedback from organisors:
-beautiful, 'it really worked for me' this kindof slow trance like medium.
(in short we are invited to develop something for museumnacht in november
- more on that in another email)
-maaike was a really great actress (i agree)

other feedback:
-it was really clear that it fell apart (timing, etc).
-interesting how you work with a form (telematic performance) that was
explored in pre-days of internet and then some how dropped out.

for my part i would like to thank you all for participating in this
unpaid, unexpectedly very demanding, and extremely chaotic performance (+
rehearsals). it was not easy - thanks for hanging in there!!! the
performance was certainly not a great success (timing was pretty off) but
considering the chaos and the single rehearsal we did quite all right.
amsterdam was pretty solid thanks to the script with timing. this
countered the lack of timing in the interface.

i think we are at some sort of a x-road. there have been many tensions
mounting, namely: diverging ideas about the future of aether9, possible
directions, where to invest energy, time, etc.

i think it is time to organise a chat meeting in which these things can be
discussed. i myself have many issues and frustrations with this group at
the moment which need to be adressed (especially in light of the upcoming
residency). i think the mailing-list is not the place to do this. the fact
that misscommunication cannot be immediately addressed in a chat channel
can lead to huge unnecessary tensions. i suggest a meeting over the
weekend. late enough that columbia/california dont have to again wake up
before 7am!

so suggestions?
saturday or sunday evening? (lets say 21:00CET?)

lets please keep this discussion off list. after all it is 'public'.

audrey}
}
\subject[mail:74]{74 - included in puredyne?}{
\it{christiaan cruz fe2cruz\high{\ref[p][mail:36].36, \ref[p][mail:38].38, \ref[p][mail:42].42, \ref[p][mail:43].43, \ref[p][mail:60].60} --- 10.05.2010}

\tf{
I'll check it out again for sure. I had a lot of trouble with
purdyne on my laptop but that was years ago. It worked well,
but never got the resolution of my laptop's LCD right. It'll
be interesting to see what has happened since then, but
after we finish the Montreal project.

As an update to that:
I got a wiimote working to control PD on a Kubuntu laptop.
Audrey's got a functioning PDpatch running on her linux box.
We both have working webcams on pd within linux distros.
And everyone must put availablities into the PiratePad
and start contributing:
    http://piratepad.net/HURO52Sh5K
-c

On 5/7/10, alejo <alejoduque at gmail.com> wrote:
>
> On May 8, 2010, at 1:17 AM, ideacritik wrote:
>
>> i can say that this will happen post 26th mai. the schedule for what
>> is to
>> be done here is very tight. certainly though its something i want to
>> look
>> into while im in montreal and can talk to the makers IRL..
>
> good, thats all i was suggesting, to have a look at it.. not just you
> but everyone.
> im convinced theres a lot more of potential there in terms of
> development.
>
> also, i never talked with aquessy about it, as said we talked about
> toonloop (another application that could be integrated in aethers
> workflow.. yes yes not now)
>
> and about liveDVD's or customized puredynes:
> http://planktum.wikidot.com/
>
> cheers,
> /a
>

}
}
\subject[mail:75]{75 - very small report from montreal}{
\it{ideacritik idea\high{\ref[p][mail:45].45, \ref[p][mail:47].47, \ref[p][mail:49].49, \ref[p][mail:50].50, \ref[p][mail:51].51, \ref[p][mail:58].58, \ref[p][mail:61].61, \ref[p][mail:64].64, \ref[p][mail:68].68, \ref[p][mail:71].71, \ref[p][mail:73].73} --- 18.05.2010}

\tf{
hi all,

chris and i are working our asses off (despite the beautiful beer drinking
terasse sitting weather taking over the city). the alpha alpha patch is
pretty much done - wii remote action up next. im half-way through the
script/scenario (will upload very soon). local actor jordan arseneault has
contributed loads to the script/scenario and i think it has tremendous
potential!

**rectification: may 26th showtime: 19h30CET this is confirmed!**

audrey

ps to answer the linux/osx question - we are working on linux exclusively
at the moment because we plan (in the future) to make a boot cd (that is
os independant, that everyone can use regardless of what operating system
they use). so those who want to try it out for now must have a linux
system and some patience to install other libraries and etc - or a
puredyne type boot cd (i havent tried this either). therefore it is
foreseeable that everyone else but the montreal crew ends up using the max
patch for the next gig :)

...that all folks! :) ...




}
}
\stoptext
