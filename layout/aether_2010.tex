\startproduct aether_2010

\starttext

\part[2010]{2010}

\marking[P]{2010}

\stylepiece{622}

\stylemailtitle

ghost trio - write up....Final (I think)

\styleinfos

ideacritik

12.02.2010

\stylemail


im going with this one from judy (below).
it emcompasses what you said in french laure and it sounds good.
thanx judy!
alright - ill keep us all updated on the when/how of this upcoming Ghost
Study in amsterdam.
laure - will you participate?? boris? cym?
audrey

>
> ³Ghost Story²  aether9
>
> Adapted by aether9 from Samuel Beckett's 1976 Teleplay for the BBC, "Ghost
> Trio".
>
> Both Beckett and aether9 separate sound from image with a voice-over that
> may or may not be located in any of the action frames. Sustained shots of
> figures flowing through rooms and across frames, alternate with the
> tension
> of extreme detail alliterated intimately. In the aether production color
> overlays chart the emotional temperature as well. Beckett recognized
> televisuality by addressing the audience directly in the voice over.
> Aether9, updates having framed the action in a nexus of simultaneous
> streaming, further abandoning the notion of temporality and location ( for
> both audience and performers) in that which is of the mind's domain. This
> time the Beethoven piece gets dubbed with \stylerefmailslow{ambient}\stylerefroad{611}{624} rain and lightning to
> further establish a possibility of web delivered emotional tonality.
>
> Let me know it anyone has any corrections....judy
>
>


\stylepiece{623}

\stylemailtitle

ghost trio - found this.

\styleinfos

Paula Vélez Bravo

15.02.2010

\stylemail


We're revealing our adaptation of  "Ghost Trio," by Beckett on  
Saturday December 15th. Written in 1975, taped in '76 and televised on  
BBC2 in 1977. It is a parallel universe played out in the same slice  
of time as the first wave of British punk. Beckett, I wish we'd met.

My fellow aethernauts and I would be delighted if you would join us  
live on-line at our first \stylerefmailthemes{chance}\stylerefroad{582}{49} to bring this play from TV to a wider  
audience. It doesn't matter what you're wearing and you won't have to  
worry about how you're getting home.

Members of Aether9 will be performing from Medellin, Yorba Linda,  
NYC , Brussels, Geneva, and Paris. The audience, those in the same  
room, will be joined at 'Videomedja 2007' at the Museum of Vojvodina,  
Novi Sad, Serbia, by two Aether9 artists from Austria and Slovinia who  
will host the Q \& A which follows the performance.

There are no complicated programs or hardware involved. It is easier  
than getting out of the house. I hope you will look through our site  
at  www.1904.cc where the collaborative process involved is quite  
transparent, our project laid out, and participants credited. Here we  
will make the links "to set up your computer" so easy that you will  
practically fall through them.

You will need to go through one link to find out what the time will be  
where you are, when it is Saturday at 11:15 PM in Serbia. The second  
link opens the audience viewing window and a third link leads to  
instructions to stream sound.

We all wait inside our distractions to see what is outside time,  
beyond words, without location, and never in complete control …….....  
Judy Nylon


\stylepiece{624}

\stylemailtitle

Pixelache sound loop files needed

\styleinfos

christiaan cruz

18.02.2010

\stylemail


create/record sound files yourself or find them from other websites
(with credits)

\stylerefmailtech{ftp}\stylerefroadfwdonly{625} your files here
\hyphenatedurl{http://cave12.org/~aether/archive/2010.03.26.Helsinki}

just follow these guidelines:

Installation Sound \stylerefmailtech{loop}\stylerefroadrwdonly{619} Guidelines
This is an 8+ hour Public installation so please select your audio files with
these things in mind:

1. Voice reading a text in narratively engaging tone. Can be an essay
about aether theories, or narrative. should be possible to
understand/follow.
2. Very short audio-loop of only a few seconds, that is interesting to
listen to when looping for 1-2 minutes (maybe with slight shift in
volume and speed). should not be too disruptive / rhythmic - it should
be usable as a background for a reading voice.
3. Avoid annoying sounds or standard crowd noise that would just blend
into a large room of talking people.
4. The content relates to the visual material we're streaming
(landscapes of us see below)
5. \stylerefmailslow{ambient}\stylerefroadrwdonly{622} synthesizer or radio wave sounds in upper frequencies.


Examples:
space sounds - \hyphenatedurl{http://www-pw.physics.uiowa.edu/space-audio/sounds/}
aurora sounds - \hyphenatedurl{http://www.auroralchorus.com/}
space radio - \hyphenatedurl{http://www.svengrahn.pp.se/sounds/sounds.htm}


”LANDSCAPES OF US” (working title)

This idea is inspired somehow by Vimeo group called ”The Pictures
Don't Move” \hyphenatedurl{http://www.vimeo.com/groups/thepicturesdontmove} and of
course Judy's idea about the leaves. Idea would be to create
landscapes or sceneries by using different streams: even though each
stream is different and from different location, together they create
a whole scenery (this is something Aether has done with drawing, I
guess, but now it would be live footage). It's pretty simple, here's a
sketch that explains a bit: \hyphenatedurl{http://www.artsufartsu.net/IMG\_2438.JPG}

But what would be the sceneries, then? Should there be one or more?
Could there be themes like, e.g, earth, wind, water and fire? Urban,
forest, countryside? What is nature anyways? It should just be your
localized nature. Like Judy's idea of shooting out your window. Just
pick some image in your area that you can safely stream the whole
8 hours. Or the aethernaut could have different setups say 8 different
1 hour landscapes. I like this change idea.

As of now the 9 aether frames will be like an exqusite corpes \stylerefmailslow{landscape}\stylerefroadfwdonly{634} image
the top three sky streams
the middle 3 frames horizon
and the lower 3 earth ground streams
A localized scene from your area is preferred


So far we have confirmed
Mari Keski-Korsu (possibly streaming in on location)

Christiaan Cruz
Manuel Schmalstieg
Judy Nylon
Paula Vélez Bravo
Cym

It'd be nice to 4 or more Aethernauts streaming live landscape
so ask Friends and family about availability on Friday March 26th Helsinki time
Just introduce them via the list and one of us can setup a \stylerefmailchat{Skype}\stylerefroadrwdonly{606} \stylerefmailchat{chat}\stylerefroadfwdonly{656} date to
help them get used to using the \stylerefmailpatch{Patch}\stylerefroad{619}{625} if needed.



\stylepiece{625}

\stylemailtitle

residency agence TOPO current situation

\styleinfos

ideacritik

19.02.2010

\stylemail


hi all,

i've been in contact with Eva Quintas president of agence TOPO where i
sent a proposal for an aether9 residency about a month ago. the current
proposal is the following: (it is still to be confirmed whether they
accept or not)

5 week residency ending with a performance at the 'geo-web installation'
festival at the Cinematheque.

theme: "lost in montreal"
this theme adresses the mobility of the actor (to be developped in
residency) and the aspect of discovery ('getting lost') linked to
cinematographic explorations of the avant-garde (esthetic). (wink wink
manu : )

schedule:

week 1-3: (audrey @ TOPO - other agents remotely present)
-\stylerefmailpatch{pd}\stylerefroadrwdonly{624} \stylerefmailpatch{Patch}\stylerefroadfwdonly{628} development
-developemnt of mobile interface (method to control \stylerefmailtech{upload}\stylerefroad{624}{627}, contrast,
etc., actions with a WII contoller)
week 4-5: (audrey and chris @ TOPO - other agents remotely present)
-explorations and experimentation with mobile interface
-script
-performance

dates (approximate) : mid-may to mid-june 2010
place: montreal

this is perhaps a bad translation and certainly an abrieviated version
which does not convey the extremely fun aspect of this new prospect of
running around town with a magic wand uploading images to the interface.
in any case as soon as/if we have confirmation we'll get down to details.

audrey




\useexternalfigure[69QvmGmd1q0M36fSEcWN3Q==][../PAR-PERFO/2010-03-26\_Pixelache/img/20100323\_5.03.55-PM.jpg]\hbox {\externalfigure[69QvmGmd1q0M36fSEcWN3Q==][width=\getvariable{pageprops}{columnwidth},factor=max]}\blank[image]

\stylepiece{627}

\stylemailtitle

Set up and concept test of PixelAche 25.3. at 11 am (CET)

\styleinfos

Mari Keski-Korsu

23.03.2010

\stylemail


Hi,

Finally got info about set up in Kerava Art Museum. They start already 
in the morning at 8 am (CET), but there won't be \stylerefmailscience{electricity}\stylerefroad{618}{634} for two 
hours, so I think 10 - 11 am (CET) would be a good time for us to start 
testing our work. This is Thursday 25.3.

John Hopkins is interested in joining our installation. He would like to 
stream Arizona sunrise at 14:30-15:00, (30min) CET
Manu, from where should he download the needed files and also the \stylerefmailtech{server}\stylerefroad{625}{629} 
listing?

mkk




\stylepiece{628}

\stylemailtitle

patch 401

\styleinfos

Dhanya Pilo

24.03.2010

\stylemail


Hi
I missed some of the chats, but wanted to know how we have access to  
these other web cameras?
i see it on the \stylerefmailpatch{Patch}\stylerefroad{625}{629}, but it is some sort of piracy or web channels?
wow!

cheers,
Dhanya





\stylepiece{629}

\stylemailtitle

rehearsal

\styleinfos

ideacritik

24.03.2010

\stylemail


i will be online by 12\textasciitilde{}12:30 CET tomorrow for rehearsal, i'll be there n
available for the afternoon.
see some of you tomorrow!

audrey

ps ill be streaming with cym from utrecht(NL), where the botanical gardens
of utrecht university used to be. the scenery will include dutch greenery,
a stream, probably grey-ish sky and a mansion in the background.

pps this is the \stylerefmailtech{webcam}\stylerefroad{627}{638} of the national dutch weather station (KNMI) which
is almost in the backyard from where we will be streaming:
\hyphenatedurl{http://www.knmi.nl/webcam/images/ispy.jpg?20427} (havent tested adress in
\stylerefmailpatch{Patch}\stylerefroad{628}{639} yet but should work :)






\page
\setuplayout[nomargins]
\placefigure[here,force]{none}{\externalfigure[/tmp/OSP\_AETHER9\_\_20100326\_FULL\_3\_0.png][width=\paperwidth,height=\paperheight]}
\placefigure[here,force]{none}{\externalfigure[/tmp/OSP\_AETHER9\_\_20100326\_FULL\_3\_1.png][width=\paperwidth,height=\paperheight]}
\placefigure[here,force]{none}{\externalfigure[/tmp/OSP\_AETHER9\_\_20100326\_FULL\_3\_2.png][width=\paperwidth,height=\paperheight]}
\placefigure[here,force]{none}{\externalfigure[/tmp/OSP\_AETHER9\_\_20100326\_FULL\_3\_3.png][width=\paperwidth,height=\paperheight]}
\page
\setuplayout[reset]

\useexternalfigure[Ky8mksmUHlPd10yQYHgOvw==][../PAR-PERFO/2010-03-26\_Pixelache/img/20100326\_2.23.28pm.jpg]\hbox {\externalfigure[Ky8mksmUHlPd10yQYHgOvw==][width=\getvariable{pageprops}{columnwidth},factor=max]}\blank[image]

\useexternalfigure[dC6RSYKo4/gi6w12bMekfw==][../PAR-PERFO/2010-03-26\_Pixelache/img/20100326\_7.15.14-PM.jpg]\hbox {\externalfigure[dC6RSYKo4/gi6w12bMekfw==][width=\getvariable{pageprops}{columnwidth},factor=max]}\blank[image]


\page
\setuplayout[nomargins]
\placefigure[here,force]{none}{\externalfigure[/tmp/OSP\_AETHER9\_\_20100326\_FULL\_3\_0.png][width=\paperwidth,height=\paperheight]}
\placefigure[here,force]{none}{\externalfigure[/tmp/OSP\_AETHER9\_\_20100326\_FULL\_3\_1.png][width=\paperwidth,height=\paperheight]}
\placefigure[here,force]{none}{\externalfigure[/tmp/OSP\_AETHER9\_\_20100326\_FULL\_3\_2.png][width=\paperwidth,height=\paperheight]}
\placefigure[here,force]{none}{\externalfigure[/tmp/OSP\_AETHER9\_\_20100326\_FULL\_3\_3.png][width=\paperwidth,height=\paperheight]}
\page
\setuplayout[reset]

\stylepiece{634}

\styleinfos
\startinfopar \perfotitle{live at Camp Pixelache}

\infopar 2010-03-26 11:00:00

\infopar Camp Pixelache

\infopar Kerava Art Museum, Helsinki (Finland)

\infopar Christiaan
Cruz,
cym,
Dhanya
Pilo,
ideacritik,
John
Hopkins,
Manuel
Schmalstieg,
Mari
Keski-Korsu,
Paula
Vélez,
Vinny
Bhagat\stopinfopar
\blackrule[color=black, width=65mm, height=0.5pt, depth=0mm]
\styleperfo
DURATIONAL PERFORMANCE PRESENTED AS AN AUDIOVISUAL INSTALLATION IN WHICH AETHERNAUTS CREATED AN EVER CHANGING \stylerefperfoslow{landscape}\stylerefroad{624}{656}.
\blackrule[color=black, width=65mm, height=0.5pt, depth=0mm]

\stylepiece{635}

\stylemailtitle

(no subject)

\styleinfos

bk

26.03.2010

\stylemail


Hi list,

i have been totally absent the recent times, but trying to follow the 
list. i will remain mostly silent until end of april.
right now 1904.cc looks and sound heavenly tropical. maybe you are 
rehearsing. it's truly beautifull.

MERDE for today's performance (as we say in french) !

Boris





\stylepiece{636}

\stylemailtitle

Helsinki says thank you!

\styleinfos

Mari Keski-Korsu

27.03.2010

\stylemail


Hi,

Manu, Paula, Audrey, Vinny, Cym, Judy, Chris, Dhanya and John.
Once more, THANK YOU very much for your hard work and amazing creation 
in Pixelache Camp yesterday. It was such a nice experience. I also want 
to apologise once more that I had difficulties on concentrating in 
between the virtual discussion of us and talking about the work with 
people in physical space. Both sides lack in this situation, that's for 
sure. But I hope you won't hate me, still :)

I uploaded some pics to my Flickr account, I will ad them to the Aether9 
group after Paula accepts me as a member of it.
\hyphenatedurl{http://www.flickr.com/photos/artsufartsu/sets/72157623587582799/}

Bows,
mkk




\stylepiece{637}

\stylemailtitle

Helsinki says thank you!

\styleinfos

Cym Net

27.03.2010

\stylemail


Hello Mari

Thank you for uploading the photos. The festival looks very nice!
After seeing the photos I really think that it is a pity that I couldn't
come...
But it was a very nice experience to stream with Audrey from the garden. It
was so strange to walk around there in this very quiet garden next to a busy
street, making pictures from grass, water and clouds, while trying not to
loose the connection. Somehow it seemed very bizarre to me, especially to do
this live. But it is really nice to get feedback and to see some photos of
the installation with the audience. The setting, the projection inside the
white cube, is really nice!



\stylepiece{638}

\stylemailtitle

Helsinki says thank you!

\styleinfos

ideacritik

27.03.2010

\stylemail


On Sat, March 27, 2010 10:28 am, Mari Keski-Korsu wrote:
> Hi,
>

hi

> Manu, Paula, Audrey, Vinny, Cym, Judy, Chris, Dhanya and John.
> Once more, THANK YOU very much for your hard work and amazing creation
> in Pixelache Camp yesterday. It was such a nice experience. I also want
> to apologise once more that I had difficulties on concentrating in
> between the virtual discussion of us and talking about the work with
> people in physical space. Both sides lack in this situation, that's for
> sure. But I hope you won't hate me, still :)
>

all very understandable - its impossible to be totally present in both
physical/virtual : )

> I uploaded some pics to my Flickr account, I will ad them to the Aether9
> group after Paula accepts me as a member of it.
> \hyphenatedurl{http://www.flickr.com/photos/artsufartsu/sets/72157623587582799/}
>

i especially like:
\hyphenatedurl{http://www.flickr.com/photos/artsufartsu/4466814176/}
(our new aether9 'do' ? :)

apart from the usual chaos of our performances i appreciated yesterday for
a few reasons. first of all it was nice to be perfoming for so long (some
hours) while people where casually walking in and out of the space (i
imagine) because this meant mkk was giving us 'live' feedback on what is
getting through to people. comments like - 'yes text helps' - or - 'add a
shot of your faces in the \stylerefmailtech{loop}\stylerefroad{629}{639}' - etc.

cym and i were indeed reflecting on what could possibly indicate to
viewers that this is live - it is otherwise just pleasant scenery. ideas
like adding the occasional mugshot, or the computer in 'the wild' are the
start of a list we should elaborate upon.

manu i loved your train and 'wandering' streaming! i also thought that
much of the sound that was streaming yesterday was such a huge improvement
on our tired sound bank. at one point the music coupled with the image was
so fitting - i had a moment - one of those randomly programmed ones...

next perfo coming april 21st in amsterdam - more on that soon.




\stylepiece{639}

\stylemailtitle

Helsinki says thank you!

\styleinfos

1.1

28.03.2010

\stylemail


Hi all,

big thanks to Mari for running the show at Camp Pixelache!

It was a very interesting perfo for me as i was testing a new streaming 
setup:
- little AsusEEE netbook (running the \stylerefmailpatch{Patch}\stylerefroadrwdonly{629} in WinXP)
- cheapo USB \stylerefmailtech{webcam}\stylerefroadfwdonly{662} (Aiptek VGA+)
- mobile USB web access

The first session was a near total failure, as it was raining heavily, 
and I realized that it's impossible to operate laptop+cam while holding 
an umbrella! In addition to that, the botanical garden where I was 
trying to stream from turned out to be in a no-\stylerefmailtech{network}\stylerefroadrwdonly{638} zone between two 
hills.

During the afternoon sessions, rain stopped, and I got used to the 
process of quickly connecting the USB modem, entering PIN, launching \stylerefmailpatch{Max}\stylerefroadfwdonly{640} 
patch, adjusting cam settings, all in a couple of minutes. And as Cym 
says, it's a very weird and fascinating feeling, sending the live images 
from a quiet natural outside location.

The connection was surpraisingly stable, even in a rolling train there 
were no disconnections. It looks like switzerland has a dense coverage, 
appart from mountainous zones. This opens lots of opportunities for 
outside streaming, and I think the direction of "mobile documentary 
event coverage" is something to explore, where the "realtime" element 
can make real sense.

Also, specially in the later hours of the performance, I think we worked 
out a good sense of composition withing our 9 frames (in terms of 
geometry, colors)... hopefully the sign that there *is* some ongoing 
progress in our familiarity with the tool / framework :)




\stylepiece{640}

\stylemailtitle

Helsinki

\styleinfos

1.1

28.03.2010

\stylemail


more videos:
\hyphenatedurl{http://1904.cc/aether/2010/community-news/trains-forest/}

i just pasted together a few clips that i had the good idea to record 
while streaming ...

do not forget: the \stylerefmailpatch{Patch}\stylerefroad{639}{658} *has the ability to record* with the little 
"rec" button under the preview window ... it creates small quicktime 
files that don't take much space on the harddrive, so you can record a 
lot ...

best,
m.



\stylepiece{641}

\stylemailtitle

rain rain rain

\styleinfos

ideacritik

12.04.2010

\stylemail


famous/classic/cult film - rainy scene - rain hitting the window - black n
white movie - death - memory ----

anyone have any ideas?

we need rain.

audrey

--



\stylepiece{642}

\stylequoteblock{

\styleinfos

Dhanya Pilo

\stylequote



hi guys,
somethings come up and I am sorry, that I am going to be in Qatar and  
wont be in town or with internet access in doha between the 18th-25th  
april, I have been invited for a Yachting tournament. so I will not be  
able to take part in this Aether session.
Apologies.
warmly,
Dhanya

}

\stylepiece{643}

\stylequoteblock{

\styleinfos

Manuel Schmalstieg

\stylequote



hi Dhanya,
Yachting in Qatar -- that's the most perfect excuse i can imagine to 
escape from a dark rainy Beckett script ...

}

\stylepiece{644}

\stylemailtitle

about aether pad

\styleinfos

Cym Net

13.04.2010

\stylemail


do we have a copy of the material on etherpad? if someone has the link to
the material, I can try to save or export the aether material that's still
on etherpad.

best, Cym
--


\stylepiece{645}

\stylemailtitle

21st april

\styleinfos

Vinny Bhagat

15.04.2010

\stylemail


Hi Aethers
I can be there if needed ..
But I need to understand the script , my role and go through /  
rehearse before the performance..
It will be 5 am/22.4 in Sydney [ 21hour CET/21.4] , and i can be  
there till 7.45 am = 23.45 CET ..
i am reading (now) the notes on the pad..

v


\stylepiece{646}

\stylequoteblock{

\styleinfos

Mari Keski-Korsu

\stylequote



ps. this vulcano-dust-flight-crisis would have made a great
simultaneous aethernaut-journalist event! I love that we can't fly! :)

}

\useexternalfigure[F1IcTBaNwAI7mJz/wUu5DA==][../PAR-PERFO/2010-04-21\_GhostTrio/img/20100421\_01.png]\hbox {\externalfigure[F1IcTBaNwAI7mJz/wUu5DA==][width=\getvariable{pageprops}{columnwidth},factor=max]}\blank[image]

\useexternalfigure[DQfcKQYnOoITOPzI8GkP3Q==][../PAR-PERFO/2010-04-21\_GhostTrio/img/20100421\_04.png]\hbox {\externalfigure[DQfcKQYnOoITOPzI8GkP3Q==][width=\getvariable{pageprops}{columnwidth},factor=max]}\blank[image]

\useexternalfigure[GdpDbNKJ7tXs+MtaP/vYyQ==][../PAR-PERFO/2010-04-21\_GhostTrio/img/20100421\_03.png]\hbox {\externalfigure[GdpDbNKJ7tXs+MtaP/vYyQ==][width=\getvariable{pageprops}{columnwidth},factor=max]}\blank[image]

\useexternalfigure[HfjGVv7yk+sKDlxeGgRGtg==][../PAR-PERFO/2010-04-21\_GhostTrio/img/20100421\_07.png]\hbox {\externalfigure[HfjGVv7yk+sKDlxeGgRGtg==][width=\getvariable{pageprops}{columnwidth},factor=max]}\blank[image]

\useexternalfigure[oLAAzB7TkKBJlEfJ2PRtgg==][../PAR-PERFO/2010-04-21\_GhostTrio/img/20100421\_06.png]\hbox {\externalfigure[oLAAzB7TkKBJlEfJ2PRtgg==][width=\getvariable{pageprops}{columnwidth},factor=max]}\blank[image]

\useexternalfigure[WjGYy6kPlEDRJWrNml/R1Q==][../PAR-PERFO/2010-04-21\_GhostTrio/img/20100421\_00.png]\hbox {\externalfigure[WjGYy6kPlEDRJWrNml/R1Q==][width=\getvariable{pageprops}{columnwidth},factor=max]}\blank[image]

\useexternalfigure[UcL3qiWyY6+xWVvSB0+xFQ==][../PAR-PERFO/2010-04-21\_GhostTrio/img/20100421\_02.png]\hbox {\externalfigure[UcL3qiWyY6+xWVvSB0+xFQ==][width=\getvariable{pageprops}{columnwidth},factor=max]}\blank[image]

\useexternalfigure[ocyJvKjxwoMm9JN6d5lkqA==][../PAR-PERFO/2010-04-21\_GhostTrio/img/20100421\_05.png]\hbox {\externalfigure[ocyJvKjxwoMm9JN6d5lkqA==][width=\getvariable{pageprops}{columnwidth},factor=max]}\blank[image]

\stylepiece{655}

\styleinfos
\startinfopar \perfotitle{GhostStudy Amsterdam}

\infopar 2010-04-21 20:30:00

\infopar IN THE EVIDENCE OF EXPERIENCE – 7 EVENINGS OF LIVE ART

\infopar Arti et Amicitiae, Amsterdam (The Netherlands)

\infopar Christiaan
Cruz,
cym,
ideacritik,
Judy
Nylon,
Manuel
Schmalstieg,
Mari
Keski-Korsu,
Paula
Vélez,
Catalina
Sierra,
Pedro
Hermelin\stopinfopar
\blackrule[color=black, width=65mm, height=0.5pt, depth=0mm]
\styleperfo
AN ADAPTATION OF SAMUEL BECKETT’S 1976 TV PLAY “GHOST TRIO”.
\blackrule[color=black, width=65mm, height=0.5pt, depth=0mm]

\stylepiece{656}

\stylemailtitle

perfo - amsterdam - feedback an dso on

\styleinfos

ideacritik

22.04.2010

\stylemail


good afternoon,

yesterday immediately after the performance maaike (actress), cym and i
had to take everything down in less than 10min. this meant we had to
immediately shut down computers, beamers, etc.

some feedback from organisors:
-beautiful, 'it really worked for me' this kindof \stylerefmailslow{slow}\stylerefroad{634}{670} trance like medium.
(in short we are invited to develop something for museumnacht in november
- more on that in another email)
-maaike was a really great actress (i agree)

other feedback:
-it was really clear that it fell apart (timing, etc).
-interesting how you work with a form (telematic performance) that was
explored in pre-days of internet and then some how dropped out.

for my part i would like to thank you all for participating in this
unpaid, unexpectedly very demanding, and extremely chaotic performance (+
rehearsals). it was not easy - thanks for hanging in there!!! the
performance was certainly not a great success (timing was pretty off) but
considering the chaos and the single rehearsal we did quite all right.
amsterdam was pretty solid thanks to the script with timing. this
countered the lack of timing in the interface.

i think we are at some sort of a x-road. there have been many tensions
mounting, namely: diverging ideas about the future of aether9, possible
directions, where to invest energy, time, etc.

i think it is time to organise a \stylerefmailchat{chat}\stylerefroad{624}{7} meeting in which these things can be
discussed. i myself have many issues and frustrations with this group at
the moment which need to be adressed (especially in light of the upcoming
residency). i think the mailing-list is not the place to do this. the fact
that misscommunication cannot be immediately addressed in a chat channel
can lead to huge unnecessary tensions. i suggest a meeting over the
weekend. late enough that columbia/california dont have to again wake up
before 7am!

so suggestions?
saturday or sunday evening? (lets say 21:00CET?)

lets please keep this discussion off list. after all it is 'public'.

audrey

\stylepiece{657}

\stylemailtitle

perfo - amsterdam - feedback an dso on

\styleinfos

1.1

22.04.2010

\stylemail


hi audrey,

good to read this feedback, and thanks for adressing directly issues of 
future development of the project.

> i think we are at some sort of a x-road. there have been many tensions
> mounting, namely: diverging ideas about the future of aether9, possible
> directions, where to invest energy, time, etc.

agreed. interestingly, the two latest performances we did, pixelache and 
now amsterdam, went into two totally different directions:
- using the interface as canvas, working without a storyline, rather 
exploring rules of visual composition.
- following a precise storyline, with caracters, precise timing, 
narrative interrelation of frames.

i notice that i find the first option is much closer to my own 
interests. so for me, that's definitely the direction i would envision 
to work with. i also wrote about this in the proposal that was sent 
recently to the list:
\hyphenatedurl{http://lists.1904.cc/mailman/private/aether/2010-April/001027.html}


> so suggestions?
> saturday or sunday evening? (lets say 21:00CET?)

possible for me on sunday, but rather at 22:00 CET.

best,
manuel

--

Nylon
14:46
OK in NYC for me 10 AM
manu
14:47
that's 11 AM in NYC
paula vélez
14:47
boris?
manu
14:47
with daylight saving time
fe2cruz
14:47
daylight savings gotta check worldtimeserver.com
Nylon
14:47
oh....manu you know I never get this right....thanks
paula vélez
14:47
for mari the same hour?
Mari Keski-Korsu
14:47
that's ok for me. i'll see you if can climb the tree high enough  i  
have to go now, it was very nice talking to you.
Nylon
14:48
bye, mari
(extrait \hyphenatedurl{http://cave12.org/~aether/archive/list\_archive\_2008/2009-April/000483.html)}



---








\stylepiece{658}

\stylemailtitle

included in puredyne?

\styleinfos

christiaan cruz

10.05.2010

\stylemail


I'll check it out again for sure. I had a lot of trouble with
purdyne on my laptop but that was years ago. It worked well,
but never got the resolution of my laptop's LCD right. It'll
be interesting to see what has happened since then, but
after we finish the Montreal project.

As an update to that:
I got a wiimote working to control \stylerefmailpatch{pd}\stylerefroad{640}{662} on a Kubuntu laptop.
Audrey's got a functioning PDpatch running on her linux box.
We both have working webcams on pd within linux distros.
And everyone must put availablities into the PiratePad
and start contributing:
    \hyphenatedurl{http://piratepad.net/HURO52Sh5K}
-c

On 5/7/10, alejo <alejoduque at gmail.com> wrote:
>
> On May 8, 2010, at 1:17 AM, ideacritik wrote:
>
>> i can say that this will happen post 26th mai. the schedule for what
>> is to
>> be done here is very tight. certainly though its something i want to
>> look
>> into while im in montreal and can talk to the makers IRL..
>
> good, thats all i was suggesting, to have a look at it.. not just you
> but everyone.
> im convinced theres a lot more of potential there in terms of
> development.
>
> also, i never talked with aquessy about it, as said we talked about
> toonloop (another application that could be integrated in aethers
> workflow.. yes yes not now)
>
> and about liveDVD's or customized puredynes:
> \hyphenatedurl{http://planktum.wikidot.com/}
>
> cheers,
> /a
>



\useexternalfigure[/Bpf2bhXHIctEi5jxrxiYQ==][../PAR-PERFO/2010-05-26\_Montreal/img/20100517\_action\_shot\_mtl\_00.jpg]\hbox {\externalfigure[/Bpf2bhXHIctEi5jxrxiYQ==][width=\getvariable{pageprops}{columnwidth},factor=max]}\blank[image]

\useexternalfigure[zyjEaBqNu3VUHzCQ3Q9Nrw==][../PAR-PERFO/2010-05-26\_Montreal/img/20100517\_action\_shot\_mtl\_03.jpg]\hbox {\externalfigure[zyjEaBqNu3VUHzCQ3Q9Nrw==][width=\getvariable{pageprops}{columnwidth},factor=max]}\blank[image]

\useexternalfigure[sKMH01AVrAFNwGUlfQ1GxA==][../PAR-PERFO/2010-05-26\_Montreal/img/20100517\_action\_shot\_mtl\_01.jpg]\hbox {\externalfigure[sKMH01AVrAFNwGUlfQ1GxA==][width=\getvariable{pageprops}{columnwidth},factor=max]}\blank[image]

\stylepiece{662}

\stylemailtitle

very small report from montreal

\styleinfos

ideacritik

18.05.2010

\stylemail


hi all,

chris and i are working our asses off (despite the beautiful beer drinking
terasse sitting weather taking over the city). the alpha alpha \stylerefmailpatch{Patch}\stylerefroadrwdonly{658} is
pretty much done - wii remote action up next. im half-way through the
script/scenario (will \stylerefmailtech{upload}\stylerefroad{639}{2} very soon). local actor jordan arseneault has
contributed loads to the script/scenario and i think it has tremendous
potential!

**rectification: may 26th showtime: 19h30CET this is confirmed!**

audrey

ps to answer the linux/osx question - we are working on linux exclusively
at the moment because we plan (in the future) to make a boot cd (that is
os independant, that everyone can use regardless of what operating system
they use). so those who want to try it out for now must have a linux
system and some patience to install other libraries and etc - or a
puredyne type boot cd (i havent tried this either). therefore it is
foreseeable that everyone else but the montreal crew ends up using the \stylerefmailpatch{Max}\stylerefroadfwdonly{670}
patch for the next gig :)

...that all folks! :) ...






\useexternalfigure[R33kpJKYWeEZ0MNd6HiOkw==][../PAR-PERFO/2010-05-26\_Montreal/img/20100526\_DSC\_0092-montreal.jpg]\hbox {\externalfigure[R33kpJKYWeEZ0MNd6HiOkw==][width=\getvariable{pageprops}{columnwidth},factor=max]}\blank[image]

\useexternalfigure[3xK6MVBCtPK2nzATqTVAFw==][../PAR-PERFO/2010-05-26\_Montreal/img/20100526\_ae-fullscreen3.jpg]\hbox {\externalfigure[3xK6MVBCtPK2nzATqTVAFw==][width=\getvariable{pageprops}{columnwidth},factor=max]}\blank[image]

\useexternalfigure[2LqMdOqt1HySEVvnM6r4UQ==][../PAR-PERFO/2010-05-26\_Montreal/img/20100526\_ae-fullscreen2.jpg]\hbox {\externalfigure[2LqMdOqt1HySEVvnM6r4UQ==][width=\getvariable{pageprops}{columnwidth},factor=max]}\blank[image]


\page
\setuplayout[nomargins]
\placefigure[here,force]{none}{\externalfigure[/tmp/OSP\_AETHER9\_\_20100526\_FULL\_ae-fullscreen1\_0.png][width=\paperwidth,height=\paperheight]}
\placefigure[here,force]{none}{\externalfigure[/tmp/OSP\_AETHER9\_\_20100526\_FULL\_ae-fullscreen1\_1.png][width=\paperwidth,height=\paperheight]}
\placefigure[here,force]{none}{\externalfigure[/tmp/OSP\_AETHER9\_\_20100526\_FULL\_ae-fullscreen1\_2.png][width=\paperwidth,height=\paperheight]}
\placefigure[here,force]{none}{\externalfigure[/tmp/OSP\_AETHER9\_\_20100526\_FULL\_ae-fullscreen1\_3.png][width=\paperwidth,height=\paperheight]}
\page
\setuplayout[reset]

\useexternalfigure[RjlDhyO5hk4IEFJBvF53ig==][../PAR-PERFO/2010-05-26\_Montreal/img/20100526\_DSC\_0038-montreal.jpg]\hbox {\externalfigure[RjlDhyO5hk4IEFJBvF53ig==][width=\getvariable{pageprops}{columnwidth},factor=max]}\blank[image]

\useexternalfigure[+wp3q7bZDkpAJoLhw1xevw==][../PAR-PERFO/2010-05-26\_Montreal/img/20100526\_ae-fullscreen4.jpg]\hbox {\externalfigure[+wp3q7bZDkpAJoLhw1xevw==][width=\getvariable{pageprops}{columnwidth},factor=max]}\blank[image]

\useexternalfigure[kJpuf9vMc/HwxEPRxCkw0w==][../PAR-PERFO/2010-05-26\_Montreal/img/20100526\_ae-fullscreen5.jpg]\hbox {\externalfigure[kJpuf9vMc/HwxEPRxCkw0w==][width=\getvariable{pageprops}{columnwidth},factor=max]}\blank[image]

\stylepiece{670}

\styleinfos
\startinfopar \perfotitle{Montréal aether9 presentation}

\infopar 2010-05-26 19:30:00

\infopar SORTIR DE L’ÉCRAN

\infopar Société des arts technologiques [S.A.T.], Montréal (Canada)

\infopar Christiaan
Cruz,
cym,
ideacritik,
Manuel
Schmalstieg,
Mari
Keski-Korsu,
Paula
Vélez,
Vinny
Bhagat\stopinfopar
\blackrule[color=black, width=65mm, height=0.5pt, depth=0mm]
\styleperfo
A PRESENTATION ABOUT THE CONCEPTUAL AND TECHNICAL DEVELOPMENTS RESULTING FROM A RESIDENCY AT AGENCE TOPO. EXPLORATION OF \stylerefperfoslow{landscape}\stylerefroad{656}{49} POSSIBILITIES OF TELEMATIC PERFORMANCE AS WELL AS AETHER9'S NEW \stylerefperfopatch{pd}\stylerefroadrwdonly{662} \stylerefperfopatch{Patch}\stylerefroadfwdonly{47} AND WIIMOTE CONTROL OPTIONS.
\blackrule[color=black, width=65mm, height=0.5pt, depth=0mm]

\stoptext

\stopproduct