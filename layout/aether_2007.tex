\startproduct aether_2007

\starttext

\part[2007]{2007}

\marking[P]{2007}

\useexternalfigure[9SVrAc5PQHRHrtdujclISQ==][../PAR-PERFO/2007-05-03\_MappingFestival/img/aether5\_9frames.jpg]\hbox {\externalfigure[9SVrAc5PQHRHrtdujclISQ==][width=\getvariable{pageprops}{columnwidth},factor=max]}\blank[image]

\useexternalfigure[/q80vfrvZvM67gUQTbV8vQ==][../PAR-PERFO/2007-05-03\_MappingFestival/img/aether1\_ekran.jpg]\hbox {\externalfigure[/q80vfrvZvM67gUQTbV8vQ==][width=\getvariable{pageprops}{columnwidth},factor=max]}\blank[image]

\useexternalfigure[fzPX0WHNUQVwTNk2SVrwiA==][../PAR-PERFO/2007-05-03\_MappingFestival/img/aether6\_.jpg]\hbox {\externalfigure[fzPX0WHNUQVwTNk2SVrwiA==][width=\getvariable{pageprops}{columnwidth},factor=max]}\blank[image]

\useexternalfigure[NGnJ+euWfy2TeYB1/d0CIg==][../PAR-PERFO/2007-05-03\_MappingFestival/img/aether4perform3.jpg]\hbox {\externalfigure[NGnJ+euWfy2TeYB1/d0CIg==][width=\getvariable{pageprops}{columnwidth},factor=max]}\blank[image]

\useexternalfigure[hMawoObnHsqocsBj9H6/Cw==][../PAR-PERFO/2007-05-03\_MappingFestival/img/aether3chat.jpg]\hbox {\externalfigure[hMawoObnHsqocsBj9H6/Cw==][width=\getvariable{pageprops}{columnwidth},factor=max]}\blank[image]

\useexternalfigure[iWPig2JAT/MBkYxGnCiE6w==][../PAR-PERFO/2007-05-03\_MappingFestival/img/aether5.jpg]\hbox {\externalfigure[iWPig2JAT/MBkYxGnCiE6w==][width=\getvariable{pageprops}{columnwidth},factor=max]}\blank[image]

\useexternalfigure[Nhjp4Vlsx+P9YyxVXTe2fA==][../PAR-PERFO/2007-05-03\_MappingFestival/img/aether6.5.jpg]\hbox {\externalfigure[Nhjp4Vlsx+P9YyxVXTe2fA==][width=\getvariable{pageprops}{columnwidth},factor=max]}\blank[image]

\useexternalfigure[VmreqK8KJBfCZXxo9qKEjg==][../PAR-PERFO/2007-05-03\_MappingFestival/img/aether1\_act1.jpg]\hbox {\externalfigure[VmreqK8KJBfCZXxo9qKEjg==][width=\getvariable{pageprops}{columnwidth},factor=max]}\blank[image]

\useexternalfigure[4xcYIh0Kr70n+24uDn8RTg==][../PAR-PERFO/2007-05-03\_MappingFestival/img/aether7\_logs.jpg]\hbox {\externalfigure[4xcYIh0Kr70n+24uDn8RTg==][width=\getvariable{pageprops}{columnwidth},factor=max]}\blank[image]

\useexternalfigure[eb5AZbpV4CAGQzFU6RKSaw==][../PAR-PERFO/2007-05-03\_MappingFestival/img/20070503\_aether1\_blackframes.jpg]\hbox {\externalfigure[eb5AZbpV4CAGQzFU6RKSaw==][width=\getvariable{pageprops}{columnwidth},factor=max]}\blank[image]

\useexternalfigure[cshPKrypDtsLqW49+eV8aQ==][../PAR-PERFO/2007-05-03\_MappingFestival/img/aether4.jpg]\hbox {\externalfigure[cshPKrypDtsLqW49+eV8aQ==][width=\getvariable{pageprops}{columnwidth},factor=max]}\blank[image]

\useexternalfigure[9pb/XPB0LpW8jdrXf6Le2Q==][../PAR-PERFO/2007-05-03\_MappingFestival/img/aether4perfromers.jpg]\hbox {\externalfigure[9pb/XPB0LpW8jdrXf6Le2Q==][width=\getvariable{pageprops}{columnwidth},factor=max]}\blank[image]

\useexternalfigure[XEsBLwUj8zb7TcL2i0KtKg==][../PAR-PERFO/2007-05-03\_MappingFestival/img/aether\_9frames.jpg]\hbox {\externalfigure[XEsBLwUj8zb7TcL2i0KtKg==][width=\getvariable{pageprops}{columnwidth},factor=max]}\blank[image]

\useexternalfigure[KeYEFs/cbJ5xsYs4EP72VA==][../PAR-PERFO/2007-05-03\_MappingFestival/img/geneveperfo3.png]\hbox {\externalfigure[KeYEFs/cbJ5xsYs4EP72VA==][width=\getvariable{pageprops}{columnwidth},factor=max]}\blank[image]

\useexternalfigure[QeTT83jnTesIFmxkYV/C1A==][../PAR-PERFO/2007-05-03\_MappingFestival/img/geneveperfo4.png]\hbox {\externalfigure[QeTT83jnTesIFmxkYV/C1A==][width=\getvariable{pageprops}{columnwidth},factor=max]}\blank[image]

\stylepiece{32}

\styleinfos
First aether9 performance
Mapping Festival 2007
Bâtiment d’Art Contemporain (BAC), Geneva (Switzerland)
Alan
Bogana,
Alejandro
Duque,
Ariane
Téodoridis,
Boris
Kish,
Chloé
Cramer,
Christiaan
Cruz,
cym,
Daphné
Dornbierer,
ideacritik,
Manuel
Schmalstieg,
Nathalie
Fougeras,
Oh
Eun
Lee,
Paula
Vélez,
Yoann
Trellu
\blackrule[color=black, width=65mm, height=0.5pt, depth=0mm]
\styleperfo
LINKING NINE PERFORMERS IN VARIOUS REMOTE LOCATIONS AROUND THE GLOBE, THIS PERFORMANCE MARKED THE BEGINNING OF THE AETHER9 PROJECT.
\blackrule[color=black, width=65mm, height=0.5pt, depth=0mm]

\stylepiece{33}

\stylemailtitle

test message

\styleinfos

ms

04.05.2007

\stylemail





\stylepiece{34}

\stylemailtitle

it seems the list is working...

\styleinfos

1.1 [*] 1904.cc 1.1 [*]

06.05.2007

\stylemail


isn't it?




\stylepiece{35}

\stylemailtitle

it seems the list is working...

\styleinfos

cym

06.05.2007

\stylemail


yes it is

On Sun, 6 May 2007 1.1 [*] 1904.cc wrote:

> isn't it?
> \_\_\_\_\_\_\_\_\_\_\_\_\_\_\_\_\_\_\_\_\_\_\_\_\_\_\_\_\_\_\_\_\_\_\_\_\_\_\_\_\_\_\_\_\_\_\_
> aether mailing list
> aether at [nospam] 1904.cc
> \hyphenatedurl{http://lists.1904.cc/mailman/listinfo/aether}
>





\stylepiece{36}

\stylemailtitle

html seems to work good

\styleinfos

christiaan cruz

07.05.2007

\stylemail


The html works good so far. You N3krozofties are FAST!
The console window doesn't pop up to the correct 
size though; usually too small.
Is there anything else you need help with?
Any specific feedback, beta testing 
calls/submissions or dirty work?
Please let me know.
 
-christiaan
-------------- next part --------------
An HTML attachment was scrubbed...
URL: \hyphenatedurl{http://lists.1904.cc/pipermail/aether/attachments/20070507/5dd8fd02/attachment.html} 



\stylepiece{37}

\stylemailtitle

html seems to work good

\styleinfos

1.1 [*] 1904.cc 1.1 [*]

07.05.2007

\stylemail


>The html works good so far. You N3krozofties are FAST!
>The console window doesn't pop up to the correct
>size though; usually too small.

depends probably how your toolbars look like.
anyway, the intended setup is double screen:
the video frames in full screen, projected.
the control window on a control monitor.
so the resize is not really needed, just practical for testing on 1 
small screen.

>Is there anything else you need help with?
>Any specific feedback, beta testing
>calls/submissions or dirty work?

- FEEDBACK: think of what else should/could be implemented in this 
"html-patch".
the principle for now is: one main frame - 
\hyphenatedurl{http://www.1904.cc/~aether/mainframe.html}
which is divided in a number of subframes (for now six)
for each location, there are different versions of subframes with 
different refresh speed (for now three)
e.g.  \hyphenatedurl{http://www.1904.cc/~aether/frame\_brussels\_m.html} (m: medium 
speed, 1sec refresh)
I attach the zipped html files so you can test in your computer...

one function that needs to be implemented would be to change the 
frame color of one video in order to highlight it - just a different 
html page, or could this be done in a smart way, like a javascript 
link that changes style settings of target frame on mouseclick?

another function i would like to test is a "hi-speed" mode: a html 
page where the jpegs would be placed in different layers, only one of 
them visible at a time, and where the page would cycle through them 
at very high speed (its quite easy to do using dreamweavers timeline 
function). there would be no refresh in this mode since refresh needs 
more time.

- BETA TESTING: yes, please test this HTML refresh mode by uploading 
some pictures and seeing if they refresh instantly as they should.
using for the moment the following directories:
\hyphenatedurl{http://artslashmedia.net/~aether/z/0z.jpg}  (select:Sydney)
\hyphenatedurl{http://artslashmedia.net/~aether/g/1g.jpg}  (select:Brussels)
\hyphenatedurl{http://artslashmedia.net/~aether/e/2e.jpg}  (select:Geneva)

Also, we should test the different existing webcam-2-ftp softwares. 
Window users should give a try to these open-source softs:
\hyphenatedurl{http://lundie.ca/fwink/}
\hyphenatedurl{http://www.webcam2000.info/}

tell us if they work, and how the .pngs or .jpgs are going the be 
named - this will allow me the include the good filepath into the 
HTML page.

- CALLS/SUBMISSIONS
yes, you are all very welcome to look for opportunities to present 
this project elsewhere, write submissions, fill out forms etc. if 
someone of you is volunteering for this job, i will be glad to 
transmit all the text material we have.
The Virginia Beach proposal looks very interesting, the detailed 
specifications are here:
\hyphenatedurl{http://www.vbgov.com/file\_source/dept/cvd/Documents/OFCA-7-0002\%20Complete\%20RFP\%20Package.pdf}
Deadline for application is: May 18.

looking forward,
manuel
-------------- next part --------------
A non-text attachment was scrubbed...
Name: aether\_html\_code.zip
Type: application/octet-stream
Size: 119204 bytes
Desc: not available
Url : \hyphenatedurl{http://lists.1904.cc/pipermail/aether/attachments/20070507/f01081c5/attachment.obj} 


\stylepiece{38}

\stylemailtitle

control interface update

\styleinfos

1.1 [*] 1904.cc 1.1 [*]

08.05.2007

\stylemail


hi all,

a new sexy html controller is online.
check it out here:
\hyphenatedurl{http://www.1904.cc/~aether/control\_panel2.html}

it should be quite intuitive, the numbered arrows will make the images  
refresh at different speed (1=slow, 3=fast).
most of the frames are still void, waiting for your images to be uploaded.

to test realtime upload you can use the lastest version of the max  
patch (v.101), which you can find here:  
\hyphenatedurl{http://1904.cc/~aether/upload\_101.pat.zip}

or you can use the webcam2ftp software of your choice pointing to the  
following servers (from left to right and from top to bottom):

\hyphenatedurl{http://1904.cc/~aether/k/} (images named 0k.jpg - 9k.jpg)
\hyphenatedurl{http://1904.cc/~aether/l/} (images named 0l.jpg - 9l.jpg)
\hyphenatedurl{http://1904.cc/~aether/m/} (images named 0m.jpg - 9m.jpg)

\hyphenatedurl{http://10111.org/~aether/0/} (images named 0x.jpg - 9x.jpg)
\hyphenatedurl{http://10111.org/~aether/1/} (images named 0x.jpg - 9x.jpg)
\hyphenatedurl{http://10111.org/~aether/x/} (images named 0x.jpg - 9x.jpg)

if you use a webcam software that has a non-customisable naming  
scheme, tell me how the images are named and we will make a specific  
frame for this. btw we are still awaiting user reports about this two  
windows webcam softs:
\hyphenatedurl{http://lundie.ca/fwink/}
\hyphenatedurl{http://www.webcam2000.info/}

we could fix an appointment for this evening or some other day to do a  
sort of jam session...

ps: for all those who subscribed during the last 24 hours, check the  
list archives for the posts you missed :  
\hyphenatedurl{http://lists.1904.cc/pipermail/aether/}

best,
manuel






\stylepiece{39}

\stylemailtitle

Re : sending testing

\styleinfos

::audrey::

09.05.2007

\stylemail


i am receiving. images from the ocean(?) the refresh every 3seconds or so, around 4-5 different images i think.




\stylepiece{40}

\stylemailtitle

good test so

\styleinfos

Paula Vélez Bravo

09.05.2007

\stylemail


audrey:
cool.
alors, la question est ... je suis en train d'envoyer normalement  
autour de 33 images differents.
fluides d'une femme avec un bonnet blanc sur la figure dans la mer...

il faut encore ameliorer le systeme. l'ideal est d'avoir des images  
fluides dejà.
mais c'est très interesant quand même.

salut.

PAula




\stylepiece{41}

\stylemailtitle

fluxus

\styleinfos

Paula Vélez Bravo

10.05.2007

\stylemail


it works beter at speed 2.
anyway there are only, maybe, 5 frames... i don't know, maybe 9...  
and i was trying to send 33 frames. to put it more fluide. not so  
interrupted. because we can not talk about movement or video. it is  
diaphorama. then we have to concive something to play the material  
this way.
unfortunaly we can not play with the "persistence retinnienne" in any  
way if things work like this.

paula



\stylepiece{42}

\stylemailtitle

fluxus / live session proposal

\styleinfos

1.1 [*] 1904.cc 1.1 [*]

10.05.2007

\stylemail


hi, thanks for the comments,

indeed we could implement a version with a longer loop, like 33 frames.
the reason of having only 9 images is that there 
will be a quicker "live" refresh. with 33 images, 
you will cycle a lot through the same images 
before you see anything new.
but we have to test it.
also, for the persistance retinienne, i will try 
to build this hi-speed htmlpage.
but it seems also clear that at higher speeds, 
even the javascript-code isn't able to refresh 
the images correctly... still a trick has to be 
found

proposal for a jam session:
date: this friday, title: "electronic cafe international"
concept: broadcast from a café/ public location 
with internet access, using possibly a mobile 
computing device (of course exceptions are 
possible for locations where showing up publicly 
with a laptop would endanger your security..)
please send a mail to tell if you will 
participate, then i'll attribute the frames.

what's the best global performance time?
i propose this: 10 PM in europe = 1 PM in 
california = 3 PM in colombia = 6 AM in australia
(poor australia.. one hour later would be possible too..)

and btw a new patch is here: \hyphenatedurl{http://1904.cc/kode/upload\_103.pat.zip}
it works at a smaller size (the actual size at 
which the images are displayed on the web 
interface)
and there is a control of the speed of the video 
(can be useful to slow it down, to match the 
upload speed...)


best,
manuel




\stylepiece{43}

\stylemailtitle

Re : fluxus / live session proposal

\styleinfos

::audrey::

10.05.2007

\stylemail



ok. bypass the refresh and the re-direct.
attached code to tell browser to get image every x mmseconds.
in the code here:
/* the magic call that tells the browser (window)
   to call the changeimage function every 1000ms (1 sec) */
    window.setInterval(changeimage, 1000);
the url of the photos on the server change for each square, these have to be changed in the code also. for now i have: \hyphenatedurl{"http://www.1904.cc/~aether/frame\_k/0k.jpg"} (0-9)

it has to be tested to see how many images the array can take, the buffer size, etc. if you give me the correct url for the photos i can do this... in any case, for now attached code (html + javascript only), comments are very explicative, curtosy of a mate...

audrey



\stylepiece{44}

\stylemailtitle

question for preperation for... tomorrow 11/05/2007 GMT/UTC+2 o'clock

\styleinfos

oheun lee

10.05.2007

\stylemail


Hi, this is ion - oheun lee - around geneva.

Thx a lot for all the hard work of aether group. I've been withdrawn for
some time but I'd love to participate.

Would you kindly explain/suggest what sort of images / objects / etc to
prepare in advance, beside "firm" internet connection \& laptop with webcam?

Best regard,

oh eun



\stylepiece{45}

\stylequoteblock{

\styleinfos



\stylequote



[14:47] cym: the woman? and the clouds in the sky?
[14:47] ideacritik: cym : that's paula

}

\stylepiece{46}

\stylemailtitle

aether news

\styleinfos

1.1 [*] 1904.cc 1.1 [*]

16.05.2007

\stylemail


hi all,

busy week, in belgrade for a few days. but here 
are some thoughts, + some extracts of our recent 
chat chatter.


a) technical considerations

some explanations about the HTML interface:
open \hyphenatedurl{http://www.1904.cc/~aether/control\_panel.html}
click on "open mainframe"
try the different buttons:

mode 1,2,3 are cycling 10 images
(1: refresh every 4 seconds, 2: every second, 3: 5 frames per second)

in mode "33", 33 images are loaded and cycled at 
high speed (actually 8 images per second).
this speed couldn't be obtained if the page was 
refreshing each image everytime it's read. 
therefore, the images are refreshed at a much 
slower speed, 1 img per second.
so if the performer is transmitting, the loop 
will be changing very slowly over time.

[23:52] ideacritik: 33 is wild... but i get all the same images after a while.
[23:52] fe2cruz: it just takes a few seconds for the browser to catch up

in mode 1,2,3, each image is refreshed at the 
moment it is played. so this is closer to 
"realtime", but it also limits the speed at which 
the images can cycle.



b) general speed considerations:

[23:54] ideacritik: i think the speed change is a 
bit 'gadjetty' myself (though the 33 is certainly 
hilarious!)
[23:54] dspstv: 33 is far to much for what is 
being uploaded.. not enuf images to make a 
sentence
[23:54] dspstv: it loops to fast
[23:54] dspstv: imo
[23:55] ideacritik: 2 is ok.
[0:25] cym: the problem with speed 33 is that you must first have 33 images :-)
[0:25] cym: 33 images that sort of go together



c) GRID FORMAT: 6 or 9 frames

[0:27] fe2cruz: are we stuck with 6 for now? or 
can we jump back to 9 on the next jam test?
[0:28] ideacritik: it's just a question of adjusting the html page.
[0:28] ideacritik: not a big deal.
[0:28] cym: if there are 9 people that take part it should work?
[0:28] ideacritik: the 9 square grid was aesthetically pleasing.
[0:29] cym: i liked the 9 grid very much

Comment: 6 frames was really for testing only, 
with a one-screen computer it's much easier like 
that, when you need to share the screen between, 
the HTML frames, the chat, the upload software... 
9-frame setting should stay imo the standard for 
the actual performance.


d) the noble art of patching:

[0:17] fe2cruz: does anyone know how to write this patch for PD?
[0:18] boris\_: not me.
[0:18] cym: i would like to have a look at it
[0:19] fe2cruz: can someone on MAX post screen 
captures of the patch? would that help at all?

no time for that this week, but more explanations 
about the inner workings of the patch will 
follow. also, i just discovered through the max 
list (thanks to a guy called vade who did an 
interesting webcam project: 
\hyphenatedurl{http://abstrakt.vade.info/?p=80} ) some techniques 
that could improve the receiving part of the 
patch significantly and eventually allow it to 
download the images fast and without crashing...


e) concept + content

[0:23] cym: i would be very interested to try to 
create a play, a story for those slow refreshing 
cameras
[0:23] dspstv: exactly
[0:23] cym: i actually like the idea that there 
is a new image only every second or even slower
[0:23] dspstv: agree
[0:23] fe2cruz: so is 33 just too much? too 
gittery should we keep it simple and stick with 1 
\& 2 speeds?
[0:24] cym: with 9 cams in a grid it should be 
possible to create a nice effect, even when the 
changes are slow
[0:24] paulav: maybe
[0:25] paulav: il like the possibility of speeds, 
but if it doesnt work very good?

[0:51] boris\_: yes, images are important and as 
cym was saying it would be good to work on some 
good script that would be playable by 6 (or 
better: 9) remoters and image per image... maybe 
packets of images , not necessarly in oreder 1 to 
33 but also a reset all function would be very 
usefull

Comment: yes, we need to develop some narrative 
outline. Concept proposal: GHOST STORY (Ghost 
Stories from Medellin, from the Californian 
desert, from the Austrian woods, from the Swiss 
heights..... we would also really need some 
Japanese or Korean ghost involved .. think Kairo).
Also with our 9-frame split-screen setup, maybe 
some inpiration could come from such 
split-narration epics as the Canterbury Tales, 
the Manuscript of Saragossa....


f) some more possible applications:
again very short term, if someone of you would 
have time and motivation to put together material 
for an application, would be great.

best,
manuel




\stylepiece{47}

\stylemailtitle

remote fairytales - performance invitation

\styleinfos

bk

06.06.2007

\stylemail


Dear aethernauts

the next performance of the aether9 group will be on Thursday 14th of 
june, online and in brussels
at 7 PM in europe = 10 AM california = noon in colombia
in an audience packed gallery in the center of  Brussels (n3krozoft Ltd 
will occupy this space for 3 days (see the attached flyer). the 
performance and the exhibit is called
"remote fairytales, contes lontains, fabel op afstand"
(last expression is flemmish - brussels is bilingual french/flemish. 
flemish is very very close to dutch).

the idea for this performance:

1) on location in brussels, next to the screen, there will be 2 persons 
on stage:
- Deirdre Foster (she is a storyteller and actress from Geneva) - she 
will tell her version on the classical tale (from the Grimm brothers) 
"La jeune fille sans mains" (The girl without hands --> 
\hyphenatedurl{http://en.wikipedia.org/wiki/The\_Girl\_Without\_Hands).}
- Me, Boris, i will be conferencing the audience about the reunion of an 
important group of patchers (i'm almost over writing this 
conference-like text extrapolating from my aether9 experience!)
we will intermingle our texts and form a strange dialogue.
our idea is that Deirdre \& I will be quite confused about what is going 
on: i will begin my conference by adressing myself to the audience in a 
similar way and setting (chair, table, lamp) than academic conferences 
are usually given, with the intention of informing he audience. Deirdre, 
dressed in a ... fancy way,  will then "interrupt me" and start telling 
her tale, with the intention of enchanting her audience. I will react by 
answering to her, in my conference mode, but with the intention of 
understanding what Deirdre is doing here. Deirdre will continue her 
tale, with the intention of unveiling the question: what am i doing in a 
digital-media performance? i thought i was in a storyteller convention!
etc. and our dialogue will go on according to what happens between us 
two, the audience and aether9 screens, but we will always stick more or 
less to the stories we wrote.


2) remote aether9 part:
in parralel to the stage action, aether9 remote action will flow on the 
screen (maybe 2 screens).
we didn't develop any script yet. but we want to.
the script will integrate formal and narrative elements related to the 
stage action.
sometimes the stage action will be on standby for the screen action to 
take all the attention. in these moments, formal elements of aether9 
will prevail.
when the stage action will be running, the narrative elements of aether9 
wil prevail, providing a visual tale.
i guess that there will be an \stylerefmailchat{IRC}\stylerefroad{7}{49} channel for aether9 with somebody in 
brussels giving cues for actions.
our texts are in french and should be translated to some extent in 
english for everybody to grasp the stories...  well for Deirdre's tale, 
the basis is on the net and for mine i'll adress this later today.
we wish that all remote performers have a common "neutral" set , for 
instance something simple like : a table with a chair, a neutral wall, a 
picture on the wall, a book and paper + pen  or typemachine on the table.
this in order to avoid total randomness of the visual aspect and 
confusion for the viewers (online especially), in order to have some unity.
not all performers would be perorming at the same time, and it would be 
good to have a common neutral state in a way.

3) sound
- there will be some sound environnment in the brussel performance.
somebody will be in charge of the sound diffusion.
we are not thinking of stremaing the sound for the moment.
- we would like to invite you to \stylerefmailtech{upload}\stylerefroadrwdonly{14} some sound files on 
aether at artslashmedia.\stylerefmailtech{ftp} \stylerefmailtech{server}\stylerefroadfwdonly{48} in the >web >"sounds\_remote\_fairytales" 
folder
- please view D. Lynch "rabbit" project for interesting sound treatment 
(in our view...)
--> here are 2 videos:
\hyphenatedurl{http://www.bzzp.biz/temp/}
(1st episode: full first episode, 10min
fragment: 2 min extract)
- you could also record yourselves saying short sentences (cf. lynch again).


4) rehearsals
- on the 13th of june, one day before the performance, we should do a 
videocheck to see how everything is running in the gallery (at the same 
time scheduled for the next day's performance.)
- this friday (8th of june), we would like to do a jam session.
this session will adress technical problems (the aether9 system was  
working well on the "electronic café international" session of the 10th 
of may.
since manuel has been working more on the \stylerefmailpatch{Patch}\stylerefroad{670}{48} and we need to \stylerefmailscience{test}\stylerefroad{32}{49} the 
latest versions.
we will also write some script with formal elements to test how it goes.
for instance:
- the table setting described before
- make a cross between the 9 screens
- 2 persons making phone calls at the same time
etc.


----------------------------------

so,
who will participate in this landmark performance?
- to my knowledge, the following aethernauts said that in principle they 
are ready to take part:
Nathalie from paris, Audrey from holland, Paula from medellin, cym from 
walkendorf, Chris from yorba linda - california.
i guess that a group from geneva will do a part too.
- Dr. Gomez \& the league?
- Rhys from australia? (i heard that he would use puppets - that would 
be good - performance time in au. would be early early in the morning.)
- Johan from berlin?
- Milena from the brazil group?
- Laure form the Kingdom of belgium?

-- 
so we need to know asap:
who will be there for the jam session this friday
who wishes to take part in the "remote fairytales" perfomance on the 
14th of june (and, if possible, the rehearsal on the 13th of june).

all ideas, suggestion, disagreements etc. welcome and awaited!!!
i love this team!
looking forward,
Boris






\stylepiece{48}

\stylemailtitle

tommorow's jam session - details

\styleinfos

bk

07.06.2007

\stylemail


hello!

tommorow's (8.06.2007) jam session is at
9 PM in europe
Noon in califonia
10 AM in colombia
11 PM in moscow

new working patches!!!
new 9 frame viewer \stylerefmailtech{HTML}\stylerefroadrwdonly{47}-page!!!

---

a couple exercices for us:

-let's try to \stylerefmailtech{upload} monochromatic images only at times (whether with 
colored light if you have or filters to put in front of the camera or 
any other way we can find)
please prepare to upload images using \stylerefmailred{red}\stylerefroad{3}{65}, blue and yellow.
+ darkness (not total)

- anything that has to do with the "Girl without hands" tale is welcome.

- if your camera films a setting with a wall (and some picture on it), a 
table, a chair and sombody (you?) sitting behind the table, perfect! 
(would be nice to try this all at the same time).

---
\stylerefmailtech{server}\stylerefroadfwdonly{49} list. 08.06.2007

1 \textbar{} 2 \textbar{} 3
4 \textbar{} 5 \textbar{} 6
7 \textbar{} 8 \textbar{} 9

Frame 1 - \hyphenatedurl{http://1904.cc/~aether/1}
Frame 2 - \hyphenatedurl{http://1904.cc/~aether/2/}
Frame 3 - \hyphenatedurl{http://10111.org/~aether/0/}

Frame 4 - \hyphenatedurl{http://10111.org/~aether/1/}
Frame 5 - \hyphenatedurl{http://artslashmedia.net/~aether/5/}
Frame 6 - \hyphenatedurl{http://aether.front.ru/6/}

Frame 7 - \hyphenatedurl{http://aether.front.ru/7/}
Frame 8 - \hyphenatedurl{http://aether.smtp.ru/cym/}
Frame 9 - \hyphenatedurl{http://aether.smtp.ru/9/}

for \stylerefmailtech{ftp} uploaders: username and password on each server are the same as 
usual.
In each case the jpegs are now named 0x.jpg to 9x.jpg
(except for cym, who uses arbitrarily frame 8. your jpg is named pw.jpg)

the html interface will be available tomorrow, linked from 1904.cc/aether

participants should announce on the list which frame they want to use.

for maxers: use the latest \stylerefmailpatch{Patch}\stylerefroad{47}{49}: \hyphenatedurl{http://1904.cc/kode/upload\_107.zip} (22 k)

or, if the new javaobjects aren't working, use this simplified version 
(without visual feedback; you'll visualise through the html interface) - 
\hyphenatedurl{http://1904.cc/kode/upload\_109.zip} (18 k)

---

boris

ps: i will gladly use frame no. 7!




\stylepiece{49}

\stylemailtitle

last news and performance preparation

\styleinfos

bk

09.06.2007

\stylemail


hello aethernauts,

we jammed yesterday with
laure\_bruxell1, paula\_medellin, manu\_geneva, boris\_brussel2 and 
amirali\_teheran (\stylerefmailchat{chat}\stylerefroadfwdonly{57}\&images)
+
christiaan\_california, cym\_walkendorf, ideakritik\_rotterdam and the 
league\_california (chat only)
no script was established so it began \stylerefmailslow{slowly}\stylerefroad{670}{226} in a quite \stylerefmailthemes{random}\stylerefroad{623}{170} jam, then 
ideakritik brilliantly gave cues according to elements of the "girl 
without hands" tale. it lasted 3 hours or so.

--

technical conclusions:
- the viewing 9frame \stylerefmailtech{HTML}\stylerefroadrwdonly{48} page works great (i attached some screenshots, 
some slots are black because we weren't 9)
- for maxers: the \stylerefmailpatch{Patch}\stylerefroad{48}{56} \stylerefmailtech{upload}\_109 works perfectly
- for \stylerefmailtech{ftp}\stylerefroadfwdonly{65} uploaders: it works well also
this means that the transmission system for next thursday's performance 
is fully fonctional!

--

practical conclusions:

- we all agreed that it is extremely helpfull to be 2 persons for the 
performance, for example:
1 person in charge of following the \stylerefmailchat{IRC}\stylerefroadrwdonly{47} chat for the cues, for the 
computer manipulations and the camera manipulations
1 person in charge of the image (as actor, puppetist, drawer etc)

- we need a fine tuned script - this script should be finished on 
wednesday at the latest. (see my propositions below)
- each performer needs a clear method. everybody develops one naturally, 
but it's better to find it before the performance than during the 
performance... will you use a fix or moving camera?  etc etc.
we need law\&order: more the formal aspects are defined, the better it 
"feels" (knowing that anyway it will be quite hazardeous - i like this 
struggle).
it's possible at any time now to upload images and visualise it on 
 >\hyphenatedurl{http://1904.cc/aether/} >new \stylerefmailscience{test}\stylerefroad{47}{65} interface!!!

- perform simple \& small things
we may think that what we are uploading is not "rich" or interesting 
enough, but then we forget that there are 8 other images!

- react quicky \& act slowly
it's difficult to be all alert \& super fast to react to cues and in the 
same way remember not to rush the action

- obviously prepare a maximum of stuff before the beggining, in this way 
you can concentrate on less things during the performance.

maxers:
- use the 109 for upload and the htlm 9frame page to view (not the more 
complex upload\_107 patch, wich may crash)
- use the start/stop button often - do not constantly upload images. 
press the stop button to rearrange the setting and prepare for next actions.
pressing once "start" then immediatly press "stop" will upload just one 
image. this is great to create controlled animations (although it's hard 
to control the order of the images, but it's verry fun nevertheless).

--

proposals for the basis of the script:
- the script form in grid that we used for the first aether9 performance 
was great i think.

- it is now certain that we will use the "Girl Without Hands" tale. 
Deirdre will tell it during the performance in brussels (we don't know 
exaclty "how")

- think it would be efficient and elegant to make a clear distinction 
between when we upload "formal" elements (like "film an empty desk 
before an empy chair") and "narrative" elements like
("The devil sends a messenger to kill the queen and daugther!")

i guess some of us prefer to do a performance with emphasis on formal 
visuals, while others are more interested in emphasizing the 
video-storytelling - it's different types of freedom and constraint.  
both aspects are very important but cannot realistically be reunited by 
every single performer yet (because it would need so much preparation).
one way to do this would be to have a constant distinction on the screen 
between who is working "fomal" and who is working "narrative".
yesterday when everybody displayed an empty chair - the impression was 
striking! common action works, and they never fail to create an 
extremely deep impression.
but the formal elements should always be thought of as going with the 
tale in a way but not interpretating it litterally.
all this shall be scripted of course.

- the n3krozoft team can edit the script according to all the input on 
the mailing list and related discussions, but if somebody or a group 
wants to take the responsability to put this sript together, it would be 
a great relief for us (because we are also in the rush of preparing the 
exhibit: building fine zoetropes, editing dvd's, printing very large 
video stills and more...). Please anonce it asap if you want to do this.

--

lineup for thursday the 14th:

sure: cym, laure, paula, nathalie, christiaan
  maybe: geneva, amirali:teheran, ideakritik
no news: johan (berlin), alejo
--> so we aren't ready yet participation-wise (there won't be any live 
performance in the brussel gallery space itself.)

--

lookking forward,
boris




\stylepiece{50}

\stylemailtitle

pd patch

\styleinfos

cym net

12.06.2007

\stylemail


hello all,

a friend of mine wrote a patch for pd last night that should do the
same thing as the max/jitter patch (upload images from the live webcam
stream to the server via ftp, naming them img1 to img33 and when they
reach 33 start again with img1).

if anyone of you is working with debian linux pd i can send the patch.
in general he would be interested to share and test the patch with
anyone.

also, he would be interested to join the jam session tomorrow.
his name is Nova Viator (or Luka Prinčič) and you can find more
information about him at
\hyphenatedurl{http://bin.viator.si/?page=biography}
+ \hyphenatedurl{http://viator.si/}

attached is a png of the pd patch he sent me

greetings,

cym





\stylepiece{51}

\stylequoteblock{

\styleinfos

Luka Prinčič

\stylequote



i think emerging pdp should work quite well...

}

\useexternalfigure[F5OqB8mwbPMsKR9hzleolg==][../PAR-PERFO/2007-06-14\_BXL/img/MAAC\_live1.jpg]\hbox {\externalfigure[F5OqB8mwbPMsKR9hzleolg==][width=\getvariable{pageprops}{columnwidth},factor=max]}\blank[image]

\useexternalfigure[V8Q6K6Ggur8IBKNYjMgx4w==][../PAR-PERFO/2007-06-14\_BXL/img/MAAC\_enligne.jpg]\hbox {\externalfigure[V8Q6K6Ggur8IBKNYjMgx4w==][width=\getvariable{pageprops}{columnwidth},factor=max]}\blank[image]

\useexternalfigure[gsUYBpSaIzGmiHTxspzjmw==][../PAR-PERFO/2007-06-14\_BXL/img/perf7END.jpg]\hbox {\externalfigure[gsUYBpSaIzGmiHTxspzjmw==][width=\getvariable{pageprops}{columnwidth},factor=max]}\blank[image]

\useexternalfigure[0fwmpiO/dqMxQiEvmE4U0A==][../PAR-PERFO/2007-06-14\_BXL/img/MAAC\_live.jpg]\hbox {\externalfigure[0fwmpiO/dqMxQiEvmE4U0A==][width=\getvariable{pageprops}{columnwidth},factor=max]}\blank[image]

\stylepiece{56}

\styleinfos
Remote Fairytales
N3krozoft Ltd micro exhibition
Maison d’Art Actuel des Chartreux (MAAC), Brussels (Belgium)
Amirali
Ghasemi,
Boris
Kish,
Chloé
Cramer,
Christiaan
Cruz,
cym,
ideacritik,
Laure
De
Selys,
Manuel
Schmalstieg,
Paula
Vélez
\blackrule[color=black, width=65mm, height=0.5pt, depth=0mm]
\styleperfo
THE OPENING PROGRAM CONSISTED IN A LECTURE ABOUT THE NOBLE ART OF PATCHING, ALONGSIDE WITH THE GRIMM BROTHERS TALE “THE GIRL WITHOUT HANDS”, NARRATED BY STORYTELLER DEIRDRE FOSTER AND ACCOMPANIED VISUALLY BY THE ÆTHER GROUP, AN INTERNATIONAL REMOTE PERFORMANCE ENSEMBLE WHOSE MEMBERS PARTICIPATED FROM LOCATIONS IN CALIFORNIA, COLOMBIA, EUROPE AND IRAN.
\blackrule[color=black, width=65mm, height=0.5pt, depth=0mm]

\stylepiece{57}

\stylemailtitle

brussels performance report

\styleinfos

bk

16.06.2007

\stylemail


hello all,

we just opened the gallery this morning in brussels for the last day of 
the exhibit here.

i write now a little performance report

/////
1. Mother Nature
we where about ready (one hour before the performance)
 when mother nature invited herself
it started to rain and rain and rain
and all at once, water filtered through the roof
of the large performance hall
and poured in buckets on our command table
we ran to unplug the curent and move tables in the dark
to protect the computers and cameras and stuff
the sound was ear-crushing: torrential rain
in 5 minutes, there was 5 cm of water everywhere
and we where soaked... and sort of annoyed...
so we rushed to transfer everything from the big hall to the
smaller exhibit hall
while the lady-gallerist was calling the fire department
/////
2. Action
quite stessfull we managed to be not too late for the beginning
and the performance started in front of about 30 people (that where 
brave enough to come in spite of the exceptional storm)
some more walked in during the show
they where welcomed with champagne and other drinks
Deirdre told the tale in a baroque white dress colored by the projection 
rays
Chloe mixed discreet sounds \& music
Boris told his story sitting in front of the audience on a desk while 
trying to film his part of the tale.
Manuel managed the projected images and chated on the \stylerefmailchat{Skype}\stylerefroad{49}{100} conference
/////
3. Audience reaction
some people where quite confused, but most of them liked it to put it 
simply and shortly.
///////
4. Debriefing
it was hard to be fully concentrate under these climatic conditions - 
but it was exciting...
- as there was no actual stage anymore, Deirdre found it extremely 
difficult to properly perform her craft, storytelling. she also feels 
frustration that we didn't have enough time to prepare a more refined 
"mise en scene" between her and Boris.
- Manuel was trying to give cues according to Deirdre's words, and 
obviously it was going much too quick for the remote performers (\& for 
manuel also who had to solve some techicalities...) - it was a BIG 
mistake to do so. we apologize for that. a better thing would be for the 
performers to switch between acts every 3 minutes for instance - like 
that everything would be co-ordinated whitout questions...
it is a shame really that the performers weren't able to display their 
work in a calm manner with sufficient time to do so. this will not 
happen again.
- Boris is basically satisfied of the opportunity to read loud his 
speech (adapted for the circumstance...).

--

pictures \& movies will come soon on the aether website.

and check cristiaan site:  \hyphenatedurl{http://zurcnaaitsirhc.blogspot.com/}

--

à bientôt,
Boris






\useexternalfigure[EZbNpjzi2UDWP1UUFpapZA==][../PAR-PERFO/2007-07-02\_Rotterdam/img/20070702\_21h21m.png]\hbox {\externalfigure[EZbNpjzi2UDWP1UUFpapZA==][width=\getvariable{pageprops}{columnwidth},factor=max]}\blank[image]

\useexternalfigure[jBOERZP+vs7XXYQsIRDzxw==][../PAR-PERFO/2007-07-02\_Rotterdam/img/20070702\_21h04m.png]\hbox {\externalfigure[jBOERZP+vs7XXYQsIRDzxw==][width=\getvariable{pageprops}{columnwidth},factor=max]}\blank[image]

\useexternalfigure[Oq8PZM1tx06/EzeEa6G0Ug==][../PAR-PERFO/2007-07-02\_Rotterdam/img/20070702\_21h13m.png]\hbox {\externalfigure[Oq8PZM1tx06/EzeEa6G0Ug==][width=\getvariable{pageprops}{columnwidth},factor=max]}\blank[image]

\useexternalfigure[SZ3yS3dPO2HsqgUjCKmK5Q==][../PAR-PERFO/2007-07-02\_Rotterdam/img/20070702\_21h18m.png]\hbox {\externalfigure[SZ3yS3dPO2HsqgUjCKmK5Q==][width=\getvariable{pageprops}{columnwidth},factor=max]}\blank[image]

\useexternalfigure[8mFUPX53W9TYYF6eZEX92w==][../PAR-PERFO/2007-07-02\_Rotterdam/img/20070702\_21h14m.png]\hbox {\externalfigure[8mFUPX53W9TYYF6eZEX92w==][width=\getvariable{pageprops}{columnwidth},factor=max]}\blank[image]

\stylepiece{63}

\stylemailtitle

Rotterdam script

\styleinfos

1.1 [*] 1904.cc 1.1 [*]

02.07.2007

\stylemail


to the rotterdam performers,
here the basic version of the script.

9h00 opening (empty desks, chairs, etc)
9h01 character 1 appears
9h02 character 2 appears
9h03 character 3 appears
9h04 character 1: action (fills a glass of water)

9h05 character 2: action (...)
9h06 character 3: action (...)
9h07 character 2: talks (in direction to 1)
      character 1: answers (in direction to 2)
9h08 character 3: leaves the space
9h09 character 2+1 seem nervous
9h10 character 3: comes back

9h11 character 3: talks
      character 2: looks to character 3
      character 1: close up action
9h12 character 3+2: close up action
      character 1: looks at the others
9h13 character 3: close up action
9h14 character 1,2,3: stand up, saluts au public.
9h15 empty desk


for additions or changes, use the wiki:
\hyphenatedurl{http://1904.cc/timeline/tiki-index.php?page=020707+rotterdam+script}



\stylepiece{64}

\stylemailtitle

Re : today -- hotel new york

\styleinfos

::audrey::

02.07.2007

\stylemail


i apologise - i will be late for what i thought was a noon online-ness from the space.
i am waiting for the dvcam which was supposed to be handed to me at 11.
audrey




\stylepiece{65}

\styleinfos
“Conférence aux Antipodes”
“Inquiry in Location” curated by Mirjana Boba Stojadinovic for the Piet Zwart Institute
Hotel New York, Rotterdam (The Netherlands)
Boris
Kish,
Christiaan
Cruz,
ideacritik,
Laure
De
Selys,
Manuel
Schmalstieg
\blackrule[color=black, width=65mm, height=0.5pt, depth=0mm]
\styleperfo
“CONFéRENCE AUX ANTIPODES” WAS A MYSTERIOUS 15 MINUTE CONFERENCE CALL BETWEEN 3 REMOTE PERFORMERS. THIS PERFORMANCE OCCURRED IN THE CONTEXT OF THE "INQUIRY IN LOCATION" EVENT, CURATED BY MIRJANA BOBA STOJADINOVIC FOR THE PIET ZWART INSTITUTE, WHICH TOOK PLACE IN ONE OF THE FORMER CONFERENCE ROOMS OF HOTEL NEW YORK, IN ROTTERDAM. IT WAS ALSO THE FIRST AETHER9 PERFORMANCE TO IMPLEMENT A LIVE AUDIO MP3 STREAM (VIA THE GISS.TV STREAMING \stylerefperfotech{server}\stylerefroad{49}{92}).
\blackrule[color=black, width=65mm, height=0.5pt, depth=0mm]

\useexternalfigure[RTXHGmXM7Hb8/dZ8nB+DGQ==][../PAR-PERFO/2007-07-02\_Rotterdam/img/20070702\_21h15m10s.png]\hbox {\externalfigure[RTXHGmXM7Hb8/dZ8nB+DGQ==][width=\getvariable{pageprops}{columnwidth},factor=max]}\blank[image]

\chatbox{\stylechatpiece{67} }{\stylechatinfo{ideacritik 21:15}}{\stylechat{ok we go good.}}

\chatbox{\stylechatpiece{68} }{\stylechatinfo{ideacritik 21:15}}{\stylechat{imon 11.30}}

\useexternalfigure[jmsL6mNKsVHOsnPqBBpu0g==][../PAR-PERFO/2007-07-02\_Rotterdam/img/20070702\_21h15m20s.png]\hbox {\externalfigure[jmsL6mNKsVHOsnPqBBpu0g==][width=\getvariable{pageprops}{columnwidth},factor=max]}\blank[image]

\useexternalfigure[GpnrqigsFPXNvwklnquz9g==][../PAR-PERFO/2007-07-02\_Rotterdam/img/20070702\_21h16m10s.png]\hbox {\externalfigure[GpnrqigsFPXNvwklnquz9g==][width=\getvariable{pageprops}{columnwidth},factor=max]}\blank[image]

\useexternalfigure[JDKBMB8S4z93notr07aTyw==][../PAR-PERFO/2007-07-02\_Rotterdam/img/20070702\_21h16m20s.png]\hbox {\externalfigure[JDKBMB8S4z93notr07aTyw==][width=\getvariable{pageprops}{columnwidth},factor=max]}\blank[image]

\useexternalfigure[Dvmu64LDi3s+vB8xJ5qHLw==][../PAR-PERFO/2007-07-02\_Rotterdam/img/20070702\_21h17m10s.png]\hbox {\externalfigure[Dvmu64LDi3s+vB8xJ5qHLw==][width=\getvariable{pageprops}{columnwidth},factor=max]}\blank[image]

\chatbox{\stylechatpiece{73} }{\stylechatinfo{ideacritik 21:17}}{\stylechat{ok last one in 10sec!!!!!!!!!!!!!!!!}}

\chatbox{\stylechatpiece{74} }{\stylechatinfo{ideacritik 21:18}}{\stylechat{we finish in 30 sec!!!!!!!!!!!!!!!!!!!!!!!!!!!}}

\chatbox{\stylechatpiece{75} }{\stylechatinfo{ideacritik 21:18}}{\stylechat{cool.}}

\chatbox{\stylechatpiece{76} }{\stylechatinfo{ideacritik 21:18}}{\stylechat{finish ! cool}}

\chatbox{\stylechatpiece{77} }{\stylechatinfo{ideacritik 21:18}}{\stylechat{merci}}

\chatbox{\stylechatpiece{78} }{\stylechatinfo{thom_edison 21:19}}{\stylechat{grrrrr!}}

\chatbox{\stylechatpiece{79} }{\stylechatinfo{ideacritik 21:19}}{\stylechat{cool.}}

\chatbox{\stylechatpiece{80} }{\stylechatinfo{ideacritik 21:19}}{\stylechat{ok j'arrete music}}

\stylepiece{81}

\stylequoteblock{

\styleinfos

Christiaan Cruz

\stylequote

 


none of my friends watch or listen to my streamed performances.They get confused, they don't like the delays and they just can't understand what is happening.

}

\stylepiece{82}

\stylequoteblock{

\styleinfos

Christiaan Cruz

\stylequote

 

My images look fuzzy because i don't use the patch and have a lot of settings done incorrectly so things get resampled and messy.

}

\stylepiece{83}

\stylemailtitle

Rotterdam performance

\styleinfos

Boba

03.07.2007

\stylemail


Dear all,

It was a true vibe to watch you all perform!!! I'm sending you million
virtual kisses from somewhere on the way to Belgrade, and I'll be watching
you on the 7th :)*

I hope to keep in touch,
Boba





\stylepiece{84}

\stylemailtitle

final TWO for 7/7/7

\styleinfos

cym net

06.07.2007

\stylemail


i will be making tree-color pasta (fusilli tricolori) with tomato
sauce, with some vegetables added to the tomato sauce

in fact i am eating the pasta right now already and didn't start
making the sauce yet, but i will try to do that still later tonight to
have all the material ready for tomorrow

i need to first download the movies of the first part and charge the
batteries before i can continue the second part of cooking. it takes
quite some planning to cook a meal online

are there any other recipes? does anyone else already know what s/he will cook?

by the way, the three colored pasta i used is green, orange and white.
i might stick to those colors also with the vegetables and add the \stylerefmailred{red}\stylerefroad{65}{93}
tomatoes as a contrast...

any suggestions how to add numbers and formulas to the recipe?

happy cooking,

cym






\stylepiece{85}

\stylemailtitle

question - logistics

\styleinfos

cym net

06.07.2007

\stylemail


it seems we are 8 now? is that okay? can we make it a 7+1 performance?
7+1 squares? for me it would be okay







\stylepiece{86}

\stylequoteblock{

\styleinfos

Nathalie Fougeras

\stylequote



Yep, i find auto-regulation by rythm of speedonumber interesting too—Nathalie

}

\useexternalfigure[uh7TxyJI3H9A2Oh83kD+ww==][../PAR-PERFO/2007-07-07\_777/img/snapshot070707-37.jpg]\hbox {\externalfigure[uh7TxyJI3H9A2Oh83kD+ww==][width=\getvariable{pageprops}{columnwidth},factor=max]}\blank[image]

\useexternalfigure[w4XU9geug7NHkzhF3fHzFg==][../PAR-PERFO/2007-07-07\_777/img/snapshot070707-19.jpg]\hbox {\externalfigure[w4XU9geug7NHkzhF3fHzFg==][width=\getvariable{pageprops}{columnwidth},factor=max]}\blank[image]

\useexternalfigure[X3FJviCedqc3qPqdOi7fNw==][../PAR-PERFO/2007-07-07\_777/img/snapshot070707-18.jpg]\hbox {\externalfigure[X3FJviCedqc3qPqdOi7fNw==][width=\getvariable{pageprops}{columnwidth},factor=max]}\blank[image]

\useexternalfigure[XwHi+uYoKiRpqQOEBvtdAg==][../PAR-PERFO/2007-07-07\_777/img/hours\_070707.png]\hbox {\externalfigure[XwHi+uYoKiRpqQOEBvtdAg==][width=\getvariable{pageprops}{columnwidth},factor=max]}\blank[image]

\useexternalfigure[W5Ehr/JqqNrdSpVokArQKA==][../PAR-PERFO/2007-07-07\_777/img/777\_performance.png]\hbox {\externalfigure[W5Ehr/JqqNrdSpVokArQKA==][width=\getvariable{pageprops}{columnwidth},factor=max]}\blank[image]

\stylepiece{92}

\stylemailtitle

070707 news

\styleinfos

bk

07.07.2007

\stylemail


\hyphenatedurl{http://1904.cc/aether/live/index.\stylerefmailtech{HTML}\stylerefroad{65}{113}\#}

Hello,
so we will be 7!

following the set-up shaped as number 7,

XXX
-XX
XX-

i propose:

NORWAY      LA1              LA2
--------              BRUSSEL    GENEVA
SLOVENIA    AUSTRIA    --------

or in other words:

league front
n3krozoft front
slovenia front

in this disposition for instance:

Dr. Gomez   Dr. Shleidan   Dr. Madmann

-------------     BXL n3kro     GVA n3kro

Luka             Cym                ------------

///////////////

if the 3 league members can coordinate something, great!

here in brussels we are of course goig to respect the first and last 7 
minutes segments (presentation \& eating) but haven't decided yet if we 
are going to explicitly respect the 5 middle 7 minute segments. surprise 
it will be.

i will carefully record the performance with a camera filming a computer 
screen.

see you all later,
Boris


\stylepiece{93}

\styleinfos
07.07.07
Celebration of the remarkable date (07.07.07)
exclusively online
Boris
Kish,
Chloé
Cramer,
cym,
Laure
De
Selys,
Luka
Prinčič,
Manuel
Schmalstieg,
Nathalie
Fougeras,
The
League
of
Imaginary
Scientists
\blackrule[color=black, width=65mm, height=0.5pt, depth=0mm]
\styleperfo
A SHARED MEAL BETWEEN PERFORMERS LOCATED ON TWO DIFFERENT CONTINENTS. CELEBRATION OF THE REMARKABLE DATE
\blackrule[color=black, width=65mm, height=0.5pt, depth=0mm]

\stylepiece{94}

\stylemailtitle

AetherOne 02 - pd patch update

\styleinfos

Luka Prinčič / Nova Viator

07.07.2007

\stylemail


On Sat, 7 Jul 2007 19:21:56 +0200
1.1 [*] 1904.cc wrote:

> thanks luka,
> 
> btw, i tested the patch on osx, substituting pdp\_v4l with pdp\_iee1394
> it was exporting the pngs successfully, but i ran into a problem with shell - actually the object seems to be buggy on OSX (even the help-patch keeps crashing).
> 
> that problem was reported on the PD list back in 2005 - cf. \hyphenatedurl{http://lists.puredata.info/pipermail/pd-list/2005-11/032615.html}
> 
> this seems to be  a kind of solution, however i dont really understand what it means:
> 
> "i had these kind of problem too with shell on macosx+linux. the issue was that whenever a process did exit whit a nonzero status a zombie pd process was left and shell claimed that the process was still running. the solution was to make sure that the return status of the executed script is zero and to [delay] the bang for a short while. as far as i remember i used a value below 100."
> 
> it's certainly not easy to troubleshoot another system than the one you use, but maybe you have an idea what i could try?




there is a workaround.

i'm attaching a modified patch. (02b)
no different names, no shell object. just one png
generated in /tmp/aether9.png

but uploading is done in separate terminal window with a 
long commandline like this:

ID=0; while (true); do if ((`expr \$ID '>' 8`)); then ID=0; else ID=`expr \$ID '+' 1`; fi; cat /tmp/aether9.png \textbar{} ncftpput -u aether -p PASSWORD -c HOSTNAME \$ID\textbackslash{}x.png;  sleep 1; done

it's also in the patch, but you have to type it, so rather copypaste from here.

also change PASSWORD and HOSTNAME.

in the patch change [pdp\_noise] to whatever else you want.
there is also a [receive] object - [r aetherone] so you can send video signals from some other opened patch with a send object like this: [s aetherone]

sleep 1 means wait one second (+ time it needs to upload). you can adjust this.



hope it helps



\useexternalfigure[U+13HRfFS5voiHpn/W2rNA==][../PAR-PERFO/2007-07-07\_777/img/20070707\_21h38m16s\_img10.png]\hbox {\externalfigure[U+13HRfFS5voiHpn/W2rNA==][width=\getvariable{pageprops}{columnwidth},factor=max]}\blank[image]

\useexternalfigure[DmdiqSyHrW+ZIU08d9oXkA==][../PAR-PERFO/2007-07-07\_777/img/20070707\_21h46m56s\_img18.png]\hbox {\externalfigure[DmdiqSyHrW+ZIU08d9oXkA==][width=\getvariable{pageprops}{columnwidth},factor=max]}\blank[image]

\stylepiece{97}

\stylequoteblock{

\styleinfos

Luka Prinčič

\stylequote



unfortunately there are new dependencies:
PiDiP, playlist object and zexy.

}

\stylepiece{98}

\stylequoteblock{

\styleinfos

Boris Kish

\stylequote



anyways most of us aethernauts are kids of the 68 generation, for most 
of us the cold war and the habit of clear enemies is non-existent.

}

\stylepiece{99}

\stylequoteblock{

\styleinfos

Judy Nylon

\stylequote



With regard to finances, the credit system is really well thought out - can it be automated, so that no one has to administer the system?! (none of us wants to be the aether accountant...)

}

\stylepiece{100}

\stylemailtitle

next performance SETUP

\styleinfos

bk

16.07.2007

\stylemail


hello

here is the setup situation for the central balkan mountains performance 
in Bulgaria on the 19th of this month
confirmed:
 > Cym + Luka (wd8\_austria)
 > Lucy/The League (norway)
 > Boris (belgium)
 > Manu + Alejo (Dortmund Germany)
 > Nathalie - did you solve your camera problem? (Paris France)

unconfirmed:
 >>> Chris?
 >>> Laure?
 >>> Judy?
 >>> Amirali?
 >>> ???

--

have a look at the pic gallery of the place, it's quite something: 
\hyphenatedurl{http://netuser.bg/2007/en/?gal=1\&page\_id=500}

-- 

i had a look to one of the chats alejo pointed to: \hyphenatedurl{http://home.gabbly.com/}
super easy to embed... for bulgaria it would be no problem. (the 
simplest way is to type gabbly.com/ in front of the url of the page you 
visit, and all others visitors doing the same will be reunited in a 
chatroom embedded on the visited url...)
issues are:
it eats quite some place on a web page.
the design is... what it is - but more horrifying is the ad embed in the 
top corner of the \stylerefmailchat{chat}\stylerefroad{57}{114} (i did not understand if we can replace it by 
what we wish).
you can see this chat at  \hyphenatedurl{http://gabbly.com/colboard.com/index.php} - 
when i tried it, 8 people where chatting at supersonic speed and it 
worked perfectly.

--

Coco Islands Time Zone: UTC+6½

--

Boris





\stylepiece{101}

\stylequoteblock{

\styleinfos

Luka Prinčič

\stylequote



aicks.. crashed my firefox (iceweasel;)..

}

\useexternalfigure[Y5CX3bKsmKz9Pk531DTfvQ==][../PAR-PERFO/2007-07-19\_Bulgaria/img/18h24a.jpg]\hbox {\externalfigure[Y5CX3bKsmKz9Pk531DTfvQ==][width=\getvariable{pageprops}{columnwidth},factor=max]}\blank[image]

\useexternalfigure[kUXMK/aqIL1neHIvZw+oig==][../PAR-PERFO/2007-07-19\_Bulgaria/img/18h35e.jpg]\hbox {\externalfigure[kUXMK/aqIL1neHIvZw+oig==][width=\getvariable{pageprops}{columnwidth},factor=max]}\blank[image]

\useexternalfigure[63Q582BJ2KRjRC1Lvn8U0w==][../PAR-PERFO/2007-07-19\_Bulgaria/img/19h04.jpg]\hbox {\externalfigure[63Q582BJ2KRjRC1Lvn8U0w==][width=\getvariable{pageprops}{columnwidth},factor=max]}\blank[image]

\useexternalfigure[OElZcxiF72O6OzNsoKTUIg==][../PAR-PERFO/2007-07-19\_Bulgaria/img/007.jpg]\hbox {\externalfigure[OElZcxiF72O6OzNsoKTUIg==][width=\getvariable{pageprops}{columnwidth},factor=max]}\blank[image]

\useexternalfigure[YZsczI7nQnebFqzjIjf4ig==][../PAR-PERFO/2007-07-19\_Bulgaria/img/006.jpg]\hbox {\externalfigure[YZsczI7nQnebFqzjIjf4ig==][width=\getvariable{pageprops}{columnwidth},factor=max]}\blank[image]

\useexternalfigure[CiEPNa7iOm8DJTBWyNmEuQ==][../PAR-PERFO/2007-07-19\_Bulgaria/img/19h15.jpg]\hbox {\externalfigure[CiEPNa7iOm8DJTBWyNmEuQ==][width=\getvariable{pageprops}{columnwidth},factor=max]}\blank[image]

\useexternalfigure[JPnVyy6nM+581qgMKYSbwQ==][../PAR-PERFO/2007-07-19\_Bulgaria/img/005.jpg]\hbox {\externalfigure[JPnVyy6nM+581qgMKYSbwQ==][width=\getvariable{pageprops}{columnwidth},factor=max]}\blank[image]

\useexternalfigure[5Wrpj55TEDy4FjDt5WhgMA==][../PAR-PERFO/2007-07-19\_Bulgaria/img/018.jpg]\hbox {\externalfigure[5Wrpj55TEDy4FjDt5WhgMA==][width=\getvariable{pageprops}{columnwidth},factor=max]}\blank[image]

\useexternalfigure[t7ynR+ILnc3QDhD3JWwmTw==][../PAR-PERFO/2007-07-19\_Bulgaria/img/008.jpg]\hbox {\externalfigure[t7ynR+ILnc3QDhD3JWwmTw==][width=\getvariable{pageprops}{columnwidth},factor=max]}\blank[image]

\useexternalfigure[yhWtxNYQOmVvhgLNF8/wlA==][../PAR-PERFO/2007-07-19\_Bulgaria/img/18h35d.jpg]\hbox {\externalfigure[yhWtxNYQOmVvhgLNF8/wlA==][width=\getvariable{pageprops}{columnwidth},factor=max]}\blank[image]

\useexternalfigure[mY1tqxqhLXj1cmJbpKmRYg==][../PAR-PERFO/2007-07-19\_Bulgaria/img/009.jpg]\hbox {\externalfigure[mY1tqxqhLXj1cmJbpKmRYg==][width=\getvariable{pageprops}{columnwidth},factor=max]}\blank[image]

\stylepiece{113}

\stylemailtitle

bulgaria - interface

\styleinfos

1.1 [*] 1904.cc 1.1 [*]

19.07.2007

\stylemail


hi all,
here is a possible setup for the Hotel Pleven performance (changes 
are possible of course):

laure(gva).JPG \textbar{} cym+luka.PNG \textbar{} audrey(rdam).JPG

chris.JPG      \textbar{} boris.JPG    \textbar{} league.JPG

objects        \textbar{} objects      \textbar{} objects

the audio player is now a bit more hidden, see the new interface 
from: \hyphenatedurl{http://1904.cc/aether/live/}

since we had some trouble (uncontrollable delays) with the .ru 
servers, i propose to drop them for now.

also, after some longer testing, it seems that png has a tendency to 
be quite heavier than jpg, specially with very colourful/contrasted 
material. so, since the \stylerefmailpatch{Max}\stylerefroadfwdonly{114} \stylerefmailpatch{Patch} has now a contrast function, i 
suggest we switch back to jpg (not for the \stylerefmailpatch{pd}\stylerefroadrwdonly{56} users though).

this are now the corresponding \stylerefmailtech{ftp}\stylerefroadfwdonly{114} directories:

frame 1 - \hyphenatedurl{ftp://1904.cc/1/}  format: jpg
frame 2 - \hyphenatedurl{ftp://1904.cc/2/}  format: png
frame 3 - \hyphenatedurl{ftp://imaginaryscience.org/3/}  format: jpg

frame 4 - \hyphenatedurl{ftp://imaginaryscience.org/4/}  format: jpg
frame 5 - \hyphenatedurl{ftp://10111.org/web/5/}  format: jpg
frame 6 - \hyphenatedurl{ftp://10111.org/web/6/}  format: jpg

frame 7 - \hyphenatedurl{ftp://artslashmedia.net/web/7/} - OBJECTS
frame 8 - \hyphenatedurl{ftp://artslashmedia.net/web/8/} - OBJECTS
frame 9 - \hyphenatedurl{ftp://artslashmedia.net/web/9/} - OBJECTS

note: user is "aether" for all servers, except imaginaryscience.org 
where the user is aether

note2: for frame 6 (league), there is only 1 self-refreshing png 
file, called 1.png

note3: i didn't include myself among the performers - will have a 
busy day tomorrow and preparing the technical/\stylerefmailtech{HTML}\stylerefroadrwdonly{92} stuff + preparing 
the performance/finding the objects will be hardly possible.

if any of you needs special settings (filenames etc), tell me as soon 
as you can.

best,
manuel



\stylepiece{114}

\stylemailtitle

(no subject)

\styleinfos

laure deselys

19.07.2007

\stylemail


hello,


leaving brussels pc-naked made it quite hard for me to
organise tomorrows performance and was also very busy.
so couldnt take part so much to the wiki and script
elaboration.
 i am and will be for tomorrow  located in france
(high-savoy), i still have a  few technical problems
but it should be solved by tomorrow midday. cross
finger.

the place where i here grab the net, is in a kind of
basement
and for me it actualy seems easier to use a window
that is ground high, then a  door. i dont know, if
considering this i should be placed somewhere else in
the nine screen strucure.


maybe i can find a solution with a door but cant
confirm it right now

for the four objects i was thinking of:

1. \stylerefmailred{red}\stylerefroad{93}{231} kimono
2. hunting gun
3. hat(s)
4. mountain boots

boris, i dont expect you to have these important
things at home, but you can also use the color or hand
gesture to make contact with them

or if its not considered usefulll enough for bulgaria
( desk type)

1. pencil sharpner
2. hair brush
3. embrella
4. goggles

and so on . theres such a mess of objects in this
place 
i must make a tactico-narrativ choice

i will send the pictures for the slide show tomorrow
for midday

more in a few hours
gotta catch up some sleep



laure.



Luka Prinčič / Nova Viator nova at viator.si 
Thu Jul 19 17:44:05 CEST 2007 
Previous message: [aether] 115 or 114 
Next message: [aether] wd8 -> live 
Messages sorted by: [ date ] [ thread ] [ subject ] [ author ] 

On Thu, 19 Jul 2007 17:03:28 +0200
1.1 [*] 1904.cc wrote:
> i set everything back to JPG except the frame used by cym/viator 
> (which is using the 0x.png - 9x.png naming scheme, hopefully it will 
> work with \stylerefmailpatch{pd}\stylerefroad{113}{119})
> 
> cym, i remind your \stylerefmailtech{ftp}\stylerefroadfwdonly{136} access:
> frame 2 - \hyphenatedurl{ftp://1904.cc/2/}

yes.

i just drove to wd8 at cym's, managed to download the linux driver for this nifty cym's hercules \stylerefmailtech{webcam}\stylerefroadrwdonly{113}, and it works!

we're uploading.

what's the \stylerefmailchat{chat}\stylerefroad{100}{120} channel this time then?

ll.

attaching the in-process screenshot *hope this list doesn't eat attaches for dinner*, wink wink...


-- 
<aav> coffee on an empty stomach is pretty nasy
<knghtbrd> aav: time to run to the vending machine for cheetos
<aav> cheetos? :)


\stylepiece{115}

\styleinfos

NetUser4 International Working Camp
Pleven (Bulgaria)
Boris
Kish,
Christiaan
Cruz,
cym,
ideacritik,
Judy
Nylon,
Luka
Prinčič,
Manuel
Schmalstieg,
Nathalie
Fougeras,
The
League
of
Imaginary
Scientists
\blackrule[color=black, width=65mm, height=0.5pt, depth=0mm]
\styleperfo
CONCEPTUAL PERFORMANCE REFLECTING UPON EVERYDAY DISPOSABLE/CONSUMER GOODS AND 'LUXURY' OBJECTS.
\blackrule[color=black, width=65mm, height=0.5pt, depth=0mm]

\stylepiece{116}

\stylemailtitle

Thoughts, and a riddle

\styleinfos

Lucy H G

19.07.2007

\stylemail


Quick thoughts!

I think this project in Bulgaria is a bit of a riddle - the script and
performance - figure out as you go - search/consume/share.  The first aether
project with its highly developed storyline, pseudo-\stylerefmailscience{science}\stylerefroad{65}{149} and mythological
themes still holds the most for me...  It was quite fun to do, and I think
it is because of the intensive scripting - even though this made it more
challenging.

I like the subject/object- consumerism theme for this unnamed last piece,
but I think timing is essential.  Coordination of objects moved from frame
to frame, etc.  Perhaps we are all in our own space time bubbles - we are
still quite disconnected during these performances, and I think a live audio
conference between us would help us to synchronize, as well as get to know
one another.

Instead of a new theme for each new project - perhaps we could return to
this study of an object, of coordination, of passing items through the frame
and communicating more "physically" with one other?  Boris had called for us
to simplify, I think...  With very little setup or concept - just play
catch, as it were.

Lucy/League




\stylepiece{117}

\stylemailtitle

Re : Thoughts, and a riddle

\styleinfos

cym net

20.07.2007

\stylemail




personally, i like to see aether 9 as a band that is playing live
concerts. (or a theatre group that is doing theatre shows, but for me
it is easier to compare it to a band)
a band has a set of songs that they perform live. they practise the
songs until they know them and can play them on the stage. they have a
set of material and can change the set/order a little for each
performance, and also create moments in the performance that leave
space for improvisation. but still the basic structure is very clear
for each band member. and at the same time it doesn't get boring to
perform the same songs/compositions (or same theatre play) again and
again.




\stylepiece{118}

\stylemailtitle

Re : Thoughts, and a riddle

\styleinfos

Judy Nylon

20.07.2007

\stylemail



\stylepiece{119}

\stylemailtitle

Re : Thoughts, and a riddle

\styleinfos

Luka Prinčič / Nova Viator

21.07.2007

\stylemail

understanding the feeling of people involved in 'meat space'...
anyway, i feel there is certain beauty (if i may use such a term here)
in the efforts and connections that are happening here, and during the
performances. 

i'm also learning a lot about group performance and how things are
bringing people closer together. the people who are performing, and the
notions of distance and global and speed and time and place and space.
i wish i had more theoretical tools to explain more precisely what i
mean. but poetics can be also quite telling.

i will/can work on a \stylerefmailpatch{pd}\stylerefroadrwdonly{114} \stylerefmailpatch{Patch}\stylerefroadfwdonly{323} and linux-enabled approaches to aether9.


\stylepiece{120}

\stylemailtitle

Re : Re : Thoughts, and a riddle

\styleinfos

::audrey::

24.07.2007

\stylemail


in response to 'getting to know each other' - as i also see this as a crucial element in a collaborative performance. especially since we cannot for the most part meet each other physically, small tidbits of knowledge become the extent of what we know about each other. these tiny tidbits are therefore also the only thing we have to go on in so far as expectation or imagination of how the other will act during a performance. in our excitement over 'the patches', have we perhaps overlooked that we must not only play, but play together? the person who 'assisted' me last performance was amazed at how little attention we payed to each others performances, 6 little autisms neatly assembled on a page. perhaps getting to know each other, even if only through these unsatisfactory mediums, is a start towards an aether9 embodied performance as opposed to several individuals operating in the same screen though otherwise seemingly detached.

i was lucky enough to meet cym last week in austria as well as to attend the workshop in geneva where i of course met manu/aether9, alejo and others who are more or less inactive for the moment. i am based in rotterdam. i've just finished my MA in media design. my thesis is about the interaction/affect between media and our mourning process, and how we perceive death. as in many of my endeavors, i interviewed many people about this very idea and incorporated their accounts into my final project/thesis. i am explaining this only because my somewhat anthropological methodology outlines one of my main interests in this group;  how do remote performers work together, communicate, make decisions, disagree, in short, form a group. the limitations of our \stylerefmailchat{communication}\stylerefroad{114}{134} media form a particular set of 'constraints' which inherently shape our performance together. perhaps my use of the word constraints is not just because it is usually derogatory. i see these
 constraints as borders from which to work with, in the same light as our 'low-tech one frame per second' refresh speed. i do not wish the frames to load quicker or to attain the perfect resolution. i prefer to take on this dimension and exploit it, in all its 'dramaturgical' glory. in the same way, i think we should exploit our 'thwarted' methods of communication. we often throw together a script at the last moment and more or less blindly perform all together without knowing what each other is doing. could we take this on as some sort of methodology? could this be made explicit as a performance tool? in this vein, i can imagine that we use simple 'unification' props such as text (which we have yet to integrate successfully but in which i have confidence). these props could be constant during each performance, and this aspect would be practiced, but the rest could be 'off the cuff' or improvised as usual. as cym noted, using text, written on a piece of
 paper as we did in the first performance, was really nice. this is an idea which i feel is a beautiful metaphor for our communication together if it is used to communicate between frames during a performance. i imagine inmates passing each other notes between jailcells, all being watched on simultaneously by the jailguard on nine tiny screens hooked up to those surveillance cameras. we use the technology but we still attempt to communicate with paper and pen... this could be a constant which we can rehearse... for example.

i hope this hasn't been too much of a digression to describe my feeling/desire to incorporate our communication process implicitly in our performances as well as to keep some things constant in order to refine our practice, focus our experiments.

audrey/ideacritik








\stylepiece{121}

\stylemailtitle

reminder: finances

\styleinfos

1.1 [*] 1904.cc 1.1 [*]

27.09.2007

\stylemail

so, Audrey, Laure, Chris, Boris,  Lucy, Nathalie, Cym, Judy, Paula, Luka, Alejo

tell me:
- your preference between Paypal or Moneybookers.com
- the email adress to which i should send the payment.

if you didn't use yet one of those systems, this will be the occasion 
to check how it really works.
for the record, for transfers of less than 50 EUR, paypal is more 
economic than moneybookers. above, moneybookers is cheaper (the 
highest transfer fee is 2,30 EUR).

best,
m.



\useexternalfigure[XRdy82iVP6XWeM3ZxoNgdA==][../PAR-PERFO/2007-10-04\_SputnikDay/img/Picture\_27.png]\hbox {\externalfigure[XRdy82iVP6XWeM3ZxoNgdA==][width=\getvariable{pageprops}{columnwidth},factor=max]}\blank[image]


\page
\setuplayout[nomargins]
\placefigure[here,force]{none}{\externalfigure[/tmp/OSP\_AETHER9\_\_Picture\_22FULL\_0.png][width=\paperwidth,height=\paperheight]}
\placefigure[here,force]{none}{\externalfigure[/tmp/OSP\_AETHER9\_\_Picture\_22FULL\_1.png][width=\paperwidth,height=\paperheight]}
\placefigure[here,force]{none}{\externalfigure[/tmp/OSP\_AETHER9\_\_Picture\_22FULL\_2.png][width=\paperwidth,height=\paperheight]}
\placefigure[here,force]{none}{\externalfigure[/tmp/OSP\_AETHER9\_\_Picture\_22FULL\_3.png][width=\paperwidth,height=\paperheight]}
\page
\setuplayout[reset]

\useexternalfigure[Jvjf+OFJsyjJCi1vWxSy6w==][../PAR-PERFO/2007-10-04\_SputnikDay/img/Picture\_23.png]\hbox {\externalfigure[Jvjf+OFJsyjJCi1vWxSy6w==][width=\getvariable{pageprops}{columnwidth},factor=max]}\blank[image]

\useexternalfigure[kxeKIiCI8wjKaQ7+NWYCgA==][../PAR-PERFO/2007-10-04\_SputnikDay/img/Picture\_26.png]\hbox {\externalfigure[kxeKIiCI8wjKaQ7+NWYCgA==][width=\getvariable{pageprops}{columnwidth},factor=max]}\blank[image]

\useexternalfigure[o6Kk4AX0kCLwJ4OBVLflCQ==][../PAR-PERFO/2007-10-04\_SputnikDay/img/Picture\_1.png]\hbox {\externalfigure[o6Kk4AX0kCLwJ4OBVLflCQ==][width=\getvariable{pageprops}{columnwidth},factor=max]}\blank[image]

\useexternalfigure[NjpJLwT1OcGZ+9TMUF6Nsw==][../PAR-PERFO/2007-10-04\_SputnikDay/img/Picture\_21.png]\hbox {\externalfigure[NjpJLwT1OcGZ+9TMUF6Nsw==][width=\getvariable{pageprops}{columnwidth},factor=max]}\blank[image]

\useexternalfigure[VhAwlB2ptm06dXGQgiw/bg==][../PAR-PERFO/2007-10-04\_SputnikDay/img/Picture\_19.png]\hbox {\externalfigure[VhAwlB2ptm06dXGQgiw/bg==][width=\getvariable{pageprops}{columnwidth},factor=max]}\blank[image]

\stylepiece{129}

\styleinfos
“untitled sound transmission”
International Sputnik Day (curated by Francis Hunger)
exclusively online
Manuel
Schmalstieg
\blackrule[color=black, width=65mm, height=0.5pt, depth=0mm]
\styleperfo
MINIMALISTIC PERFORMANCE CELEBRATING THE 50TH ANNIVERSARY OF THE SPUTNIK SATELLITE LAUNCH.
\blackrule[color=black, width=65mm, height=0.5pt, depth=0mm]

\stylepiece{130}

\stylemailtitle

Re : more archives online

\styleinfos

::audrey::

06.10.2007

\stylemail


we are trying to find someone to do the music-audio ... the idea i mentionned about streaming audio from there, to participants, and back into the room... no one confirmed thus far. do we have a backup audio streamer available - i know boris and laure are in transit so not available...

i`m off to a chalet for a couple of days, more upon return.

so who?s in???

audrey



\stylepiece{131}

\stylemailtitle

Re : oct. 26th & Judy update thru mid Nov

\styleinfos

Judy Nylon

07.10.2007

\stylemail

However with regard to stills or camera footage
.I am right now only a few blocks from Tesla’s studio on Bleeker Street NYC. Some of the architectural detail photographed here might be useful. I am also not short of either Pigeons or Pearls
..if there is any script directive, I can snap and send.

The last Aether note is: we have had issues with scripts. I applied and am still in the running for a year of mentoring at the  Public Theater (Joseph Papp’s). They are only choosing 10 of  650 applications so I am not holding my breath, but I am in a smaller group of those seeking to develop theater for new applications.

I am still uncomfortable with the tech I have and a public facing arts practice is hard to turn around quickly, but I am inspired with the possibilities we have and just wanted to give you some glimpse of the scheduling and the insane juggling involved to run it on a dollar and a dream.     JUDY    



\stylepiece{132}

\stylemailtitle

Re : aether Digest, Vol 6, Issue 4

\styleinfos

::audrey::

08.10.2007

\stylemail

À : aether at [nospam] 1904.cc
Envoyé le : dimanche 7 octobre 2007, 15 h 24 min 40 s
Objet : Re: [aether] aether Digest, Vol 6, Issue 4


I'm in audio, box, whatever you need 
streaming remotely that is
just let me know

-christiaan



\stylepiece{133}

\stylemailtitle

Tr. : aether

\styleinfos

1.1 [*] 1904.cc 1.1 [*]

11.10.2007

\stylemail


i imagine a very sober cover design, little graphics, lots of text 
and a title such as:

Aether9 Communication Systems
Research \& Development Group
Proceedings Volume 1, Issue 1

please all english-native-speaking: does this sound correct? or would 
you have other proposals (for the title)?

best,
manu



\stylepiece{134}

\stylemailtitle

26 october reconfirmation

\styleinfos

Paula Vélez Bravo

11.10.2007

\stylemail


so, the script and everything will be at the same link i supposed.

will connect today to \stylerefmailchat{Skype}\stylerefroad{120}{175} to see if we can try things up.

other thing, the sound streeming... i would like to try. it will be  
using giss tv?

ok.
see you soon on the net.

paula



\stylepiece{135}

\stylemailtitle

rehearsal times

\styleinfos

nicola unger

19.10.2007

\stylemail

wednesday 24 morning 11.00-14.00

thursday to be confirmed and friday gigtime as well
audrey will figure out techsetup today, not sure how
far we get.

cu soon!
nicola



\stylepiece{136}

\stylemailtitle

RE : last thougths this evening

\styleinfos

fougeras nathalie

22.10.2007

\stylemail

sorry for today.... ok i bought my \stylerefmailtech{webcam}\stylerefroad{114}{144} so tomorrow morning it's possible for me to participate to the rehearsal
see you soon
cheers
Natali



\stylepiece{137}

\stylemailtitle

Re : Re : RE : last thougths this evening

\styleinfos

::audrey::

23.10.2007

\stylemail


26.10.2007
14:30: general rehearsal (all *must* be present)
15:15: second/last general rehearsal (all *must* be present)

*things to schedule:
sound check : paula tu es disponible quand? et le vendredi comment tot p-e etre online avec le streaming du son.
je vais uploader de l'audio de trains holandais que j'aimerais incorporer au soundscape, quand sera tu online pour qu'on en discute?

a\textbar{}idea



\useexternalfigure[pN0n0IYPSodDh5Nd9/eHDQ==][../PAR-PERFO/2007-10-26\_WORM/img/20071024\_14h43m.png]\hbox {\externalfigure[pN0n0IYPSodDh5Nd9/eHDQ==][width=\getvariable{pageprops}{columnwidth},factor=max]}\blank[image]

\useexternalfigure[8W+tsjZLunkr/iloIlmKGQ==][../PAR-PERFO/2007-10-26\_WORM/img/20071024\_14h17m10s\_worm.png]\hbox {\externalfigure[8W+tsjZLunkr/iloIlmKGQ==][width=\getvariable{pageprops}{columnwidth},factor=max]}\blank[image]

\useexternalfigure[PKLyUv3Ca6ECFw6p0Pomiw==][../PAR-PERFO/2007-10-26\_WORM/img/20071024\_14h17m13s\_worm.png]\hbox {\externalfigure[PKLyUv3Ca6ECFw6p0Pomiw==][width=\getvariable{pageprops}{columnwidth},factor=max]}\blank[image]

\useexternalfigure[bq8EwzvAHBcV8rsx5lh7uw==][../PAR-PERFO/2007-10-26\_WORM/img/20071026\_14h00m.png]\hbox {\externalfigure[bq8EwzvAHBcV8rsx5lh7uw==][width=\getvariable{pageprops}{columnwidth},factor=max]}\blank[image]

\useexternalfigure[+FbRh53fYsC4doL+SBhCEg==][../PAR-PERFO/2007-10-26\_WORM/img/20071026\_14h12m.png]\hbox {\externalfigure[+FbRh53fYsC4doL+SBhCEg==][width=\getvariable{pageprops}{columnwidth},factor=max]}\blank[image]

\stylepiece{143}

\styleinfos
Stationary Æmotion
Introducing
WORM, Rotterdam (The Netherlands)
ideacritik,
Manuel
Schmalstieg,
Nathalie
Fougeras,
Nicola
Unger,
Paula
Vélez
\blackrule[color=black, width=65mm, height=0.5pt, depth=0mm]
\styleperfo
PROTAGONIST FLEES HER LIFE AND EMBARKS ON A TRAIN ADVENTURE, ESCAPING THE LAW AND HER LOVER.
\blackrule[color=black, width=65mm, height=0.5pt, depth=0mm]

\stylepiece{144}

\stylemailtitle

aether9 level2

\styleinfos

cym net

26.10.2007

\stylemail

i just spent the whole week in ljubljana teaching 18 students
\stylerefmailtech{HTML}\stylerefroad{136}{148}/webdesign (from 10am till 21pm every day..), so maybe i am
focused a bit too much on such details, but anyway.. the interface
looks really nice in firefox and would be good if it looks the same in
explorer...

anyway, i am right now watching, waiting for the performance to start.
some of my students also said they will watch tonight, so it would be
good to put some message or something online, that performance is
starting 30 mins later. I am not sure if it is enough just that the
number of seconds how long to wait keeps on changing. would be better
to put a line on the bottom that says
* performance will start 30 mins later - 22:30 CET *

okay, i will keep on watching and waiting and take some photos

cym

\stylepiece{145}

\stylemailtitle

aether9 level2

\styleinfos

1.1 [*] 1904.cc 1.1 [*]

27.10.2007

\stylemail

oops, indeed this looks not as intended.. and it has been so since 
the first aether performance!
can it be that i never opened it in explorer since 6 months ? or is 
this dependent on the version of explorer? anyway, we'll add this 
little missing tag...




\stylepiece{146}

\stylemailtitle

tremor participation form

\styleinfos

Paula Vélez Bravo

30.10.2007

\stylemail

\textbackslash{}                          R E C E P T I O  
N                               \textbackslash{}
\textbackslash{}----------------------------------\textbar{}------------------------------------ 
---\textbackslash{}
\textbar{}                      \textbar{}                \textbar{}          
\textbar{}                       \textbar{}
\textbar{}     video frame      \textbar{}                V         \textbar{}      video  
frame      \textbar{}
\textbar{}     in place 1       \textbar{}                          \textbar{}      in place  
3       \textbar{}
\textbar{}                      \textbar{}       video frame         
\textbar{}                       \textbar{}
\textbar{}                      \textbar{}     public bathroom in    
\textbar{}                       \textbar{}
\textbar{}                      \textbar{}     someother country     
\textbar{}                       \textbar{}
\textbar{}                      \textbar{}         place 2           
\textbar{}                       \textbar{}
\textbar{}\_\_\_\_\_\_\_\_\_\_\_\_\_\_\_\_\_\_\_\_\_\_\textbar{}\_\_\_\_\_\_\_\_\_\_\_\_\_\_\_\_\_\_\_\_\_\_\_\_\_\_\textbar{} 
\_\_\_\_\_\_\_\_\_\_\_\_\_\_\_\_\_\_\_\_\_\_\_\textbar{}
\textbar{}                      \textbar{}                           
\textbar{}                       \textbar{}
\textbar{}                      \textbar{}    video body and         
\textbar{}                       \textbar{}
\textbar{}    video frame       \textbar{}  faces fragments from    \textbar{}     video  
frame       \textbar{}
\textbar{}    in place 4        \textbar{}  pornochat images live   \textbar{}      in place  
5       \textbar{}
\textbar{}                      \textbar{}                           
\textbar{}                       \textbar{}
\textbar{}                      \textbar{}                           
\textbar{}                       \textbar{}
\textbar{}                      \textbar{}                           
\textbar{}                       \textbar{}
\textbar{}\_\_\_\_\_\_\_\_\_\_\_\_\_\_\_\_\_\_\_\_\_\_\textbar{}\_\_\_\_\_\_\_\_\_\_\_\_\_\_\_\_\_\_\_\_\_\_\_\_\_\_\textbar{} 
\_\_\_\_\_\_\_\_\_\_\_\_\_\_\_\_\_\_\_\_\_\_\_\textbar{}
\textbar{}                      \textbar{}                           
\textbar{}                       \textbar{}
\textbar{}                      \textbar{}                           
\textbar{}                       \textbar{}
\textbar{}    video frame       \textbar{}      video frame         \textbar{}     video  
frame       \textbar{}
\textbar{}     in place 6       \textbar{}      in place 7          \textbar{}    in place  
8         \textbar{}
\textbar{}                      \textbar{}                           
\textbar{}                       \textbar{}
\textbar{}                      \textbar{}                           
\textbar{}                       \textbar{}
\textbar{}\_\_\_\_\_\_\_\_\_\_\_\_\_\_\_\_\_\_\_\_\_\_\textbar{}\_\_\_\_\_\_\_\_\_\_\_\_\_\_\_\_\_\_\_\_\_\_\_\_\_\_\textbar{} 
\_\_\_\_\_\_\_\_\_\_\_\_\_\_\_\_\_\_\_\_\_\_\_\textbar{}

I also thougth about sound and performance in place... and i thougt  
that it could be interesting to put 9 or 8 channels in separate  
speakers. to send sound from all aethernauts places picked from  
people talking dirty in all languages.

well... i will continue tomorrow... waiting for new ideas.



\stylepiece{147}

\stylemailtitle

tremor participation confirmed

\styleinfos

Paula Vélez Bravo

01.11.2007

\stylemail

street,
sur un mur de la rue. outdoors at night, like christian said last  
night : an aether midnight drive-in.
if it is complicated i could do it maybe, i'm just asking ... in a  
cool place call: "pato feo films"

a friend of mine is working in this space... he is the one who  
invites us to participate in this festival

christiaan said yesterday:
its a fun space, looks like some cool sound and video stuff
i like how everything is on the floor. so you have to sit japanese style

there are some chairs and a sofa, but mostly you have to sit on the  
floor. so i'm trying to do it there... i will see. i need people to  
help me. maybe this friend of mine that manage that place.


PAula




\stylepiece{148}

\stylemailtitle

bogotá tempete

\styleinfos

Paula Vélez Bravo

04.11.2007

\stylemail


AETHER9:

à bogotá..... puis, pas facil faire un reperage, a part de prendre  
quelques images... mais il s'est passé un truc incroyable.
Une tempete de verglas, neige, pluie...

je suis venu pour le festival rock al parque. ajourd'hui je regraette  
de ne pas avoir allée, car les photos et films que j'aurai pu prendre  
j'en reve...

voici c'est qui c'est passe aujourd'hui, regardez quelques photos du  
journal.

encore vous dirai que ce n'est pas qu'une tempete de neige. mais en  
colombie ne neige pas!

seulement à 5.000 metres d'altitude. et voici, on est à bogotá à  
2600mts... rarement il y a des phienomenes comme ça.

\hyphenatedurl{http://www.eltiempo.com/multimedia/galerias/granizada/GALERIAFOTOS-} 
WEB-PLANTILLA\_GALERIAFOTOS-3801694.\stylerefmailtech{HTML}\stylerefroad{144}{155}

\hyphenatedurl{http://www.eltiempo.com/multimedia/galerias/granizadalectores/} 
GALERIAFOTOS-WEB-PLANTILLA\_GALERIAFOTOS-3801410.html


paula





\stylepiece{149}

\stylemailtitle

urgent tremor RDV2

\styleinfos

Paula Vélez Bravo

08.11.2007

\stylemail


nicola and audrey:
sorry about your internet connection.
i don't know how are we going to di now.
i have been waiting aethernautes since more than an hour... to fixe  
things, do a rehersal... mmm i don't know, i start to be worried  
about it.
don't know exacly how are we going to do.
so... i have to go to get some videos fprm the street to use in the  
performance.

i need participants ready to do some \stylerefmailscience{test}\stylerefroad{116}{324} tomorrow night (night in  
european time) it could be 18h eurpoean time.

we have to think about a plan B. maybe a quick new script and see if  
other AETHERnautes want to participate.
I know the hour it is not easy for you all. 1am in saturday morning  
10 november... for me it will be 9 november 19h.

so, tomorrow i wish all participants will be connected the sooner you  
can...
i will try to be connected since 10am  so 16h european time.

ok.
paula


El 8/11/2007, a las 14:31, nicola unger escribió:

> hi all, audrey and me are still in a situation without
> internet at home...
> so better not count on us, it is really stressfull to
> get online and timing is shit...!
> sorry...
>



\chatbox{\stylechatpiece{150} }{\stylechatinfo{boris 23:58}}{\stylechat{question : dans les 3 cellules du bas, il y aura les lignes comme pour la dernière perfo???}}

\chatbox{\stylechatpiece{151} }{\stylechatinfo{aether9:  23:58}}{\stylechat{mais, si vous pouvez enregistrer ce qui se passe sur place, simplement une camera qui filme le tout, toi et l'écran, et qui prend le son de la salle......}}

\chatbox{\stylechatpiece{152} }{\stylechatinfo{aether9:  23:58}}{\stylechat{paula avait l'idée de mettre des "images de fenetres de train", filmées à bogota}}

\chatbox{\stylechatpiece{153} }{\stylechatinfo{aether9:  23:59}}{\stylechat{et aussi, j'avais pensé que si jamais on a du retard, on pourrait animer ces frames...}}

\chatbox{\stylechatpiece{154} }{\stylechatinfo{aether9:  23:59}}{\stylechat{avec des imageries ayant trait à des trains, attente, salles d'attente, trains.......}}

\chatbox{\stylechatpiece{155} }{\stylechatinfo{paula vélez 23:59}}{\stylechat{1.preparer les panneaux ou textes d'afichage. 2. preparer vetements,a ccesoires, lieux train simulacre, (fnetr du train pourrai etre un ecran tele cache derriere simulation de fenetre avec \stylerefchatlogtech{loop}\stylerefroad{148}{179} du payssage)}}

\useexternalfigure[Gxc3i1fL4iTh1d1vIE+22w==][../PAR-PERFO/2007-11-09\_Bogota/img/2480677108\_2be946fd0e\_o.jpg]\hbox {\externalfigure[Gxc3i1fL4iTh1d1vIE+22w==][width=\getvariable{pageprops}{columnwidth},factor=max]}\blank[image]

\useexternalfigure[2Ji4DW2AF1X91/0eGb0rAw==][../PAR-PERFO/2007-11-09\_Bogota/img/BogotaTremor.png]\hbox {\externalfigure[2Ji4DW2AF1X91/0eGb0rAw==][width=\getvariable{pageprops}{columnwidth},factor=max]}\blank[image]

\chatbox{\stylechatpiece{158} }{\stylechatinfo{paula vélez 00:00}}{\stylechat{3. l'ecran de proyection sur la rue. une tissu a acheter.}}

\chatbox{\stylechatpiece{159} }{\stylechatinfo{paula vélez 00:00}}{\stylechat{4. video beam}}

\chatbox{\stylechatpiece{160} }{\stylechatinfo{paula vélez 00:00}}{\stylechat{5. camera pour enregistrer performance}}

\chatbox{\stylechatpiece{161} }{\stylechatinfo{paula vélez 00:01}}{\stylechat{6. musicien pret a travailler, parch installe sur son ordi, lui aprendre a l'utiliser}}

\chatbox{\stylechatpiece{162} }{\stylechatinfo{aether9:  00:01}}{\stylechat{ouf!}}

\chatbox{\stylechatpiece{163} }{\stylechatinfo{aether9:  00:01}}{\stylechat{6: comme on disait, pas obligé qu'il streame, il peut faire comme il veut...}}

\useexternalfigure[yTqQOmZrPf78CsY+0nHHMA==][../PAR-PERFO/2007-11-09\_Bogota/img/20071109\_09h45m21s\_rehearsal.png]\hbox {\externalfigure[yTqQOmZrPf78CsY+0nHHMA==][width=\getvariable{pageprops}{columnwidth},factor=max]}\blank[image]

\useexternalfigure[M3fKZutxu7a1kx1TgRoyxg==][../PAR-PERFO/2007-11-09\_Bogota/img/20071109\_10h34m25s\_rehearsal.png]\hbox {\externalfigure[M3fKZutxu7a1kx1TgRoyxg==][width=\getvariable{pageprops}{columnwidth},factor=max]}\blank[image]

\useexternalfigure[eYvXEqsuxi5PpcPs3EZKSQ==][../PAR-PERFO/2007-11-09\_Bogota/img/20071109\_18h44m12s\_before.png]\hbox {\externalfigure[eYvXEqsuxi5PpcPs3EZKSQ==][width=\getvariable{pageprops}{columnwidth},factor=max]}\blank[image]

\stylepiece{167}

\styleinfos
Stationary Æmotion
Tremor 4 <Live.Doc> festival
multi-locational throughout Bogotá (Colombia)
Christiaan
Cruz,
Laure
De
Selys,
Manuel
Schmalstieg,
Paula
Vélez,
Juliana
Restrepo
\blackrule[color=black, width=65mm, height=0.5pt, depth=0mm]
\styleperfo
PROTAGONIST FLEES HER LIFE AND EMBARKS ON A TRAIN ADVENTURE, ESCAPING THE LAW AND HER LOVER.
\blackrule[color=black, width=65mm, height=0.5pt, depth=0mm]

\useexternalfigure[yzqLlHj8GcMkpIs1pH4z/w==][../PAR-PERFO/2007-11-09\_Bogota/img/20071109\_19h18m41s\_after.png]\hbox {\externalfigure[yzqLlHj8GcMkpIs1pH4z/w==][width=\getvariable{pageprops}{columnwidth},factor=max]}\blank[image]

\useexternalfigure[aBJmtCjQHNWbObIP6gkbBw==][../PAR-PERFO/2007-11-09\_Bogota/img/20071109\_22h34m54s\_countdown.png]\hbox {\externalfigure[aBJmtCjQHNWbObIP6gkbBw==][width=\getvariable{pageprops}{columnwidth},factor=max]}\blank[image]

\stylepiece{170}

\stylemailtitle

Re : videocapture did works

\styleinfos

::audrey::

11.11.2007

\stylemail


----- Message initial ----
De : Paula Vélez Bravo ciruela at [nospam] une.net.co

>did somebody put the video already in the net? i would like to see it with sound incorporated.

hi all,
i am just now going through all the aether mails. as you know i am offline. in addition to that, the dv tape of the WORM performance has rather embarassingly ended up in someone else's camera which is now in turkey and etc so i won't even get my paws on the thing for another whole week. for this disorganisation i apologise as i find the immediacy of watching the performance from the 'outside' very important for critique. once i get the tape i'll be on the case asap.

a\textbar{}idea 

ps some feedback i received :
-it would have been interesting if the actions performed/mimiked by the actors have a connection to the medium, its limitations, politics, etc. rather than just stretching for example... 
-the storyline was not very clear
-it was not clear whether the brussels/geneva performances were live or recorded (which made the tension very interesting - or the intrigue).
-the cues which i gave to nicola (by whispering into a microphone) were for some very nice, bringing my role in the performance more clear and also adding to the intrigue of not knowing if the cues go to nicola or also to the other actors (who is guiding who?). for others much for the same reason a bit confusing, why those cues? (i.e. front/back/left/enter/leave/apple/...) they seemed too \stylerefmailthemes{random}\stylerefroad{49}{178}.

in general the feedback was positive and the critical points were brought up because they saw potential in the project.

that's all for now. still reading the last 20 aether mails... :S






\stylepiece{171}

\stylemailtitle

sunday ghost [TITLE suggestion]

\styleinfos

Judy Nylon

09.12.2007

\stylemail

Please consider for the Aether adaptation of "Ghost Trio" could be called "Ghost Given". It sounds like 'forgiven' and suggests the intangibility of what is received and the uncertainly of who is actually doing the giving. 

Judy


\useexternalfigure[bT12KCLhEJfIikVMzcemyQ==][../PAR-PERFO/2007-12-15\_GhostTrio/img/20071210\_23h21\_10\_SafariScreenSnapz002.jpg]\hbox {\externalfigure[bT12KCLhEJfIikVMzcemyQ==][width=\getvariable{pageprops}{columnwidth},factor=max]}\blank[image]

\useexternalfigure[xlb3Le3Tjud6RyrwMt/dkQ==][../PAR-PERFO/2007-12-15\_GhostTrio/img/20071210\_21h58m00s\_xpcrash.jpg]\hbox {\externalfigure[xlb3Le3Tjud6RyrwMt/dkQ==][width=\getvariable{pageprops}{columnwidth},factor=max]}\blank[image]

\useexternalfigure[RF44YvLfP6Qr/wkUfilRig==][../PAR-PERFO/2007-12-15\_GhostTrio/img/20071210\_21h59m00s.png]\hbox {\externalfigure[RF44YvLfP6Qr/wkUfilRig==][width=\getvariable{pageprops}{columnwidth},factor=max]}\blank[image]

\stylepiece{175}

\stylemailtitle

a marvelous fantasio rescripture von the SPOOKY TRIO de beket

\styleinfos

bk

11.12.2007

\stylemail


 > music:
- i didn't produced anything, contradictory to what i announced 
yesterday on \stylerefmailchat{Skype}\stylerefroad{134}{311}. questions:
- do we want to use Beethoven’s Piano Trio?
If yes, do we want to respect more or less the indications from Beckett 
OR do we want to freely play with this trio, with mixed loops or samples 
for example? (i understand that the uploaded music is only a fraction of 
the second movement of the trio. all the music indicated by Beckett 
commes from this second movement. i do not have the full piece - i 
uploaded, for the sake of the example, the part that goes with "Act1, 
l.31" - i also uploaded a couple nice loops).
- There's very little (faint) music in the original indications 
(repertoried in the wikipedia page).
- If we decide to use musical elements, is there sound appart from this 
music, for example do we want to stream some sounds of aint winds? Or 
"pastel noise"?
- Judy, how do you generally feel this music part? What are your ideas?



\stylepiece{176}

\stylemailtitle

RIGHTS

\styleinfos

Judy Nylon

12.12.2007

\stylemail

I called the agency here in NYC about the rights. They have never done anything for live and on-line for anything they represent and have not had any request for this TV play while she has worked there. The agent I spoke with said she would like a write up of what we are doing emailed to her     kate \{a+\}  gbagency.com      so that she might pass it along to Edward Beckett (owns the estate) who lives in England....to request permission and establish a fee for performance rights.

I too suggest we wait. There is not time before Saturday. JUDY


\stylepiece{177}

\stylemailtitle

Beckett estate & performance rights alarm

\styleinfos

Judy Nylon

13.12.2007

\stylemail


At some point we must consider the ways around this because we will not receive this man's blessing at any price. However, if everything remains web-based in its production and presentation, he will have nowhere to drop his lawsuits \& 'cease and desist' orders. We will have to change the name of the play/never call it an adaptation and perhaps morph it into something else......in my opinion......possibly retaining Beckett himself as the character (f)

Judy--


\stylepiece{178}

\stylemailtitle

today's reheasal & re-write of announcement

\styleinfos

Judy Nylon

14.12.2007

\stylemail

We’re presenting a live multi-based web link up on Saturday December 15th. Loosely adapted from a theater piece written in 1975, taped in ‘76 and televised on BBC2 in 1977. We are mixing it with a bit of dub on classic piano and samples.  It is a parallel universe played out in the same slice of time as the first wave of British punk. The language will be English. The development of live storytelling without a single fixed location is on going. 

My fellow aethernauts and I would be delighted if you would join us on-line at our first \stylerefmailthemes{chance}\stylerefroad{170}{255} to bring this piece from TV to a wider audience. It doesn’t matter what you’re wearing and you won’t have to worry about how you’re getting home.

Members of Aether9 will be performing from different locations:  Medellin, Yorba Linda, NYC , Brussels, Geneva, and Paris. The audience, those in the same room, will be joined at ‘Videomedja 2007’ at the Museum of Vojvodina, Novi Sad, Serbia, by two Aether9 artists from Austria and Slovinia who will host the Q \& A which follows the performance.

There are no complicated programs or hardware involved. It is easier than getting out of the house. I hope you will look through our site at  www.1904.cc where the collaborative process involved is quite transparent, our project laid out, and participants named. Here we will make the links “to set up your computer” so easy that you will practically fall through them to the new venue.

You will need to go through one link to find out what the time will be where you are, when it is Saturday at 11:15 PM in Serbia. The second link opens the audience viewing window and a third link leads to instructions to stream sound.



\chatbox{\stylechatpiece{179} }{\stylechatinfo{laure.dit 23:58}}{\stylechat{you know paula its maybe easier that with the corridor and the kid images you just prefilm them and \stylerefchatlogtech{upload}\stylerefroad{155}{180} when needed, so you dont have to get messed up with the camera. what do u think ?}}

\chatbox{\stylechatpiece{180} }{\stylechatinfo{laure.dit 23:58}}{\stylechat{you know paula its maybe easier that with the corridor and the kid images you just prefilm them and \stylerefchatlogtech{upload}\stylerefroad{179}{253} when needed, so you dont have to get messed up with the camera. what do u think ?}}

\useexternalfigure[eIXT5gaR8UBr9V1cC732hw==][../PAR-PERFO/2007-12-15\_GhostTrio/img/Ghost Trio 2.png]\hbox {\externalfigure[eIXT5gaR8UBr9V1cC732hw==][width=\getvariable{pageprops}{columnwidth},factor=max]}\blank[image]

\useexternalfigure[zDguz9fYFwqtoFlEqa4hCA==][../PAR-PERFO/2007-12-15\_GhostTrio/img/Ghost Trio.png]\hbox {\externalfigure[zDguz9fYFwqtoFlEqa4hCA==][width=\getvariable{pageprops}{columnwidth},factor=max]}\blank[image]

\chatbox{\stylechatpiece{183} }{\stylechatinfo{laure.dit 00:00}}{\stylechat{this corridor is nice though}}

\chatbox{\stylechatpiece{184} }{\stylechatinfo{laure.dit 00:00}}{\stylechat{this corridor is nice though}}

\stylepiece{185}

\styleinfos
Ghost Study
Videomedeja Festival
Museum of Vojvodina, Novi Sad (Serbia)
Boris
Kish,
Chloé
Cramer,
Christiaan
Cruz,
cym,
Judy
Nylon,
Laure
De
Selys,
Manuel
Schmalstieg,
Nathalie
Fougeras,
Paula
Vélez,
Richard
Milan,
Luka
Prinčič,
Juan
Guillermo
Caicedo,
Pedro
Hermelin,
Jonathan
Goldstein
\blackrule[color=black, width=65mm, height=0.5pt, depth=0mm]
\styleperfo
ADAPTATION OF SAMUEL BECKETT’S 1976 TV PLAY “GHOST TRIO”.
\blackrule[color=black, width=65mm, height=0.5pt, depth=0mm]

\chatbox{\stylechatpiece{186} }{\stylechatinfo{nathalie 00:00}}{\stylechat{o now we have 3 greys windows}}

\chatbox{\stylechatpiece{187} }{\stylechatinfo{nathalie 00:00}}{\stylechat{o now we have 3 greys windows}}

\stylepiece{188}

\stylemailtitle

novi sad 5pm and 23:15

\styleinfos

cym net

15.12.2007

\stylemail

so i think also aether 9 will be in the three rooms at the same time,
but i will check everything tomorrow.

there is wireless connection here everywhere in the museum. i will see
tomorrow if it works on my computer.

but probably in the daytime i will go around novi sad for a walk and
collect some impressions of the town.
so i will be online at 5pm for rehearsal.

see you,

cym



\chatbox{\stylechatpiece{189} }{\stylechatinfo{laure.dit 00:00}}{\stylechat{but the door i dont understand is it a sliding door}}

\chatbox{\stylechatpiece{190} }{\stylechatinfo{laure.dit 00:00}}{\stylechat{but the door i dont understand is it a sliding door}}

\chatbox{\stylechatpiece{191} }{\stylechatinfo{Nylon 00:01}}{\stylechat{hi}}

\chatbox{\stylechatpiece{192} }{\stylechatinfo{Nylon 00:01}}{\stylechat{hi}}

\chatbox{\stylechatpiece{193} }{\stylechatinfo{nathalie 00:01}}{\stylechat{hi judy}}

\chatbox{\stylechatpiece{194} }{\stylechatinfo{nathalie 00:01}}{\stylechat{hi judy}}

\chatbox{\stylechatpiece{195} }{\stylechatinfo{boris 23:21}}{\stylechat{now paula get him out}}

\chatbox{\stylechatpiece{196} }{\stylechatinfo{boris 23:21}}{\stylechat{and chris prepare to get in}}

\chatbox{\stylechatpiece{197} }{\stylechatinfo{boris 23:22}}{\stylechat{chris get in}}

\chatbox{\stylechatpiece{198} }{\stylechatinfo{boris 23:22}}{\stylechat{(paula watch out he shouldn't show his face)}}

\chatbox{\stylechatpiece{199} }{\stylechatinfo{boris 23:22}}{\stylechat{chris is in!!!}}

\chatbox{\stylechatpiece{200} }{\stylechatinfo{boris 23:22}}{\stylechat{manu prepare}}

\chatbox{\stylechatpiece{201} }{\stylechatinfo{boris 23:22}}{\stylechat{now chris get out}}

\chatbox{\stylechatpiece{202} }{\stylechatinfo{boris 23:23}}{\stylechat{manu get in}}


\page
\setuplayout[nomargins]
\placefigure[here,force]{none}{\externalfigure[/tmp/OSP\_AETHER9\_\_20071215\_23h23m35s\_FULL\_pallet\_0.png][width=\paperwidth,height=\paperheight]}
\placefigure[here,force]{none}{\externalfigure[/tmp/OSP\_AETHER9\_\_20071215\_23h23m35s\_FULL\_pallet\_1.png][width=\paperwidth,height=\paperheight]}
\placefigure[here,force]{none}{\externalfigure[/tmp/OSP\_AETHER9\_\_20071215\_23h23m35s\_FULL\_pallet\_2.png][width=\paperwidth,height=\paperheight]}
\placefigure[here,force]{none}{\externalfigure[/tmp/OSP\_AETHER9\_\_20071215\_23h23m35s\_FULL\_pallet\_3.png][width=\paperwidth,height=\paperheight]}
\page
\setuplayout[reset]

\chatbox{\stylechatpiece{204} }{\stylechatinfo{boris 23:23}}{\stylechat{soon F gets back to stool}}

\chatbox{\stylechatpiece{205} }{\stylechatinfo{boris 23:24}}{\stylechat{manu get out}}

\chatbox{\stylechatpiece{206} }{\stylechatinfo{boris 23:24}}{\stylechat{he gets on stool}}

\chatbox{\stylechatpiece{207} }{\stylechatinfo{boris 23:24}}{\stylechat{zoom on hands}}

\chatbox{\stylechatpiece{208} }{\stylechatinfo{boris 23:24}}{\stylechat{unzoom from hands}}

\chatbox{\stylechatpiece{209} }{\stylechatinfo{boris 23:25}}{\stylechat{(judy i went too fast i know...)}}

\chatbox{\stylechatpiece{210} }{\stylechatinfo{boris 23:25}}{\stylechat{he listens now}}

\chatbox{\stylechatpiece{211} }{\stylechatinfo{boris 23:32}}{\stylechat{now is manu}}

\chatbox{\stylechatpiece{212} }{\stylechatinfo{boris 23:33}}{\stylechat{manu looks imself in miror}}

\useexternalfigure[nlS/1r52SfWVhfvw/2PGVQ==][../PAR-PERFO/2007-12-15\_GhostTrio/img/20071215\_23h34m09s\_mirror.jpg]\hbox {\externalfigure[nlS/1r52SfWVhfvw/2PGVQ==][width=\getvariable{pageprops}{columnwidth},factor=max]}\blank[image]

\chatbox{\stylechatpiece{214} }{\stylechatinfo{boris 23:34}}{\stylechat{manu out of frame}}

\chatbox{\stylechatpiece{215} }{\stylechatinfo{boris 23:34}}{\stylechat{F gets on stool}}

\chatbox{\stylechatpiece{216} }{\stylechatinfo{boris 23:34}}{\stylechat{he listen now}}

\chatbox{\stylechatpiece{217} }{\stylechatinfo{boris 23:35}}{\stylechat{paila get corridor and kid ready...}}

\chatbox{\stylechatpiece{218} }{\stylechatinfo{boris 23:35}}{\stylechat{paula}}

\chatbox{\stylechatpiece{219} }{\stylechatinfo{boris 23:35}}{\stylechat{now i corridor}}

\chatbox{\stylechatpiece{220} }{\stylechatinfo{boris 23:35}}{\stylechat{now corridor and kid}}

\chatbox{\stylechatpiece{221} }{\stylechatinfo{boris 23:37}}{\stylechat{ok eventually the kid is there then...}}

\chatbox{\stylechatpiece{222} }{\stylechatinfo{boris 23:37}}{\stylechat{now F comes back to stool}}

\chatbox{\stylechatpiece{223} }{\stylechatinfo{boris 23:37}}{\stylechat{and kid out frame}}

\chatbox{\stylechatpiece{224} }{\stylechatinfo{boris 23:38}}{\stylechat{paula: corridor only or door only please}}

\chatbox{\stylechatpiece{225} }{\stylechatinfo{boris 23:38}}{\stylechat{now zoom on his hadns and face}}

\chatbox{\stylechatpiece{226} }{\stylechatinfo{boris 23:38}}{\stylechat{get ready to \stylerefchatlogslow{fade}\stylerefroad{49}{227} to colours}}

\chatbox{\stylechatpiece{227} }{\stylechatinfo{boris 23:38}}{\stylechat{\stylerefchatlogslow{fade}\stylerefroad{226}{229} out}}

\useexternalfigure[J20CeC6DAYbwXepfmD3p3Q==][../PAR-PERFO/2007-12-15\_GhostTrio/img/20071215\_23h39m04s\_face.jpg]\hbox {\externalfigure[J20CeC6DAYbwXepfmD3p3Q==][width=\getvariable{pageprops}{columnwidth},factor=max]}\blank[image]

\chatbox{\stylechatpiece{229} }{\stylechatinfo{boris 23:39}}{\stylechat{\stylerefchatlogslow{fade}\stylerefroad{227}{235} out everybody}}

\chatbox{\stylechatpiece{230} }{\stylechatinfo{boris 23:39}}{\stylechat{paula: blue only}}

\chatbox{\stylechatpiece{231} }{\stylechatinfo{boris 23:40}}{\stylechat{now chloé: credits}}

\chatbox{\stylechatpiece{232} }{\stylechatinfo{boris 23:40}}{\stylechat{the others: location}}

\useexternalfigure[G4CeuR0eWegH4UF+0OOhDQ==][../PAR-PERFO/2007-12-15\_GhostTrio/img/20071215\_23h41m17s\_END.jpg]\hbox {\externalfigure[G4CeuR0eWegH4UF+0OOhDQ==][width=\getvariable{pageprops}{columnwidth},factor=max]}\blank[image]

\useexternalfigure[ufqmBeuvSs09GW7haWZfcg==][../cover/20071215\_23h41m17s\_END\_cut.png]\hbox {\externalfigure[ufqmBeuvSs09GW7haWZfcg==][width=\getvariable{pageprops}{columnwidth},factor=max]}\blank[image]

\chatbox{\stylechatpiece{235} }{\stylechatinfo{boris 23:41}}{\stylechat{now \stylerefchatlogslow{fade}\stylerefroad{229}{236} out again}}

\chatbox{\stylechatpiece{236} }{\stylechatinfo{boris 23:41}}{\stylechat{everybody \stylerefchatlogslow{fade}\stylerefroad{235}{280} out again}}

\chatbox{\stylechatpiece{237} }{\stylechatinfo{nathalie 23:41}}{\stylechat{euh live to brussel too}}

\chatbox{\stylechatpiece{238} }{\stylechatinfo{nathalie 23:41}}{\stylechat{:)}}

\chatbox{\stylechatpiece{239} }{\stylechatinfo{nathalie 23:41}}{\stylechat{live from brussel}}

\useexternalfigure[UhLZHnP9aZsOMq/OM9ED3g==][../PAR-PERFO/2007-12-15\_GhostTrio/img/20071215\_23h42m20s\_credits.png]\hbox {\externalfigure[UhLZHnP9aZsOMq/OM9ED3g==][width=\getvariable{pageprops}{columnwidth},factor=max]}\blank[image]

\chatbox{\stylechatpiece{241} }{\stylechatinfo{paula vélez 23:42}}{\stylechat{shittttt}}

\chatbox{\stylechatpiece{242} }{\stylechatinfo{paula vélez 23:42}}{\stylechat{;(}}

\chatbox{\stylechatpiece{243} }{\stylechatinfo{nathalie 23:43}}{\stylechat{what paula?}}

\chatbox{\stylechatpiece{244} }{\stylechatinfo{boris 23:43}}{\stylechat{what happened?}}

\chatbox{\stylechatpiece{245} }{\stylechatinfo{paula vélez 23:43}}{\stylechat{(sweat)}}

\chatbox{\stylechatpiece{246} }{\stylechatinfo{nathalie 23:43}}{\stylechat{it 's nice performance}}

\chatbox{\stylechatpiece{247} }{\stylechatinfo{boris 23:43}}{\stylechat{yes!}}

\chatbox{\stylechatpiece{248} }{\stylechatinfo{nathalie 23:43}}{\stylechat{bravo all}}

\chatbox{\stylechatpiece{249} }{\stylechatinfo{boris 23:43}}{\stylechat{judy and jonathan: great!!!}}

\chatbox{\stylechatpiece{250} }{\stylechatinfo{fe2cruz 23:43}}{\stylechat{16 minute running time for this one. anarchy of the actors in colombia?}}

\chatbox{\stylechatpiece{251} }{\stylechatinfo{nathalie 23:43}}{\stylechat{yeap}}

\chatbox{\stylechatpiece{252} }{\stylechatinfo{boris 23:43}}{\stylechat{is cym around? Cyyyym?}}

\chatbox{\stylechatpiece{253} }{\stylechatinfo{paula vélez 23:43}}{\stylechat{to do the c orridor mouvement we didnt stop well the \stylerefchatlogtech{upload}\stylerefroad{180}{318} buton}}

\chatbox{\stylechatpiece{254} }{\stylechatinfo{nathalie 23:44}}{\stylechat{i find it's very well}}

\chatbox{\stylechatpiece{255} }{\stylechatinfo{boris 23:44}}{\stylechat{ok you will have one more \stylerefchatlogthemes{chance}\stylerefroad{178}{321}...}}

\chatbox{\stylechatpiece{256} }{\stylechatinfo{paula vélez 23:44}}{\stylechat{caus it was a friend and she think she was stoping but no}}

\chatbox{\stylechatpiece{257} }{\stylechatinfo{nathalie 23:44}}{\stylechat{but people not know that}}

\chatbox{\stylechatpiece{258} }{\stylechatinfo{luka 00:07}}{\stylechat{we have all sound, very good}}

\chatbox{\stylechatpiece{259} }{\stylechatinfo{luka 00:07}}{\stylechat{steam is running and nicely all over the room}}

\chatbox{\stylechatpiece{260} }{\stylechatinfo{luka 00:07}}{\stylechat{people are quiet and watching and listening}}

\chatbox{\stylechatpiece{261} }{\stylechatinfo{boris 00:07}}{\stylechat{great!}}

\chatbox{\stylechatpiece{262} }{\stylechatinfo{fe2cruz 00:07}}{\stylechat{ you can see the new images?}}

\chatbox{\stylechatpiece{263} }{\stylechatinfo{luka 00:08}}{\stylechat{looking good, all frames seem to be running...}}

\chatbox{\stylechatpiece{264} }{\stylechatinfo{luka 00:08}}{\stylechat{cym is not by the laptop,...}}

\chatbox{\stylechatpiece{265} }{\stylechatinfo{boris 00:08}}{\stylechat{zoom on hands}}

\chatbox{\stylechatpiece{266} }{\stylechatinfo{luka 00:08}}{\stylechat{making photos}}

\chatbox{\stylechatpiece{267} }{\stylechatinfo{boris 00:08}}{\stylechat{unzoom}}

\useexternalfigure[L1McbkvZLMcU1Ipl1AWjjw==][../PAR-PERFO/2007-12-15\_GhostTrio/img/20071216\_00h08m55s.png]\hbox {\externalfigure[L1McbkvZLMcU1Ipl1AWjjw==][width=\getvariable{pageprops}{columnwidth},factor=max]}\blank[image]

\chatbox{\stylechatpiece{269} }{\stylechatinfo{boris 00:09}}{\stylechat{ACT2}}

\chatbox{\stylechatpiece{270} }{\stylechatinfo{cym 00:09}}{\stylechat{my batteries of camera are empty}}

\chatbox{\stylechatpiece{271} }{\stylechatinfo{cym 00:09}}{\stylechat{no photos}}

\chatbox{\stylechatpiece{272} }{\stylechatinfo{boris 00:09}}{\stylechat{he listens now}}

\chatbox{\stylechatpiece{273} }{\stylechatinfo{cym 00:09}}{\stylechat{but it looks great here really great!}}

\chatbox{\stylechatpiece{274} }{\stylechatinfo{nathalie 00:09}}{\stylechat{mm}}

\chatbox{\stylechatpiece{275} }{\stylechatinfo{boris 00:09}}{\stylechat{he listens again}}

\chatbox{\stylechatpiece{276} }{\stylechatinfo{boris 00:09}}{\stylechat{paula get ready!}}

\chatbox{\stylechatpiece{277} }{\stylechatinfo{cym 00:09}}{\stylechat{just go on it is reall cool}}

\chatbox{\stylechatpiece{278} }{\stylechatinfo{boris 00:09}}{\stylechat{he leaves to paula}}

\chatbox{\stylechatpiece{279} }{\stylechatinfo{boris 00:10}}{\stylechat{now paula!}}

\chatbox{\stylechatpiece{280} }{\stylechatinfo{cym 00:10}}{\stylechat{images are so \stylerefchatlogslow{slow}\stylerefroad{236}{324} here}}

\chatbox{\stylechatpiece{281} }{\stylechatinfo{cym 00:10}}{\stylechat{but voice is great}}

\chatbox{\stylechatpiece{282} }{\stylechatinfo{cym 00:10}}{\stylechat{and people are very very concentrated watching}}

\chatbox{\stylechatpiece{283} }{\stylechatinfo{boris 00:10}}{\stylechat{chris get ready}}

\chatbox{\stylechatpiece{284} }{\stylechatinfo{boris 00:10}}{\stylechat{paula get man out!!!}}

\chatbox{\stylechatpiece{285} }{\stylechatinfo{boris 00:12}}{\stylechat{now manu get out!}}

\chatbox{\stylechatpiece{286} }{\stylechatinfo{boris 00:12}}{\stylechat{lol}}

\chatbox{\stylechatpiece{287} }{\stylechatinfo{boris 00:12}}{\stylechat{back on stool}}

\chatbox{\stylechatpiece{288} }{\stylechatinfo{boris 00:13}}{\stylechat{zoom on hands}}

\chatbox{\stylechatpiece{289} }{\stylechatinfo{boris 00:13}}{\stylechat{unzoom}}

\chatbox{\stylechatpiece{290} }{\stylechatinfo{boris 00:13}}{\stylechat{he listens}}

\chatbox{\stylechatpiece{291} }{\stylechatinfo{boris 00:14}}{\stylechat{he goes to paula}}

\useexternalfigure[HBjCNv8CmqdsN26mwWHJTQ==][../PAR-PERFO/2007-12-15\_GhostTrio/img/20071216\_00h14m14s\_again.png]\hbox {\externalfigure[HBjCNv8CmqdsN26mwWHJTQ==][width=\getvariable{pageprops}{columnwidth},factor=max]}\blank[image]

\chatbox{\stylechatpiece{293} }{\stylechatinfo{boris 00:14}}{\stylechat{paula it's you now}}

\chatbox{\stylechatpiece{294} }{\stylechatinfo{boris 00:15}}{\stylechat{now }}

\chatbox{\stylechatpiece{295} }{\stylechatinfo{boris 00:15}}{\stylechat{paula: man out of frame}}

\chatbox{\stylechatpiece{296} }{\stylechatinfo{boris 00:15}}{\stylechat{ok}}

\chatbox{\stylechatpiece{297} }{\stylechatinfo{boris 00:15}}{\stylechat{he comes back to stool}}

\stylepiece{298}

\stylemailtitle

new script for ghost trio.....looks more clear. JN

\styleinfos

Judy Nylon

17.12.2007

\stylemail


Re: feedback: Beyond the compliments and stirring of the imagination the useful thoughts from friends were as follows: 1) an actor said that it still lacks emotive room in the small screens and we might insert extreme close-up shots to allow us to receive the interior part of the acting. OR...soon maybe we will be able to feature that one square larger when something like that is going on in the script. 2) a filmmaker in Berlin loved the piece and was familiar with  Beckett and seemed to know that it was impossible to get the rights to do any adaptations of Beckett's plays.  Her only critical comment (among lots of praise) was that the credits were  impossible to follow and she wasn't wild about the handwritten  "live from  wherever" cards.  I thought seeing everyone on camera for a second was reminiscent  of the informal credits on a Doris Dorrie film. I liked it. 3) I had one person who is close to 70 years say that she couldn't get to it and that the interface still
 wasn't easy enough. I mention her because she is someone who has a foundation that gives grants and we do want to get everybody to be able to see what we do. 4) tomorrow at their Xmas party, I will hear if anyone at HWKs (www.harvestworks.org) saw this and will start them thinking about sponsoring us to have a gig at the New Museum which just opened a state of the art building on the Bowery a couple of weeks ago. I have been to Location1 (the art space I mentioned before) and feel that they have very little to offer us.

Onward....Thanks one and all...good show, it was a pleasure.  Judy

\stoptext

\stopproduct