\startproduct aether_2009

\starttext

\part[2009]{2009}

\marking[P]{2009}

\useexternalfigure[dw3xFrvXcdswg4cX5y6zPg==][../PAR-PERFO/2009-03-01-Uttarakhand/img/aether9-in-india.jpg]\medskip\hbox {\externalfigure[dw3xFrvXcdswg4cX5y6zPg==][width=\getvariable{pageprops}{columnwidth},factor=max]}\blank[image]

\stylepiece{539}

\styleinfoperfo
\startinfopar \perfotitle{N3krozoft Ltd featuring aether9}

\infopar 2009-03-01 00:00:00

\infopar 4th annual Carnival of e-Creativity

\infopar Sattal Estate, Uttarakhand (India)

\infopar Boris
Kish,
Chloé
Cramer,
Manuel
Schmalstieg,
Yuri
Gruznov\stopinfopar
\blackrule[color=black, width=60mm, height=0.5pt, depth=0mm]
\styleperfo
A STAGE PERFORMANCE INSPIRED BY EARLY SCIENTIFIC EXERIMENTS ON TELEPATHY AND EXTRA-SENSORY PERCEPTION (ESP). DURING THE PLAY, REMOTE PERFORMERS WERE SUBJECTED TO A SET OF TESTS DESIGNED BY PSYCHOLOGIST KARL ZENER (1903-1964).
\blackrule[color=black, width=60mm, height=0.5pt, depth=0mm]

\stylepiece{540}

\stylequoteblock{

\styleinfos

Paula Vélez

\stylequote



but i lost my typewriter... do you realise manu, someone can lost his laptop, but a typewriter? one day i will lost my head

}

\stylepiece{541}

\stylemailtitle

updated may 23rd performance order/shifts Sunsets Sunrises

\styleinfos christiaan cruz

19.05.2009

\stylemail


12 confirmed aethernaut streams
2 event locations/presenters
3 groups of 2 hour shifts between 2-7pm Colombia time
2 possible sunsets
1 possible sunrise
5 continents
8 timezones
11 cities
according time zones, possible sunsomething \& list request
feel \stylerefmailcash{free}\stylerefroad{581}{543} to trade times if needed

1st Group 2 hours 2PM Colombia Time


1.UTC/GMT+2: Cym, Presenting in Croatia
2.UTC/GMT+3: Mari Keski-Korsu. Tunnila Village, Finland possible @ SunDown 10PM?
3.UTC/GMT+2: Manuel Schmalstieg. Neuchâtel, Switzerland


2nd Group start @ 3:30PM Colombia Time
4.UTC/GMT+2: Frauke Frech, Kiel, Germany
5.UTC/GMT+2: Chloé Cramer. Brussels,Belgium
6.UTC/GMT+2: Boris Kish, Brussels,Belgium
7.UTC+9:30: Vinny Bhagat, Adelaide, South Australia Sunrising 6:13am

Last Group start @ 5:00PM Colombia Time


8.UTC/GMT-6: Alexat0r, Wisconsin or Colorado, USA


9.UTC/GMT-7: Christiaan Cruz. Yorba Linda, California


10.UTC/GMT-5: Paula Vélez. Streaming and Projecting in Medellín, Colombia

Continuous Group full 5 hours
11.UTC/GMT+5:30: Dhanya Pilo. Mumbai, India (15 min intervals)
12.UTC/GMT-4: Judy Nylon. New York. USA (30 min Intervals) Sunseting 7:14pm

\hyphenatedurl{http://www.worldtimeserver.com/convert\_time\_in\_CO.aspx?y=2009\&mo=5\&d=23\&h=14\&mn=0}
\hyphenatedurl{http://www.srrb.noaa.gov/highlights/sunrise

/sunrise.html}

---


\stylepiece{542}

\stylemailtitle

correct spelling, names etc?

\styleinfos Mari Keski-Korsu

20.05.2009

\stylemail


Hello,

Sorry for not being able to \stylerefmailchat{Skype}\stylerefroad{501}{543} lately. Could someone write a short, 
simple summary about the actions done simultaneously? I won't be able to 
see or hear you due to the \stylerefmailslow{slow}\stylerefroad{501}{544} internets and this would help. There was 
something about \stylerefmailthemes{luck}\stylerefroad{323}{543} in group changes and then there was something about 
sunset? I'll be able to stream a little bit about the sunset, it is 
setting behind the trees around the time of the my stream. I can of 
course try to climb on the roof of the house (a bit scared about that). 
What is the idea with the sunset? Is it about just showing a sunset what 
ever way we feel like or is there some sort of concept like everyone is 
trying to show it from same angles or something?

mkk
---


\stylepiece{543}

\stylemailtitle

skype summary main poimts

\styleinfos christiaan cruz

20.05.2009

\stylemail


Mari \& all the Aethers,

To make it easier I think rather than \stylerefmailthemes{luck}\stylerefroadfwdonly{544} they mean: \stylerefmailthemes{chance}\stylerefroadrwdonly{542}

Props of luck gambling or chance ect:
Dice
drawing lots - sticks
throwing bones
games of chance
play
gaming imagery
Aethers Aleatoric

synchronization will also happen by chance
hopefully we can work with you via \stylerefmailchat{Skype}\stylerefroad{542}{545}
otherwise we can also text your phone messages SMS
we can instruct you to add colors or adjust your stream
simple instructions like its too busy or a cooler or warmer color
likewise if skype is too much for your connection you may SMS me too
213-806-6962
\stylerefmailcash{free}\stylerefroad{541}{544} via \hyphenatedurl{http://www.gizmosms.com/}

we'll have 1 person in each group function like a moderator
and try to keep all of the cells even and balanced

sunset/sunrise is just another way to accent our locations
if the trees are in your way don't climb your roof
its not important as not many of us won't have a sunrise or sunset
just have a local paper, a clock or snow
anything live from your location that the rest of us won't have

It seems like a proper rehearsal may be difficult to organize.
too many timezones. However since this is a festival about
imporvisation and our main theme could be
 Moving Images of Chance
then rehearsals may actually be the wrong thing to do.

I suggest all streamers just gather as many objects that they can.
Objects that might support or even oppose the idea of Chance.
Lots of Colors, pictures, books. Be prepared to write or Draw.
Have prisms, Kaliedescopes or lenses to put over your camera.
Try puppets in lights or just as shadows.
Use mirrors to open up your space.
Be aware of the other features available in the \stylerefmailpatch{Patch}\stylerefroad{370}{545}: cam-links \& effects
have your camera near a nice window to grab imagery outside
or even stream outside in a nice location as long as the light isn't an issue

Everyone has at least 2 hours. So we should all bring lots of things
to play with in front of the camera and be ready to change those things
if your moderator feels your cell needs to be adjusted. Anywone with
luck enough to have connections for proper viewing and listening of
all the streams should try to communicate to everyone how all 9 cells
are working and if any adjustments need to be made.

---


\stylepiece{544}

\stylemailtitle

personnal summary

\styleinfos chloé cramer

20.05.2009

\stylemail


*Those are my notes about the content of saturday's jam *:
playing cards
about \stylerefmailthemes{luck}\stylerefroad{543}{545} : being lucky (can lead to any personal interpretation)
clocks (time zone, time on location)

following the music : improvising on the music.

abstract lines ...


*Judy wrote :*
**
Nature Weightless Wet Anxious Travel
Hot Languid Food Soft \stylerefmailcash{money}\stylerefroad{543}{545} Couples/Pairs
Dance Industrial Ancient Magic Cool Depth
Focal Shift
NECESSITY TO PREPARE DIFFERENT KIND OF PROPS, ACCORDING TO THE SUBJECT.
I would like to add that the mutual watching of other frames should not be
forgotten, so that we really play together (if there is no master of
ceremony). Maybe some frames could also be passive or still, or showing a
\stylerefmailslow{slow}\stylerefroad{542}{545} minimalistic action, in order to llow other frames to catch the viewers
attention. Let's consider each other as an orchestra and trully imagine what
it sounds like when all musicians are playing solo in the same time...
by the way, is there a leader of the performance or not ?!
See you soon
chloé
---


\stylepiece{545}

\stylemailtitle

resume, chat and ideas for JAM

\styleinfos Paula Vélez Bravo

22.05.2009

\stylemail


LIST OF PARTICIPANTS. JAM 23-05-09 (A list of confirmed participants  
and a schedule, division in time/slots)

Final Roll Call, Info, Credits -> Ready To Go!


12 confirmed aethernaut streams
2 event locations/presenters
3 groups of 2 hour shifts between 2-7pm Colombia time
5 continents
8 timezones
11 cities

1st Group 2 hours 2PM Colombia Time


1.UTC/GMT+2: Cym, Presenting in Croatia
  \hyphenatedurl{http://networkcultures.org/wpmu/videovortex/video-vortex-split/}
2.UTC/GMT+3: Mari Keski-Korsu. Tunnila Village, Finland
3.UTC/GMT+2: Manuel Schmalstieg. Neuchâtel, Switzerland


2nd Group start @ 3:30PM Colombia Time


4.UTC/GMT+2: Frauke Frech, Kiel, Germany
5.UTC/GMT+2: Chloé Cramer. Brussels,Belgium
6.UTC/GMT+2: Boris Kish, Brussels,Belgium
7.UTC+9:30: Vinny Bhagat, Adelaide, South Australia Sunrising 6:13am

Last Group start @ 5:00PM Colombia Time


8.UTC/GMT-6: Axelat0r, Wisconsin, USA
9.UTC/GMT-7: Christiaan Cruz. Yorba Linda, California
10.UTC/GMT-5: Paula Vélez. Streaming and Projecting in Medellín,  
Colombia
   \hyphenatedurl{http://translate.google.com/translate?u=http\%3A\%2F\%2Fwww.elmamm.org\%2Fsitio\%2Fprogramacion\%2Ftranssesiones.html\&sl=es\&tl=en\&hl=en\&ie=UTF-8}

Continuous Group full 5 hours


11.UTC/GMT+5:30: Dhanya Pilo. Mumbai, India (15 min intervals)
12.UTC/GMT-4: Judy Nylon. New York. USA (30 min Intervals) Sunseting  
7:14pm

\hyphenatedurl{http://www.worldtimeserver.com/convert\_time\_in\_CO.aspx?y=2009\&mo=5\&d=23\&h=14\&mn=0}
\hyphenatedurl{http://www.srrb.noaa.gov/highlights/sunrise/sunrise.html}


AETHER9 5 horas de JAM (improvisación y música)
Trans-Sesiones \hyphenatedurl{https://transesiones.wordpress.com}
Sábado 23 de mayo 2009. Desde las 2:00 pm hasta las 7:00 pm
MAMM. Museo de Arte Moderno de Medellín, Colombia

12 transmisiones de AETHERnautas confirmados
2 eventos dos locaciones/presentadores
3 grupos de 2 horas cada uno entre las 2-7 pm Horario Colombia
5 continentes
8 zonas horarias
11 ciudades

aqui la hora del jam en su zona horaria:
\hyphenatedurl{http://www.worldtimeserver.com/convert\_time\_in\_CO.aspx?y=2009\&mo=5\&d=23\&h=14\&mn=0}
\hyphenatedurl{http://www.srrb.noaa.gov/highlights/sunrise/sunrise.html}

SET UP AND EQUIPEMENTS IN MEDELLIN for streaming and projection.

1. Two screens (one for \stylerefmailchat{Skype}\stylerefroadrwdonly{543} \stylerefmailchat{chat}\stylerefroadfwdonly{562} and \stylerefmailpatch{Patch}\stylerefroad{543}{581} the other for MAIN AETHER  
WINDOW)
2. Two projectors.
3. Sound transmission   (i would like to know MANU, if that sound  
patch i'm using could have a doble transmission to another channel  
where quality could be maximum as Chris proposed?)

\hyphenatedurl{http://giss.tv:8000/aether.mp3}

4. Video channels transmission:
\hyphenatedurl{http://1904.cc/}
\hyphenatedurl{http://www.livestream.com/transsesiones}
\hyphenatedurl{http://www.ustream.tv/channel/transsesiones}

RECORDING.
Who will record the hole performance?
We should maybe asign different people for different moments.. it will  
be long!

Who will record sound?
who will record video?

I will try to do it here, but ... cool if someonelse do it.
one at cd quality and the other really low


IDEAS for JAM we talked in last chats

At 2:00 pm (COL time) we will try to start at time with this jam

CYM gives the signal, in a moment of *highter intensity* will start.  
Just need to be sure everyone is clear and knows they must write  
"greetings to split" when Cym gives that command around 20minutes into  
our performance

and then 'greetings to split', in different languages.
(Like, projection in Croatia. Videovortex Festival. Cym, hello from  
Vorva Linda, California, Here  is the sun in Australia, C'est le  
printemps à Brusels,...


1.Playing cards, dice, coins, casino...
about \stylerefmailthemes{luck}\stylerefroadfwdonly{582} : being lucky --> astrology, karma
clocks.  could be a nice "leitmotiv", a clock , like an old alarm. to  
use it in BREAKS.

We say the comand: CLOCKS and everyone put clocks in their frames.

2. Drawings, abstract lines, B\&W

3. \stylerefmailslow{ambience}\stylerefroad{544}{581}, landscapes, and using remote webcams.

4. Simulating people making, "Foley sound" . Crunching, steps, claps,  
hammers, using in the image objects that makes think about sound!

5. Simulating playing instruments, following music.

6.Adapting to less frames.

7.Words to play with:

Luck Nature    Weightless   Wet   Anxious   Travel
Hot   Languid   Food   Soft   \stylerefmailcash{money}\stylerefroad{544}{581}   Couples/Pairs
Dance   Industrial   Ancient   Magic   Cool   Depth
Focal Shift

8. spaceships, \stylerefmailscience{Tesla}\stylerefroadrwdonly{539}, transmision, theremyn... representation of  
\stylerefmailscience{magnetic}\stylerefroadfwdonly{548} fields..

9. Toys, objects, playhouse

10. try to play simon says.

11. use \stylerefmailred{red}\stylerefroad{539}{617}, green, bleu patch colors, and contrast B\&W images... some  
times of course colored and what ever you want of course....


THINK about a moment of a little perfomance for 30 minutes, to make  
muscicians do the soundtrack live.... (aether9 making images and  
muscicians following)



the interface with 9 frames
dissolve interface
  sometimes some of the frames black
abstract lines

the teams themselves can decide how to lay things out
hell they can move around in just one cell if they want too



IDEAS IN THE AIR of  posible STRUCTUREs FOR THE JAM
For the JAM I have being talking to musicians to see if they have an  
structure to follow. Because i think it is very nice to have knoledge  
of that to prepare something for aether9.

1. One of the musicians will base his part of the JAM direction  
working with SILENCE in music.
2. Another one will work with a synth and use, from the computer,  
algoritms like dinamic systems . That will not be repetitive sound  
secuences. Some could take a long time, others not.
\stylerefmailthemes{random}\stylerefroadrwdonly{544} typical aether atmospheres

1. Writing in skype session , needs of noises, voices, music, strings,  
punctual "bruitage".
2. Using colors as we have being doing, in frames we want to have  
particulary "musicalised".Like in SIMON SAYS toy.
\hyphenatedurl{http://www.youtube.com/watch?v=2jmVU2eykuI}


ambience

sound effects

foley sound
---



\useexternalfigure[LoIyIYCdHWt4ps5NQZ3V1w==][../PAR-PERFO/2009-05-23\_Medellin/img/words.jpg]\medskip\hbox {\externalfigure[LoIyIYCdHWt4ps5NQZ3V1w==][width=\getvariable{pageprops}{columnwidth},factor=max]}\blank[image]

\useexternalfigure[iGyBvxpeb1s1Dt284b7bhw==][../PAR-PERFO/2009-05-23\_Medellin/img/watches.jpg]\medskip\hbox {\externalfigure[iGyBvxpeb1s1Dt284b7bhw==][width=\getvariable{pageprops}{columnwidth},factor=max]}\blank[image]

\stylepiece{548}

\styleinfoperfo
\startinfopar \perfotitle{Jam Session}

\infopar 2009-05-23 00:00:00

\infopar Trans-Sesiones

\infopar Museo de Arte Moderno, Medellín (Colombia)/Video Vortex 4, Split (Croatia)

\infopar Axelat0r,
Boris
Kish,
Chloé
Cramer,
Christiaan
Cruz,
Dhanya
Pilo,
Frauke
Frech,
Judy
Nylon,
Manuel
Schmalstieg,
Mari
Keski-Korsu,
Paula
Vélez,
Vinny
Bhagat\stopinfopar
\blackrule[color=black, width=60mm, height=0.5pt, depth=0mm]
\styleperfo
MARATóN DE IMPROVISACIóN VISUAL REMOTO. AETHER9 , ACEPTó UNIRSE A ESTE EJERCICIO DE IMPROVISACIóN Y EXPERIMENTACIóN DURANTE LAS 5 HORAS QUE DURARá EL JAM.
\blackrule[color=black, width=60mm, height=0.5pt, depth=0mm]

\useexternalfigure[H6rhdt6RnCN/+hkRo5kJsg==][../PAR-PERFO/2009-05-23\_Medellin/img/20090523\_20h30m28s\_preparing.png]\medskip\hbox {\externalfigure[H6rhdt6RnCN/+hkRo5kJsg==][width=\getvariable{pageprops}{columnwidth},factor=max]}\blank[image]

\useexternalfigure[3NN/m94Qu8fMVBp6czG2GA==][../cover/20090523\_20h30m28s\_preparing.png]\medskip\hbox {\externalfigure[3NN/m94Qu8fMVBp6czG2GA==][width=\getvariable{pageprops}{columnwidth},factor=max]}\blank[image]

\useexternalfigure[VzB04AB8r7egmGyb3jnEGw==][../PAR-PERFO/2009-05-23\_Medellin/img/20090523\_20h34m28s\_preparing.jpg]\medskip\hbox {\externalfigure[VzB04AB8r7egmGyb3jnEGw==][width=\getvariable{pageprops}{columnwidth},factor=max]}\blank[image]


\page
\setuplayout[nomargins]
\placefigure[here,force]{none}{\externalfigure[/tmp/OSP\_AETHER9\_\_20090523\_20h37m20s\_FULL\_preparing\_0.png][width=\paperwidth,height=\paperheight]}
\placefigure[here,force]{none}{\externalfigure[/tmp/OSP\_AETHER9\_\_20090523\_20h37m20s\_FULL\_preparing\_1.png][width=\paperwidth,height=\paperheight]}
\placefigure[here,force]{none}{\externalfigure[/tmp/OSP\_AETHER9\_\_20090523\_20h37m20s\_FULL\_preparing\_2.png][width=\paperwidth,height=\paperheight]}
\placefigure[here,force]{none}{\externalfigure[/tmp/OSP\_AETHER9\_\_20090523\_20h37m20s\_FULL\_preparing\_3.png][width=\paperwidth,height=\paperheight]}
\page
\setuplayout[reset]

\useexternalfigure[jfdSfipKYGi2XWrrs0Z5WA==][../PAR-PERFO/2009-05-23\_Medellin/img/20090523\_20h57m22s\_preparing.jpg]\medskip\hbox {\externalfigure[jfdSfipKYGi2XWrrs0Z5WA==][width=\getvariable{pageprops}{columnwidth},factor=max]}\blank[image]

\useexternalfigure[CymqtFxM/ySXxDo60lmycg==][../PAR-PERFO/2009-05-23\_Medellin/img/20090523\_20h59m20s\_starting.jpg]\medskip\hbox {\externalfigure[CymqtFxM/ySXxDo60lmycg==][width=\getvariable{pageprops}{columnwidth},factor=max]}\blank[image]

\useexternalfigure[vi6EoYzmtRTpVKeSdhw5ig==][../PAR-PERFO/2009-05-23\_Medellin/img/20090523\_21h24m54s\_started.jpg]\medskip\hbox {\externalfigure[vi6EoYzmtRTpVKeSdhw5ig==][width=\getvariable{pageprops}{columnwidth},factor=max]}\blank[image]

\useexternalfigure[BhR7HkL1kj0RwyHfSHUiBQ==][../PAR-PERFO/2009-05-23\_Medellin/img/20090523\_21h26m28s\_started.jpg]\medskip\hbox {\externalfigure[BhR7HkL1kj0RwyHfSHUiBQ==][width=\getvariable{pageprops}{columnwidth},factor=max]}\blank[image]

\useexternalfigure[v2B0xer+N2fgAbrlGu2upw==][../PAR-PERFO/2009-05-23\_Medellin/img/20090523\_21h35m26s\_seeds.jpg]\medskip\hbox {\externalfigure[v2B0xer+N2fgAbrlGu2upw==][width=\getvariable{pageprops}{columnwidth},factor=max]}\blank[image]

\useexternalfigure[eGNKs7xsyjBJeKkA/vJ6UA==][../PAR-PERFO/2009-05-23\_Medellin/img/20090523\_21h58m20s\_.png]\medskip\hbox {\externalfigure[eGNKs7xsyjBJeKkA/vJ6UA==][width=\getvariable{pageprops}{columnwidth},factor=max]}\blank[image]

\stylepiece{559}

\stylemailtitle

aether proceedings volume 1 in print

\styleinfos

1.1

30.05.2009

\stylemail


hi list,

so finally i created the first volume of the "aether9 proceedings" print 
series, where the idea is to collect all the chat sessions occurring 
before and during performances.

the first issue contains the chat logs of May 2nd and 3rd 2007, during 
our first performance at Mapping festival 2007.

the PDF file is available here: 
\hyphenatedurl{http://1904.cc/~aether/material/aether\_proceedings/}

i'm still going to revise this edition and add an Editorial Foreword / 
Introduction and Conclusion / performers list ... tell me if you have 
suggestions of what else could/should be included there.

more to follow!

best,
manuel





\stylepiece{560}

\stylequoteblock{

\styleinfos

Audrey Samson

\stylequote



read the publication PDF - i really enjoyed it. partially nostalgia of course, but i really appreciate how the tech discussion is interwoven with script/concept discussion and also occasional absurdities. these processes cannot be clearly divided (or categorised seperately as such) and i think it comes out wonderfully in this chat (also there is a nice culmination to the performance - whilst being an anticlimax).

}

\stylepiece{561}

\stylemailtitle

ok for 22nd

\styleinfos Frauke Frech

17.07.2009

\stylemail


alright then, without any rehearsal
-was just wondering...
anyways i like authentic situations better!

when is starting time on 22nd?

frauke
---


\stylepiece{562}

\stylemailtitle

22nd call 9pm Manila time

\styleinfos christiaan cruz

17.07.2009

\stylemail


For the officical Aether Manila Sextet
Performer's Call time is 9pm Manila
with official show start-time 10pm Manila
get your time here:
\hyphenatedurl{http://www.worldtimeserver.com/convert\_time\_in\_PH.aspx?y=2009\&mo=7\&d=22\&h=21\&mn=0}


Confirmed performers are:
Manuel presentation + performance
Paula presentation + performance
Mari performance
chloe performance
christiaan presentation + performance moderator/screencap
frauke performance
Vinny performance
Tengal - organizer ASEUM Philippines
?Kim \stylerefmailchat{Skype}\stylerefroad{545}{581} presentation
boris maybe performance

>>>>>>>>>>>>>>>>>>>>>>>>>>>>>>>>>>>>>>>>>>>>>>>
Presentation meeting is locked in for Sunday Evening Europe
19th of July, Aether skype meeting
11-1pm  Califirnia
1-3pm   Colombia
7-9PM   Highlander Time UK
8-10pm  Brussells
2-4am   Manila/Taipei/HK
other times:
\hyphenatedurl{http://www.worldtimeserver.com/convert\_time\_in\_US-CA.aspx?y=2009\&mo=7\&d=19\&h=11\&mn=0}
---


\useexternalfigure[qlzSfElxB0K2snCDZEC/8A==][../PAR-PERFO/2009-07-25\_Manila/img/20090722\_14h27m50s.jpg]\medskip\hbox {\externalfigure[qlzSfElxB0K2snCDZEC/8A==][width=\getvariable{pageprops}{columnwidth},factor=max]}\blank[image]

\useexternalfigure[vdlsnHz28bkSQUOVCiENsQ==][../PAR-PERFO/2009-07-25\_Manila/img/20090722\_14h28m25s.jpg]\medskip\hbox {\externalfigure[vdlsnHz28bkSQUOVCiENsQ==][width=\getvariable{pageprops}{columnwidth},factor=max]}\blank[image]

\useexternalfigure[908UuQ90095Elt2dL7SKNA==][../PAR-PERFO/2009-07-25\_Manila/img/20090722\_14h30m12s.jpg]\medskip\hbox {\externalfigure[908UuQ90095Elt2dL7SKNA==][width=\getvariable{pageprops}{columnwidth},factor=max]}\blank[image]

\useexternalfigure[QI13mVydh93o+54tUbIPgw==][../PAR-PERFO/2009-07-25\_Manila/img/20090722\_14h33m55s.jpg]\medskip\hbox {\externalfigure[QI13mVydh93o+54tUbIPgw==][width=\getvariable{pageprops}{columnwidth},factor=max]}\blank[image]

\useexternalfigure[8gEzRnTf9z0BXFvlKGV0sQ==][../PAR-PERFO/2009-07-25\_Manila/img/20090722\_14h35m19s.jpg]\medskip\hbox {\externalfigure[8gEzRnTf9z0BXFvlKGV0sQ==][width=\getvariable{pageprops}{columnwidth},factor=max]}\blank[image]

\useexternalfigure[6CaOaJN6yVsZkH/SVGSI6A==][../PAR-PERFO/2009-07-25\_Manila/img/20090722\_14h46m06s.jpg]\medskip\hbox {\externalfigure[6CaOaJN6yVsZkH/SVGSI6A==][width=\getvariable{pageprops}{columnwidth},factor=max]}\blank[image]

\useexternalfigure[9AqszlnfsJoOl6z9vCds/Q==][../PAR-PERFO/2009-07-25\_Manila/img/20090722\_14h47m08s.jpg]\medskip\hbox {\externalfigure[9AqszlnfsJoOl6z9vCds/Q==][width=\getvariable{pageprops}{columnwidth},factor=max]}\blank[image]

\useexternalfigure[DgUSnXyAKhJpzt0W670ljA==][../PAR-PERFO/2009-07-25\_Manila/img/20090722\_14h57m08s.jpg]\medskip\hbox {\externalfigure[DgUSnXyAKhJpzt0W670ljA==][width=\getvariable{pageprops}{columnwidth},factor=max]}\blank[image]

\useexternalfigure[GazCA4LSVFj3oiyaGXNccw==][../PAR-PERFO/2009-07-25\_Manila/img/20090722\_14h58m29s.jpg]\medskip\hbox {\externalfigure[GazCA4LSVFj3oiyaGXNccw==][width=\getvariable{pageprops}{columnwidth},factor=max]}\blank[image]

\useexternalfigure[DTTmtR3ItSmSyIfbV2kmqA==][../PAR-PERFO/2009-07-25\_Manila/img/20090722\_15h00m23s.jpg]\medskip\hbox {\externalfigure[DTTmtR3ItSmSyIfbV2kmqA==][width=\getvariable{pageprops}{columnwidth},factor=max]}\blank[image]

\useexternalfigure[C7G7mxq/iUn55vFerhPW7w==][../PAR-PERFO/2009-07-25\_Manila/img/20090722\_15h09m20s.jpg]\medskip\hbox {\externalfigure[C7G7mxq/iUn55vFerhPW7w==][width=\getvariable{pageprops}{columnwidth},factor=max]}\blank[image]

\useexternalfigure[ussEPyYhY16IWZGLO6CCEg==][../PAR-PERFO/2009-07-25\_Manila/img/20090722\_15h13m48s.jpg]\medskip\hbox {\externalfigure[ussEPyYhY16IWZGLO6CCEg==][width=\getvariable{pageprops}{columnwidth},factor=max]}\blank[image]

\useexternalfigure[ZHCl2sAZw3qwPPZMi0K2lQ==][../PAR-PERFO/2009-07-25\_Manila/img/20090722\_15h41m17s.jpg]\medskip\hbox {\externalfigure[ZHCl2sAZw3qwPPZMi0K2lQ==][width=\getvariable{pageprops}{columnwidth},factor=max]}\blank[image]

\useexternalfigure[TnW/pOu0uP5Liqhip1uv0A==][../PAR-PERFO/2009-07-25\_Manila/img/20090722\_15h46m39s.jpg]\medskip\hbox {\externalfigure[TnW/pOu0uP5Liqhip1uv0A==][width=\getvariable{pageprops}{columnwidth},factor=max]}\blank[image]

\useexternalfigure[ilRo4vdlq2+ga/yCufPZzQ==][../PAR-PERFO/2009-07-25\_Manila/img/20090722\_16h15m37s.jpg]\medskip\hbox {\externalfigure[ilRo4vdlq2+ga/yCufPZzQ==][width=\getvariable{pageprops}{columnwidth},factor=max]}\blank[image]

\useexternalfigure[aUCMWxn/wiGus2rs88WuwQ==][../PAR-PERFO/2009-07-25\_Manila/img/20090722\_16h24m05s.jpg]\medskip\hbox {\externalfigure[aUCMWxn/wiGus2rs88WuwQ==][width=\getvariable{pageprops}{columnwidth},factor=max]}\blank[image]

\useexternalfigure[7k0jGJgYyYuG/Vqv8ZNktA==][../PAR-PERFO/2009-07-25\_Manila/img/20090722\_17h14m31s.jpg]\medskip\hbox {\externalfigure[7k0jGJgYyYuG/Vqv8ZNktA==][width=\getvariable{pageprops}{columnwidth},factor=max]}\blank[image]

\stylepiece{580}

\styleinfoperfo
\startinfopar \perfotitle{aether9}

\infopar 2009-07-25 00:00:00

\infopar ASEUM

\infopar Manila (Philippines)

\infopar Christiaan
Cruz,
Dhanya
Pilo,
Frauke
Frech,
Kim
Xupei,
Manuel
Schmalstieg,
Mari
Keski-Korsu,
Paula
Vélez,
Vinny
Bhagat\stopinfopar
\blackrule[color=black, width=60mm, height=0.5pt, depth=0mm]
\styleperfo
IMPROVISED TRANSMISSION USING A 6-FRAME INTERFACE IN COMBINATION WITH LIVE MUSICAL PERFORMANCES AT THE SABAW MEDIA KITCHEN IN MANILA.
\blackrule[color=black, width=60mm, height=0.5pt, depth=0mm]

\stylepiece{581}

\stylemailtitle

ASEUM epilogue

\styleinfos christiaan cruz

27.07.2009

\stylemail


Don't fret Chloé,

Manuel experienced the brunt of the term "Filipino Time"
Its an island mentality where things always seem to start
go or arrive later than proposed or expected. I had figured
this would happen so I did not want to burden everyone with
rehearsals and other meetings.

I've been on \stylerefmailchat{chat}\stylerefroad{562}{589} with Tengal since the festival and
everything seems to have worked out for the best. It
really was the biggest international symposium in that
lower region of Asia every assembled and Aether was
a big part of it. All the weird technical communications
issues and lack of proper \stylerefmailcash{funding}\stylerefroadfwdonly{541} aside it really was
a Success for the region. I have been involved in
earlier symposiums where they were not much more
than a website and mailing list. This one had real
events and exchanges.

They had lots of fun listening to Manuel do his presentation.
Even though we had communications issues and we started
\stylerefmailslow{slow}\stylerefroad{545}{611} the large audience still did enjoy listening to Manuel
count backwards. All those Filipino artists and students
hardly ever hear European accents. Its always Hollywood/TV
american english.

So maybe we could all go to the Philippines in the flesh
next year with any new \stylerefmailcash{grant} \stylerefmailcash{money}\stylerefroadrwdonly{545}? I go there regularly
so I wouldn't need a ticket. I'm sure that \stylerefmailcash{free} housing
would be available then too.

Also sincere apologies to Vinny who was streaming to
dead frames 7-9. We will be sure to Specify that those
cells are not working when we use the SEXTET. maybe we
could create a warning or flag in the \stylerefmailpatch{Patch}\stylerefroad{545}{595}.

Thanks to everyone that participated via streaming and or
just piping in on the list.

now on to -> Berlin \& then cimatics?

-christiaan
---




\stylepiece{582}

\stylemailtitle

berlin PREPARATION

\styleinfos chloé cramer

03.09.2009

\stylemail

- HOW?
Determine the way we are going to discuss:
> are we going to have a moderator? (A person who takes care that anyone has
the opportunity to express and also refrains us to spend more time than
necessary on each subject)
> are we going to work in small groups or alltogether?
> How much time do we give for general matters and separate topics / how
much time for the performance developpment itself.
> regarding the performance, knowing how long it takes to develop a
performance, it might be good to define it before the actual meeting and
start working on it before, in order to spend more "practical rehearsing
time" than brain-storming/developpment time.
> should we make a schedule with morning, afternoon and night sessions ?

- PREPARATION OF BERLIN
In order to have clear and efficient discussion, we propose to reflect
before the Meeting. Here are a couple of propositions FOR EVERY PERSON WHO
WILL ATTEND THE BERLIN MEETING (physically or remotly):
> reflect on the above subjects.
> if you have a performance idea, summarize it (setting, story, aesthetic,
or any other feature)
> prepare a small selection of screenshots (maximum 5) of your favorite
aether9 moments (that you find inspiring, beautiful, interesting or
whatever) and explain why in a few words (to be presented and discussed at
the begginging of the Meeting)

!!! THE IDEA IS NOT TO HAVE BIG MAILING-LIST CONVERSATION NOW !!! We have
the \stylerefmailthemes{chance}\stylerefroad{545}{623} to have a physical meeting to communicate more directly. The idea
is to prepare this meeting according to the subjects we would like to
discuss, and anyone who has strong wish to discuss any aspect of the aether9
should express it before, so that we are ready to interact.

OK ?
See you soon,
Chloé (\& Boris)
---


\stylepiece{583}

\stylemailtitle

berlin PREPARATION

\styleinfos

ideacritik

12.09.2009

\stylemail



On Sat, September 12, 2009 3:36 pm, 1.1 wrote:
> i agree with audrey, thinking that it will be very important to focus on
> global priorities of the aether project, not only working for the one
> performance .... it seemed to me recently that jam-like performances
> worked quite well, when enough people are involved.
>
> talking about phone apps, do any of you people have any experience with
> mobile phone platforms, i.e. are you using iphones? nokia linux phones?
> google android phones?
>

i have a dinosaur phone (not even color display! :)
however i could borrow some smart phones from this friend that does the
treasuremapper. also if we are serious about the phone idea i could meet
him (here in NL) and see if he thinks his thing can merge with ours and
how complicated it is technically (i.e. is it feasible to
achieve/implement in the berlin week).

otherwise i have no experience with such things... but always willing to
learn of course!! so let me know if i should suss out the situation.

> wondering if any "webcam2ftp" application exists for the iphone already?

no clue. does anyone have an iphone? ... (last year all my students did...
how depressing).

>
> looking fwd,
> manuel
---


\stylepiece{584}

\stylemailtitle

berlin flat/munich

\styleinfos ideacritik

16.09.2009

\stylemail


regarding the flat - i also trust your judgement manu.

in addition, i spoke to cym yesterday (IRL!!) and she might be coming as
well so i think a bigger flat is best -

i also think it would be nice to have us all in the same space - though
that will be extremely intense and ironically diametrically opposite to
our usual relationship : )

audrey
---


\stylepiece{585}

\stylemailtitle

impotent science

\styleinfos

christiaan cruz

19.09.2009

\stylemail


I woke up, made fresh coffee and had perky j-pop playing.
Still don't think i'm scientific, romantic or caffinated
enough to respond to you properly audrey.
glad you liked the podcast sans the announcer

the list feels warmer than others to me anyways
because it feels like a theatre group. and you
get a reward kinda feeling too since you actually
participate and do things rather than just talk
or obsess over them




\stylepiece{586}

\stylequoteblock{

\styleinfos

Audrey Samson

\stylequote



one of the things i have always liked the most about aether is its non-slick, stop-motion like and pixelated aesthetic. for me this challenges many important hypes:
-the constant 'newness' of new media (higher speed, more pixels, faster cpus, etc)
-the ever higher resolution (which in the end is simply not accessible for most of the world which is also an important aether9 concern)
and most importantly misguided idea that these things contribute to the value of the message.

}

\stylepiece{587}

\stylemailtitle

way to berlin

\styleinfos Paula Vélez Bravo

24.10.2009

\stylemail


Tomorrow I will take my plane,

long trip: MEdellín-Bogotá-Madrid\_Paris\_Berlin.

see you then monday.

bye

Paula
---


\stylepiece{588}

\stylemailtitle

website details

\styleinfos

1.1

25.10.2009

\stylemail


so, i think the new website, even under construction, looks better than
the old one ....

therefore i activated it, the new wordpress site is now at:
\hyphenatedurl{http://1904.cc/aether}

to log into the admin area, you need to go to this adress:
\hyphenatedurl{http://1904.cc/aether/wp-admin}


the old one, for reference, is at \hyphenatedurl{http://1904.cc/aether-v2}

the live interface is still at the same place.

open the champaign!


best,
manuel
---


\stylepiece{589}

\stylemailtitle

reports of the 1st 1/2 day in residency :

\styleinfos ideacritik

26.10.2009

\stylemail


dear aethernautes,

daily reports will be sent to the list with contents of the day's
discussion. after initial group discussion we will split off into smaller
groups to focus on specific issues detailed below (tentative list). for
those remote participants, if you feel there is a topic that interests you
particularly and want to give feedback on - please tell us - we try to
include remote participation : ) we can do this over \stylerefmailchat{IRC}\stylerefroad{581}{593} for example.

currently present are:
boris, judy, manu, frauke, audrey

laure and paula will be joining us this evening.

we have arrived at \_\_\_\_\_-micro research today. we have enjoyed a superb
vedgetable vindaloo cooked by kati! re-energising us for the hours to come
: )

the main subjects of discussion we foresee are:
-website design (manu will give a presentation on the wordpress website at
14:00 tomorrow)
-technical development (specifics to be determined)
-PR (publication - contents, distribution, flipbook, prints)
-performance (narrative, formal stuff, content, nomadic idea,
storyboarding/script, sets/scenography) > directions? i.e. workshops,
festivals, vjing [proposals: "urban screen", pact, rote fabrik, okno,
mousonturm]
-financial - grants, sponsorships, agent/producer
-saturday's performance (who/what) - dr. mabuse still restaged

question: who would be interested in following the discussion over IRC
(irc.freenode.net \#aether9)? what timeframe and what discussion do you
think is of interest to you?

proposed schedule of the week:

tonight: preparation of slideshow for tomorow's presentation, initial
discussion about saturday's performance

tuesday:
offline till 14:00 - painting and prepararing gallery space.
14:00 : manu's presentation about website
19:00 : presentation

wednesday + thursday (schedule not yet determined)
[chloe and cym arrive in berlin]

friday :
11:00\textasciitilde{}14:00 : interviews with diana mccarthy
rest of the day prepare for saturday's performance

saturday >>> PERFORMANCE

sunday :
offline debriefing with diana mccarthy

that's all folks!

updates coming daily : )

frauke/audrey

---

\stylepiece{590}

\stylemailtitle

Fwd: Video-Performance, Rupert Goldsworthy Gallery

\styleinfos 1.1

26.10.2009

\stylemail


FYI, here is another short compiled version of our press-text, can be useful to send around:

*****


rupert goldsworthy gallery is premiering an aethereal performance, a wildly dispersed yet intimate collision of live sounds and images.

Developed by an online networked group of international visual artists/collectives working through Oceania, Asia, the Middle East, Europe, and into the Americas, aether9 is a matrix for video/audio performance, the foundations being  distributed authorship, cross-platform, and 'remote' performance of streaming audio and visual material to a nomadic web interface. 

>From October 26th to 31st, an intensive real-life residency of members of the aether9 group is held at at \_\_\_\_\_-micro research [berlin], where the artists meet in physical space for the first time.


further information:
\hyphenatedurl{http://1904.cc/aether/}
---


\useexternalfigure[/aN6g2qsTo1FAmV9R++OpQ==][../PAR-PERFO/2009-10-31\_Berlin/img/20091028\_192723\_radio.png]\medskip\hbox {\externalfigure[/aN6g2qsTo1FAmV9R++OpQ==][width=\getvariable{pageprops}{columnwidth},factor=max]}\blank[image]

\useexternalfigure[5K+Cmod0C4sTaMX1ZyH2Gg==][../PAR-PERFO/2009-10-31\_Berlin/img/20091028\_23h50m52s\_.jpg]\medskip\hbox {\externalfigure[5K+Cmod0C4sTaMX1ZyH2Gg==][width=\getvariable{pageprops}{columnwidth},factor=max]}\blank[image]

\stylepiece{593}

\stylemailtitle

[Fwd: daily reports - aetheric activity 29.10.2009]

\styleinfos ideacritik

29.10.2009

\stylemail


28.10.2009 - daily report aetheric activity

Morning discussions:

-\stylerefmailchat{IRC}\stylerefroadrwdonly{589} vs \stylerefmailchat{Skype}\stylerefroadfwdonly{594} discussion (what are the pros and cons)
-division into work groups:
        *the radio show group
        *the saturday performance group
        *PR/publication group

Radio show group:
aether9 was broadcasting on Herpstradio.de in the Berliner Runde radio
show at 19:00 CET @ HKW.
\hyphenatedurl{http://1904.cc/aether/2009/news/live-on-backyardradio-berlin/}
Link in Herpstradio:


\hyphenatedurl{http://herbstradio.org/plan/sendung/8614.html

\#Magazin-Berliner\_Runde-Aether9}

PR group:
the PR group discussed crucial ‘producibles’ that will be invaluable to
potential up and coming young and wealthy collectors. Such as, the deluxe
collector’s kit (both digital and tangible) amongst numerous other
novelties.

Publication:
an aether9 publication is soon to come, including documentation of
performances, interviews conducted this week in berlin and an essay by
Diana McCarthy. discussions are still going around edition and printing
issues.

Performance group:
Saturday, aether agents will be directed by doctor Nylon and her assistant
from the Rupert Goldworthy Gallery in an attempt to materialise the
Californian. This script is an interpretation of Doctor Mabuse by Fritz
Lang. Further script development to follow.


audrey/aether9 group


---


\stylepiece{594}

\stylemailtitle

performance group get your ass on IRC

\styleinfos ideacritik

29.10.2009

\stylemail


log in to \stylerefmailchat{IRC}\stylerefroad{593}{595} \_\_\_\_ !!!!!!!!!

---


\stylepiece{595}

\stylemailtitle

mid-way almost the end report

\styleinfos ideacritik

30.10.2009

\stylemail


yesterday (thursday 29th):

main groups were performance, future directions and \stylerefmailpatch{pd}\stylerefroadrwdonly{581} \stylerefmailpatch{Patch}\stylerefroadfwdonly{611}.

** we have bought aether9.com \& .org domains **

Performance:
the performance will reflect of what has 'impressed' the aether9 meeting
in berlin. for the first time in aether9 history the performance will be
partially streamed from a taxi speeding through berlin.

notes: we will insert more documentary in the fiction.

Future directions >>
festivals - residencies - workshops - tours - agent

Festivals:
*FLOSS Orientated festivals
Interested parties: audrey, Manuel.
*Generic media art festivals:
Presenting aether9 as an installation.
Interested parties: Manuel.
*Art fairs/biennale
Interested parties: Boris

Residencies:
Could be dedicated to development of specific aspects:
-       Software development (with some programmer giving technical support)
-       \stylerefmailpatch{puredata} patching
-       Mobile platform support (Symbian, mobile linux...)

workshops: [2 people minimum]
[*move to PD necessary + formal language and notation should be formulated]
*educational context (in art academies and universities)
-> the WORKSHOP PDF IS NOW READY TO BE CIRCULATED
*performative context (in festivals)
Interested parties: Manu, audrey, Judy, cym

speaking engagements:
Interested parties: Judy

tours:
A group of 3-4 aethernauts could be touring with a performance. Activity
in a Festival or a Workshop could be followed by a couple of other dates.

AGENT/Representation:
Should contact possible representants.
Interested parties: Judy.

Clothing of the performers / Visual identity / in-kind sponsorship
We will propose to fashion designers to sponsorize the aether9 group with
clothes, luggage.
Interested parties: cym, Judy

->PD patch:
working with dynabolic boot CD as way around 'shell' object problem. one
still uploaded to \stylerefmailtech{server}\stylerefroad{446}{617}! (micro success)

from the madly working aether9 team in berlin... also we are on \stylerefmailchat{IRC}\stylerefroad{594}{597} : )

---


\stylepiece{596}

\stylemailtitle

mid-way almost the end report

\styleinfos

alejo

30.10.2009

\stylemail


eia,

will try to follow the taxi stream!

On Oct 30, 2009, at 11:26 AM, ideacritik wrote:

> workshops: [2 people minimum]
> [*move to PD necessary + formal language and notation should be  
> formulated]

really good news

:)

>
> ->PD patch:
> working with dynabolic boot CD as way around 'shell' object problem.  
> one
> still uploaded to server! (micro success)
>

[shell] is buggy, i can tell from personal experience but just read  
the ml to confirm if in doubt.. i have replaced it in the locus sonus  
stream patched by [popen] that is part of the moonlib libraries.

other suggestions could use pdlua or pyext. it all depends how simple  
or complex is what you need to achieve.

> from the madly working aether9 team in berlin... also we are on  
> IRC : )

will try to jump in to catch up the action, but weather is nice here  
and this means last bike rides of the year for me.

saludos a los lovely anfitriones
/a
---


\stylepiece{597}

\stylemailtitle

I add names to the mailing message

\styleinfos Paula Vélez Bravo

30.10.2009

\stylemail


OFFICIAL STATEMENT
FOR IMMEDIATE RELEASE
AETHER AGENTS in Berlin
comming from Geneva, Rotterdam, New York, Brussels, Medellin,
or streaming remotely from Sydney, Yorba Linda

Scheduled for October 31st - 21:00 CET

Aether9, an \stylerefmailscience{experiment}\stylerefroad{548}{600} in collaborative realtime storytelling
is gathering souls in a dramatically online fashion.
BERLIN hosts a physical reunion,
a seance,
workshop,
conspiracy, residency, and eventually a performance.
wtf

ALL SOULS BERLIN >>>

Tommorow night, Saturday, Halloween,
as a culmination of a week long residencey @ \_\_\_\_\_-micro research
aether9 will present a live online performance before an audience in  
the rupert goldsworthy gallery in Berlin.

The wireless building of a mass mind, controlling aether9  
agents ...amongst Berliners unaware.
Aether9 presents an interpretation of Fritz Lang's Doktor Mabuse
streaming from a taxi speeding through Berlin.

Join us online at 21:00 CET --> \hyphenatedurl{http://1904.cc}

///

about aether9
Aether9 is a collaborative art project exploring the field of realtime  
video transmission. It was initiated in May 2007 during a workshop at  
the Mapping Festival in Geneva, Switzerland. Developed by an  
international group of visual artists and collectives working in  
different locations (Europe, North and South America) and  
communicating solely through the Internet, ther9 is a framework for  
networked video/audio performance, and the collaborative development  
of dramarturgical rules particular to Internet modes of \stylerefmailchat{communication}\stylerefroad{595}{606}.  
The system functions as an open platform for participants of any  
technical level to transmit imagery in real-time and interact through  
a structured narrative performance questioning the issues of presence/ 
absence, remote/local, identity and intimacy in the context of the  
electronic space.

\hyphenatedurl{http://1904.cc/aether/}

///

residence @ \_\_\_\_\_-micro research [berlin] \hyphenatedurl{http://www.1010.co.uk/}

/// PERFORMERS

/Judy Nylon /Audrey Samson /Paula Vélez /Chloé Cramer /Frauke Frech /Boris  
Kish /Laure Deselys /Cym /Manuel Schmalstieg /Christian Cruz /Vinny Bhagat
---


\useexternalfigure[PkUM6GM6i5PZcU/vbxZ6Hw==][../PAR-PERFO/2009-10-31\_Berlin/img/20091030\_15h50m52s\_berlin\_pd\_screenshot.png]\medskip\hbox {\externalfigure[PkUM6GM6i5PZcU/vbxZ6Hw==][width=\getvariable{pageprops}{columnwidth},factor=max]}\blank[image]

\useexternalfigure[FdNNVX4dkJkPy9PSIDTy9g==][../PAR-PERFO/2009-10-31\_Berlin/img/20091030\_15h50m55s\_hands01.jpg]\medskip\hbox {\externalfigure[FdNNVX4dkJkPy9PSIDTy9g==][width=\getvariable{pageprops}{columnwidth},factor=max]}\blank[image]

\stylepiece{600}

\styleinfoperfo
\startinfopar \perfotitle{All Souls}

\infopar 2009-10-31 00:00:00

\infopar Residencey @ _____-micro research

\infopar rupert goldsworthy gallery, Berlin (Germany)

\infopar Boris
Kish,
Chloé
Cramer,
Christiaan
Cruz,
cym,
Frauke
Frech,
ideacritik,
Judy
Nylon,
Laure
De
Selys,
Manuel
Schmalstieg,
Paula
Vélez,
Vinny
Bhagat\stopinfopar
\blackrule[color=black, width=60mm, height=0.5pt, depth=0mm]
\styleperfo
SATURDAY, HALLOWEEN, AS A CULMINATION OF A WEEK LONG RESIDENCEY @ \_\_\_\_\_-MICRO RESEARCH, AETHER9 PRESENTED A LIVE ONLINE PERFORMANCE BEFORE AN AUDIENCE IN THE RUPERT GOLDSWORTHY GALLERY IN BERLIN. THE WIRELESS BUILDING OF A MASS MIND, CONTROLLING AETHER9 AGENTS …AMONGST BERLINERS WHO ARE UNAWARE OF THEIR PRESENCE. AETHER9 PRESENTS AN INTERPRETATION OF FRITZ LANG’S DOKTOR MABUSE STREAMING FROM A TAXI SPEEDING THROUGH BERLIN.
\blackrule[color=black, width=60mm, height=0.5pt, depth=0mm]

\useexternalfigure[XIM84RebF2DC5l0mkLtGkA==][../PAR-PERFO/2009-10-31\_Berlin/img/20091031\_10h00m38s.png]\medskip\hbox {\externalfigure[XIM84RebF2DC5l0mkLtGkA==][width=\getvariable{pageprops}{columnwidth},factor=max]}\blank[image]

\useexternalfigure[wyExeQI7dcrEnp6SA+7fdQ==][../PAR-PERFO/2009-10-31\_Berlin/img/20091031\_10h06m00s.png]\medskip\hbox {\externalfigure[wyExeQI7dcrEnp6SA+7fdQ==][width=\getvariable{pageprops}{columnwidth},factor=max]}\blank[image]

\useexternalfigure[osS6WziBa1uUz89eOzTXOA==][../PAR-PERFO/2009-10-31\_Berlin/img/20091031\_11h09m56s.png]\medskip\hbox {\externalfigure[osS6WziBa1uUz89eOzTXOA==][width=\getvariable{pageprops}{columnwidth},factor=max]}\blank[image]

\useexternalfigure[2jmuLK4RP/8/EPgrNVOacQ==][../cover/20091031\_11h09m56s.png]\medskip\hbox {\externalfigure[2jmuLK4RP/8/EPgrNVOacQ==][width=\getvariable{pageprops}{columnwidth},factor=max]}\blank[image]


\page
\setuplayout[nomargins]
\placefigure[here,force]{none}{\externalfigure[/tmp/OSP\_AETHER9\_\_20091031\_11h09m58s\_FULL\_0.png][width=\paperwidth,height=\paperheight]}
\placefigure[here,force]{none}{\externalfigure[/tmp/OSP\_AETHER9\_\_20091031\_11h09m58s\_FULL\_1.png][width=\paperwidth,height=\paperheight]}
\placefigure[here,force]{none}{\externalfigure[/tmp/OSP\_AETHER9\_\_20091031\_11h09m58s\_FULL\_2.png][width=\paperwidth,height=\paperheight]}
\placefigure[here,force]{none}{\externalfigure[/tmp/OSP\_AETHER9\_\_20091031\_11h09m58s\_FULL\_3.png][width=\paperwidth,height=\paperheight]}
\page
\setuplayout[reset]

\stylepiece{606}

\stylemailtitle

flat joke and other thoughts from c3cil

\styleinfos chloé cramer

02.11.2009

\stylemail

myself. if you agree with this pattern (or would like to think about it with
me), it could be used for future developpment or for workshops. By the way,
I would be interested to lead workshops to guide performance developpment
part, without taking part myself. Would be interesting!
(In addition, I am now starting a project were 100 people are creating a
part of a parade, me and another guy being the artisitic coordinator who
will eventually "stage direct" and write what comes out from all these
imputs...) :

\_ NEEDS FOR A PERFORMANCE:
* one person doing only continuity \stylerefmailchat{chat}\stylerefroad{597}{624} and looking at the interface
* one person doing music only
* one voice person doing the narratives
* each image-uploader should deal 2 frames maximum, not more, in order to be
really aware and creative, and especially in order to be FASTER.

\_PREPARATION: how to write an efficient script step by step:
* brainstorm around a thema connected with smth actual, happening now.
think of images, not ideas. think of interaction of frames.
>> there are very efficient techniques of brainstorm. I learned some
recently and think we should try them on. you can have a first collection of
homogenoeous idea in not more than 3 hours.
* more accurate brainstorm with maximum 4 people, ending with 4 propositions
for a performance-
*1-2 persons, after discussion, write final proposition
* 1 person write the text (the story to be told, it can be a poem, whatever,
but must be entertaining for the lenght of the performance)
* others organise a general disposition of interface, coherent with the
story and esthetic/ image choices.
* rehearsal: absolutely necessary to do at least one with *everybody*.

thank you all and see you soon aetherians,
Chloé, analisa, c3cil and the others
---


\stylepiece{607}

\stylemailtitle

flat joke and other thoughts from c3cil

\styleinfos

christiaan cruz

04.11.2009

\stylemail


yes it was nice
even remote i felt really close to you all
especially at the rehearsal

but on to the issues to be resolved.

something has to be done about the uploading problems
it plagued us in lebanon and really killed the final minutes of berlin.

did we just have too many people viewing the website for Berlin?
I'm assuming this is the root of the problem.

i was able to stream only after restarting max and then deleting the
dummy (0kb) images that were writing incorrectly on our servers.
but this would only fix it for a minute. There would be write
errors immediately after. This was happening to those trying to
fix the locked hand and credits frames at the end of our performance.
There must be some way to verify the capacity of our servers and
anticipate the number of viewers we have for each performance.

the patch does give a red light and an 'error writing image" message.
however, there must be a way to prevent this from happening.

streaming for me worked fine during rehearsals when no one was viewing
the webpage. But it seems like now that we have a good audience
its impossible to perform because you can't stream when too many
people are viewing the interface website. It seems all of our
servers chrash during the performance from too much traffic.
Is my assumption correct? If not there must be some other way
to correct these upload issues during our performances. It can't
be on my end as my isp was not having any problems.

we have some time before manuel's \& vinny's shows. We should really
try to correct this issue before then.

-christiaan
---


\stylepiece{608}

\stylequoteblock{

\styleinfos

Laure Deselys

\stylequote



It feels like we are more in a digestive moment of our culture of images, narrative, questioning simultaneity in the global world, the real output is not yet there.
It feels also we are squeezed between the contradiction of doing real experimentation and being entertaining.

}

\stylepiece{609}

\stylemailtitle

meeting

\styleinfos Frauke Frech

18.11.2009

\stylemail


>  i would be in on the 21st. (next saturday)
>
> ///
>   

this saturday i'm available around noon/1pm.

in case loads of us can't be there we could also meet in smaller groups, no?
i really need to clear some things within our continuation and also just 
getting in touch with you again.

after this intense week it suddenly feels so withdrawn again...

hope an rdv works out soon

excited  frauke
---









\stylepiece{610}

\stylequoteblock{

\styleinfos

Paula Vélez

\stylequote



I'm happy we are still working with a slow frame rate when I see results of video high rate examples

}

\stylepiece{611}

\stylemailtitle

meeting contents

\styleinfos Vinny Bhagat

04.12.2009

\stylemail


hi Aethers ..

Sunday is ideal time for a meeting / checks / rehearsal .. [ please ,  
even if some of you are available ]

Monday performance in Adelaide , we are on at 8.30pm Adelaide time  
and play till 9.30 pm  , ie

Adelaide (Australia - South Australia) Monday, December 7, 2009 at  
8:30:00 PM
Brussels : Monday, December 7, 2009 at 11:00:00 AM

please calculate your local times .. checks will start earlier ..

So far the tests i have done , with the wi fi available in the venue  
[ 1-3 Mb - bandwidth ] and mobile broadband/ USB modem  ( varies but  
often less than equal to 1Mb bandwidth ) ... " \stylerefmailslow{slow}\stylerefroad{581}{622} is the word .. "

I will be using 3 computers in Adelaide with individual internet  
connections ..

1) Projecting Aethers interface :  i would suggest , please remove  
the audio stream .. Atleast for the interface that we use to stream  
in venue . At the moment , whats streaming on 1904.cc , takes long  
time to playback images properly .. there is a long buffering time ..
For me in Sydney , its totally fine , because i am on 15Mb+ bandwidth ..
For online audience we can make an interface that consists of all the  
content < which needs faster internet ..

2) To playback my AUDIO[ Syd]  in Adelaide  : What are your  
suggestions , i can use Ustream , Live stream or icecast/ Giss Tv  
setup using \stylerefmailpatch{pd}\stylerefroad{595}{617}  ..
What you suggest ?

3) Stream AV from the performance space in Adelaide

I will write you more soon , please reply, your thoughts n  
suggestions . Sometime today i will send invites  .. people in  
adelaide are excited and it has been advertised properly over there ..

i have been spending way too much time at work , so apologies for not  
writing much , i read all your emails , and maintain the circuit of  
thoughts , inside me ..

Talk to you all soon , beautiful friends ..
'
*bliss *
\textasciitilde{}vinny
---


\useexternalfigure[zXeqrFatteloxeYuVDb9DA==][../PAR-PERFO/2009-12-07\_Adelaide/img/20091207\_screenshot-18.png]\medskip\hbox {\externalfigure[zXeqrFatteloxeYuVDb9DA==][width=\getvariable{pageprops}{columnwidth},factor=max]}\blank[image]

\useexternalfigure[gmxQaY41B8RdNA6s1xV/gA==][../PAR-PERFO/2009-12-07\_Adelaide/img/20091207\_screenshot-13.png]\medskip\hbox {\externalfigure[gmxQaY41B8RdNA6s1xV/gA==][width=\getvariable{pageprops}{columnwidth},factor=max]}\blank[image]


\page
\setuplayout[nomargins]
\placefigure[here,force]{none}{\externalfigure[/tmp/OSP\_AETHER9\_\_20091207\_FULL\_screenshot-10\_0.png][width=\paperwidth,height=\paperheight]}
\placefigure[here,force]{none}{\externalfigure[/tmp/OSP\_AETHER9\_\_20091207\_FULL\_screenshot-10\_1.png][width=\paperwidth,height=\paperheight]}
\placefigure[here,force]{none}{\externalfigure[/tmp/OSP\_AETHER9\_\_20091207\_FULL\_screenshot-10\_2.png][width=\paperwidth,height=\paperheight]}
\placefigure[here,force]{none}{\externalfigure[/tmp/OSP\_AETHER9\_\_20091207\_FULL\_screenshot-10\_3.png][width=\paperwidth,height=\paperheight]}
\page
\setuplayout[reset]

\useexternalfigure[xPOPIUGKBOL49whVT6QdCA==][../PAR-PERFO/2009-12-07\_Adelaide/img/20091207\_screenshot-3.png]\medskip\hbox {\externalfigure[xPOPIUGKBOL49whVT6QdCA==][width=\getvariable{pageprops}{columnwidth},factor=max]}\blank[image]

\useexternalfigure[o2cfdTvRgnhiaYHVA/vR3A==][../cover/20091207\_screenshot-18.png]\medskip\hbox {\externalfigure[o2cfdTvRgnhiaYHVA/vR3A==][width=\getvariable{pageprops}{columnwidth},factor=max]}\blank[image]

\stylepiece{617}

\stylemailtitle

Adelaide performance

\styleinfos 1.1

07.12.2009

\stylemail


Hi all,
for the Adelaide performance in a couple of hours, some important points:

* the frameset has been modified -- i removed the two "weak servers", nr
4 and 5, that we struggled with during the berlin performance. at the
moment i write this, those frames should show black, with an occasional
\stylerefmailred{red}\stylerefroad{545}{3} or blue monochrome appearing.

If you recently accessed 1904.cc, you might use this direct link to make
sure your browser doesn't show you an old version:
\hyphenatedurl{http://1904.cc/live/frameset\_09\_adelaide/}

* this means of course that i updated the \stylerefmailtech{server}\stylerefroad{595}{619}-list, included with the
\stylerefmailpatch{Max}\stylerefroadfwdonly{619} \stylerefmailpatch{Patch}\stylerefroadrwdonly{611} -- the new list is included with the newest patch, nr 400.

You find it here:
\hyphenatedurl{http://1904.cc/~aether/kode/max\_image\_upload/}

The password of the z7/rar archive is the "old" aether password, the one
that's easier to remember.

Meet you in 10 hours:)


Best,
Manuel
---


\stylepiece{618}

\styleinfoperfo
\startinfopar \perfotitle{aether9}

\infopar 2009-12-07 20:00:00

\infopar COMA (Creative Original Music Adelaide)

\infopar The Wheatsheaf Hotel, Adelaide (Australia)

\infopar Christiaan
Cruz,
Dhanya
Pilo,
ideacritik,
Manuel
Schmalstieg,
Maria
Fava,
Vinny
Bhagat\stopinfopar
\blackrule[color=black, width=60mm, height=0.5pt, depth=0mm]
\styleperfo
AETHER9 IMPROVISED JAM SESSION IN LIVE COLLABORATION WITH SHIVNAKAUN (VINNY BHAGAT).
\blackrule[color=black, width=60mm, height=0.5pt, depth=0mm]

\stylepiece{619}

\stylemailtitle

report: Adelaide performance + future

\styleinfos 1.1

08.12.2009

\stylemail


Hi all,

Some short report about the Adelaide performance, to which the Aether9
team contributed:

The project was coordinated by Vinny Bhagat (aka Shivnakaun), who was
streaming live electro-acoustics from his studio in Sydney (transmitting
through giss.tv from a \stylerefmailpatch{puredata}\stylerefroadrwdonly{617} \stylerefmailpatch{Patch}\stylerefroadfwdonly{624}).

The stream was (supposedly, I didn't hear it) accompanied live
on-location in Adelaide (at The Wheatsheaf Hotel, 39 George St
Thebarton) by Kym Gluyas on saxophone. The aether9 interface was
projected and visible to the live audience.

The Aethernauts who generated the visuals:
Christiaan Cruz, Dhanya Pilo, ideacritik, Manuel Schmalstieg, Maria Fava

This was the first participation in the aether9 project for Maria Fava
(aka eskoitaus), located in Nice (N3krozoft members will remember her
from the CeC festival last February, where she was VJing).

The visuals were fully improvised. The performers decided to restrict
themselves to a black/white color palette for better graphic coherence.
Also, due to the little number of Aethernauts, only 4-5 frames of the
interface were used.

We didn't encounter any technical / \stylerefmailtech{upload}\stylerefroadrwdonly{617} speed issues this time.
Except ideacritik who was still unable to upload to frame 5, although
the \stylerefmailtech{server}\stylerefroadfwdonly{624} feeding this frame has been changed. Very mysterious.

Links to the COMA website (Creative Original Music Adelaide)

\hyphenatedurl{http://www.coma.net.au/news.php}
\hyphenatedurl{http://www.coma.net.au/coma-gigs-past.php}
\hyphenatedurl{http://www.coma.net.au/band\_page.php?Band\_ID=49}

Link to the performance page I created:
\hyphenatedurl{http://1904.cc/aether/2009/news/adelaide-performance/}

Best regards,
Manuel

---



\stylepiece{620}

\stylemailtitle

link PD section code + IRC in contact

\styleinfos

ideacritik

11.12.2009

\stylemail


PD link??
je cherchais a ajouter PD code dans la section CODE
\hyphenatedurl{(http://bin.viator.si/?page=AetherOne} - c'est pas ca - une idee ou il a
transpose ca? je ne trouve pas)

i added IRC channel in contact:
\hyphenatedurl{http://1904.cc/aether/contact/}
(even though its not really 'our' channel)

idea

---




\stylepiece{621}

\stylequoteblock{

\styleinfos

Christiaan Cruz

\stylequote

 

I don't think there was an official title to our images in
Australia but it seemed linked to loud coffee cups

}

\stoptext

\stopproduct