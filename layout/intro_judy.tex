\part[2012]{2012}
\marking[P]{2012}

\stylepiece{671}
\stylequote
“The aether9 performance was like a hypnotic late night drive... a wall of reality changing softly, gently, always”. 
\styleinfos
PJ Norman
\stylemailjudy
Aether9 developed four different kinds of performances with distributed authorship: Theatrical adaptations (Beckett), myth based storytelling (Little Red Riding Hood), duration and simultaneity based pieces for VJ work and installations, and self-scripted stories designed to test our software developments (All Souls). Each style was a venture into the dramatic possibilities for low budget ensemble streamed performance. We built a tabletop alternative to the glossy high production values of advertiser-funded spectacles, playing with the idea of a ‘people to people’ broadcast that was all-inclusive. Everything was shared, previously owned or built by one of us. Our production expenses were kept to a minimum. Aether9 stories have no resolution but are rather like a clip of believable moods or moments strung together and passed back and forth among performers, working like the fingers suspending a cat's cradle. Once put down, in a dormant state, a cat's cradle is a loop of string that may be picked up and reanimated at any time. The aether9 interface, looped at the end of a performance, is left idling like a dream engine in neutral gear. 
   
We hacked into traffic cams, surf cams, or private surveillance, using the image compositionally like a filmic B-roll within the 9-frame multicast to suggest place and overcome the claustrophobic atmosphere of webcasts, where you are tethered to camera and computer. The hacked cams also functioned in the story, as does a tiny figure in a Chinese landscape painting, to remind the viewer of the relationship of the talking head or the psychic landscape to the larger external universe. Aether9 provided a framework for improvising and translating the techniques used by the studios and auteurs in early filmmaking. We used hand held kaleidoscopic lens and lenticular postcards were moved in front of the camera. The doll house in LRRH (Little Red Riding Hood) provided eight different movable tabletop sets, controlled with the knees and feet while both hands are on the camera, there were delightful wolf shadows cast by paper cutouts attached to chopsticks and naturally, there were masks. The functionality of the aether9 software, downloadable from our website, was frequently improved after each performance, to stretch the mobility or facilitate operating requirements of the next show. Without compromising our low tech mandate, while side stepping the limitations put in place by international telecoms, we familiarized ourselves with open code, open culture, and orphaned works. Performing with aether9 was before all else FUN. Interpreting the scripts, sourcing props, writing adaptive code and trying things out at rehearsals was something we each looked forward to. To have a virtual gang of people with whom you only interact creatively is the modern dream in a complicated life. That we sometimes managed three shows a month is telling of how important aether9 became to each performer.
 
There was really no choice but to develop a ‘hands on’ methodology to using the early software prototypes in live work. You had a lag time of about four seconds before you would see what you had just done. You must anticipate when to change the image the same way you would feel when to 'ride the beat late' in jazz. It reminded me of singing through a dub setting on a ‘space-station’ in the old days. The 9-box screen works with the eye, as in the Nicolas Roeg film "The Man Who Fell To Earth" to filter varied information, making a single (or weave of multiples) thought line, composed in the juxtaposition, but coming out of the interstices of the images. During performances this helped set up what may be called a story line or time progression. By setting a low upload speed, we could compensate for different deliveries of bandwidth we confronted in various locations. We realized that at some point in each performance we had to step forward and identify our personal location and sometimes be actually seen for a moment because the audience did not realize we were working live otherwise. We have used a 6-box screen and even four, but the having nine, the basis of sacred geometry, works best because of the way the eye is wired to the brain. More successfully, we started using both color blocks, referencing 'color field painting', and intermixed blocks of action: the ability to tune color and transparency in the patch (software), to relate the two was crucial. To have... say, six screens throw up solid black, while only three people displayed action to suggest cross screen slow motion movement, required scripting that borrowed from dance notation and simulated animation. We used Skype/IRC as a teleprompter; the chosen 'director of remote coordination' for each performance typing out upcoming script cues via networked text messages. The sound source was limited to one point of origin, to conserve bandwidth and avoid chaotic ambient mixes. There has usually been an aether9 agent present at every performance to introduce the performance and monitor the connection to the servers hosting the aether9 interface. On the occasions when there is an added layer of acting or musical performance on stage, there are two directors (one for the venue stage and one remote) coordinating the performance.
 
Aether9 was never easily tied to one locality by topic, or nation. We avoided the extremes of the confrontational in religion, politics and sexuality.  By passing the lead during performances we could VJ superhumanly long durational performances as we spanned many time zones. Both the performers and the downlink locations moved like a stretched net. All one needed to perform was a camera, a computer with a (Max/Jitter or Pure Data) patch and web access. It’s an update of the quote from Jean-Luc Godard "all you need for a movie is a girl and a gun" . The expanded cinema of aether9 related the witness to the poet, allowing room for experimentation and impermanence because aether9 was nomadic, shared and autonomous. 
 
It was from the beginning our intention to share everything as we go. We offered workshops and continually involved other creatives in the performances, working alongside of us locally as well as at the downlink point of the aether9 show. Probably the performance that felt the most collaboratively immersed with local artists, was the show in Beirut that took place outside on Hamra Street at the end of Ramadan. Here we followed an English script among ourselves but gave over the creation of the story's Arabic text, storyline, and live music to local artists. The entire story spun on love and the view of a city where every rooftop is covered with aerials. 
 
Our chosen language was English although only very few of us were native speakers; among us we were holding every passport necessary to perform worldwide without visas and all necessary equipment could fit in carry-on luggage. However, our choice to reach the more remote audiences, those we could not easily otherwise have interacted with, put us out of sync with the largest arts organizations which have become more mainstream and less internationally focused. We thank the forward thinking residencies, gallerists and institutions that reached out to us, without making us conform to grant criteria (always tied to either residence or nationality). It was extremely important psychologically to have a local viewing point and a physical audience when nothing was anchored in a single time or place.
 
Human behavioral dynamics are the third leg of the support in the development of media practices for the arts. Collaborating with people from different generations and different backgrounds starts off with a difficulty in a long distance exchange of revelatory experience based truths. The slow but ever-changing performances that aether9 produced were meditations for the performers. The performances were sometimes loosely scripted, spontaneous, but there was a graphic score for overall visuals from which we improvised. We often prepared together via IRC/Skype chats by compiling text lists of symbolic objects like clocks, sea shells, or by selecting image fields like the color, royal blue or silver foil and worked these objects on camera within the personal landscapes of our settings around the world. Using instant messages in this way, language becomes gestural; grammar and spelling are never corrected unless meaning is lost or the text is shown to the audience. There were times when conversation switched to French if just the French speakers were on line. There was an arc of time over which we got to know each other, without exchanging personal details, revealing ourselves by the selection of objects and showing our workspaces on camera. Slowly unveiling.
 
The Berlin residency was the first time many of us met after working together for about two years. I still have not met all the aether agents, but Berlin provided a surprisingly smooth transition to working face to face. This was also the first chance we had to use 'internet-on-the-go' USB keys; we scripted on the spot to test our first nomadic live transmission. Watching the characters moving across Berlin in a taxi heading toward us... everything they were seeing was streamed to the gallery wall in real-time; it was a breakthrough moment for future low-tech theatrical scripting. Most of the performances we scripted 'in house' were designed to test new ideas about how we might use our software or refine the software after failures. All of these pieces were performed only once. The only two scripts we performed more than once were the adaptation of Beckett and LRRH, the decoding of the Little Red Riding Hood myth.
 
In preparing to do something with Beckett’s 1977 "Ghost Trio" teleplay, for BBC2, we considered that while he worked on it in London he was himself immersed in the ground zero of punk. Was this a melancholy memory of youth or another impending apocalypse at this point in his life? We watched a 1965 super 8 that has now vanished from the web, filmed in a empty run down ‘old law’ tenement apartment in NYC. It was written by Beckett, starred Buster Keaton and a Chihuahua and director Alan Schneider must have been there too. Who did what off camera; who knows. It blew the dust off some of the sacredness of Beckett’s work. He was game for experiments. We were true to his script except where the technology available to us was not taken into account in his direction. He might have appreciated Beethoven’s Fifth Piano Trio dubbed and mixed with NYC traffic horns, leaking through from the street, all the way to the Oude Kerk in Amsterdam or having the wet child of his ‘spirit world’ streamed from another continent. Again, who knows?
 
Aether9, from the beginning was offering inclusion to participants at differing stages in the development of their art practice and technical proficiency. We had very different skill sets and we all wanted aether9 to be truly multidisciplinary, showing the ‘human hand’ in our creation: including sculpture and drawings on camera, sound composition, authoring both code and script, and all translation was handled among us. Interest in our progress was facilitated by a FB group, aether9's mailing list, news links from member's sites, our own aether9.org wiki, as well as venue/gallery listings. We gave interviews about aether9 within the interested community; but generally we rushed headlong into performance without stopping to merchandise anything like a T-shirt, and spent very little time asking for money. Time was of the essence. As video conferencing and communication over distance became commonplace and ubiquitous, aimed at families spread over several localities or corporate tele-conferencing, the novelty that attracted us to this project slowly faded. Today multiple stream video conferencing is a pay-to-play offering on Skype Premium. Over half of us have changed our countries of residence and have gone on to other projects in emerging media. We remain in touch; our experiments in communication technology as aether9 have expanded our imaginations and shaped our dramaturgical considerations.
 
\setupindenting[no]
\styleinfos
Judy Nylon
